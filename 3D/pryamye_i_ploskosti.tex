\chapter{Прямые и плоскости}

\section{Определение положения плоскости}

\begin{wrapfigure}{r}{34 mm}
\centering
\includegraphics{mppics/s-ris-1}
\caption{}\label{1938/s-ris-1}
\end{wrapfigure}

\paragraph{Изображение плоскости.}\label{1938/s2}
В обыденной жизни многие предметы, поверхность которых напоминает геометрическую плоскость, имеют форму прямоугольника: переплёт книги, оконное стекло, поверхность письменного стола и тому подобное.
При этом если смотреть на эти предметы под углом и с большего расстояния, то они представляются нам имеющими форму параллелограмма.
Поэтому принято изображать плоскость на рисунке в виде параллелограмма.

Плоскость обычно обозначают одной буквой, например «плоскость $M$» (рис.~\ref{1938/s-ris-1}).

\paragraph{Основные свойства плоскости.}\label{1938/s3}
Укажем следующие свойства плоскости, которые принимаются без доказательства, то есть являются аксиомами:

1) \emph{Если две точки прямой принадлежат плоскости, то и каждая точка этой прямой принадлежит плоскости.}

2) \emph{Если две плоскости имеют общую точку, то они пересекаются по прямой, проходящей через эту точку.}

3) \emph{Через всякие три точки, не лежащие на одной прямой, можно провести плоскость, и притом только одну.}


\paragraph{} \label{1938/s4}
\so{Следствия}. Из последнего предложения можно вывести следствия:

1) \emph{Через прямую и точку вне её можно провести плоскость (и только одну).} Действительно, точка вне прямой вместе с какими-нибудь двумя точками этой прямой составляют три точки, через которые можно провести плоскость (и притом одну).

2) \emph{Через две пересекающиеся прямые можно провести плоскость (и только одну).} Действительно, взяв точку пересечения и ещё по одной точке на каждой прямой, мы будем иметь три точки, через которые можно провести плоскость (и притом одну).

3) \emph{Через две параллельные прямые можно провести только одну плоскость.} Действительно, параллельные прямые, по определению, лежат в одной плоскости;
эта плоскость единственная, так как через одну из параллельных и какую-нибудь точку другой можно провести не более одной плоскости.


\paragraph{Вращение плоскости вокруг прямой.}\label{1938/s5} 
\textbf{\emph{Через каждую прямую в пространстве можно провести бесчисленное множество плоскостей.}}

\begin{wrapfigure}{o}{31 mm}
\centering
\includegraphics{mppics/s-ris-2}
\caption{}\label{1938/s-ris-2}
\end{wrapfigure}

В самом деле, пусть дана прямая $a$ (рис.~\ref{1938/s-ris-2}).
Возьмём какую-нибудь точку $A$ вне её.
Через точку $A$ и прямую $a$ проходит единственная плоскость (§~\ref{1938/s4}).
Назовём её плоскостью $M$.
Возьмём новую точку $B$ вне плоскости $M$.
Через точку $B$ и прямую $a$ в свою очередь проходит плоскость.
Назовём её плоскостью $N$.
Она не может совпадать с $M$, так как в ней лежит точка $B$, которая не принадлежит плоскости $M$.
Мы можем далее взять в пространстве ещё новую точку $C$ вне плоскостей $M$ и $N$.
Через точку $C$ и прямую $a$ проходит новая плоскость.
Назовём её~$P$.
Она не совпадает ни с $M$, ни с $N$, так как в ней находится точка $C$, не принадлежащая ни плоскости $M$, ни плоскости $N$.
Продолжая брать в пространстве всё новые и новые точки, мы будем таким путём получать всё новые и новые плоскости, проходящие через данную прямую $a$.
Таких плоскостей будет бесчисленное множество.

Все эти плоскости можно рассматривать как различные положения одной и той же плоскости, которая вращается вокруг прямой $a$.

\paragraph{Задачи на построение в пространстве.}\label{1938/s6}
Все построения, которые делались в планиметрии, выполнялись в одной плоскости при помощи чертёжных инструментов.
Для построений в пространстве чертёжные инструменты становятся уже непригодными, так как чертить фигуры в пространстве невозможно.
Кроме того, при построениях в пространстве появляется ещё новый элемент — \rindex{плоскость}\textbf{плоскость}, построение которой в пространстве нельзя выполнять столь простыми средствами, как построение прямой на плоскости.

Поэтому при построениях в пространстве необходимо точно {определить, что значит выполнить то или иное построение} и, в частности, что значит построить плоскость в пространстве.
Во всех построениях в пространстве мы будем предполагать, что:

1) плоскость может быть построена, если найдены элементы, определяющие её положение в пространстве (§§~\ref{1938/s3} и \ref{1938/s4}), то есть что мы умеем построить плоскость, проходящую через три данные точки, через прямую и точку вне её, через две пересекающиеся или две параллельные прямые;

2) если даны две пересекающиеся плоскости, то дана и линия их пересечения, то есть мы умеем найти линию пересечения двух плоскостей;

3) если в пространстве дана плоскость, то мы можем выполнять в ней все построения, которые выполнялись в планиметрии.

{Выполнить какое-либо построение в пространстве — это значит свести его к конечному числу только что указанных основных построений.}
При помощи этих основных задач можно решать и задачи более сложные.

В этих предложениях и решаются задачи на построение в стереометрии.

\paragraph{Пример задачи на построение в пространстве.}\label{1938/s7}\ 

\so{Задача}.
\emph{Найти точку пересечения данной прямой $a$ \emph{(рис.~\ref{1938/s-ris-3})} с данной плоскостью~$P$.}

\begin{wrapfigure}{o}{55 mm}
\centering
\includegraphics{mppics/s-ris-3}
\caption{}\label{1938/s-ris-3}
\end{wrapfigure}

Возьмём на плоскости $P$ какую-либо точку $A$.
Через точку $A$ и прямую $a$ проводим плоскость $Q$.
Она пересекает плоскость $P$ по некоторой прямой $b$.
В плоскости $Q$ находим точку $C$ пересечения прямых $a$ и $b$.
Эта точка и будет искомой.

Если прямые $a$ и $b$ окажутся параллельными, то задача не будет иметь решения.



\section{Параллельные прямые и плоскости}

\subsection*{Параллельные прямые}

\paragraph{Предварительное замечание.}\label{1938/s8}
Две прямые могут быть расположены в пространстве так, что через них нельзя провести плоскость.
Возьмём, например (рис.~\ref{1938/s-ris-4}), две такие прямые $AB$ и $DE$, из которых одна пересекает некоторую плоскость $P$, а другая лежит на ней, но не проходит через точку ($C$) пересечения первой прямой и плоскости~$P$.
Через такие две прямые нельзя провести плоскость, потому что в противном случае через прямую $DE$ и точку $C$ проходили бы две различные плоскости:
одна $P$, пересекающая прямую $AB$, и другая, содержащая её, а это невозможно (§~\ref{1938/s3}).

\begin{wrapfigure}{o}{44 mm}
\centering
\includegraphics{mppics/s-ris-4}
\caption{}\label{1938/s-ris-4}
\end{wrapfigure}

Две прямые, не лежащие в одной плоскости, конечно, не пересекаются, сколько бы их ни продолжали;
однако их не называют параллельными, оставляя это название для таких прямых, которые, \so{находясь в одной плоскости}, не пересекаются, сколько бы их ни продолжали.

Две прямые, не лежащие в одной плоскости, называются \rindex{скрещивающиеся прямые}\textbf{скрещивающимися}.

\subsection*{Прямая и плоскость, параллельные между собой}

\paragraph{}\label{1938/s9}
\so{Определение}. Плоскость и прямая, не лежащая в этой плоскости, называются \rindex{параллельность}\textbf{параллельными}, если они не пересекаются, сколько бы их ни продолжали.
При этом плоскость считается параллельной любой прямой в ней лежащей.

\begin{wrapfigure}{r}{55 mm}
\vskip-6mm
\centering
\includegraphics{mppics/s-ris-5}
\caption{}\label{1938/s-ris-5}
\end{wrapfigure}

\paragraph{}\label{1938/s10}
\mbox{\so{Теорема}.} \textbf{\emph{Если прямая}} ($AB$, рис.~\ref{1938/s-ris-5}) \textbf{\emph{параллельна какой-нибудь прямой}} ($CD$), \textbf{\emph{расположенной в плоскости}} ($P$), \textbf{\emph{то она параллельна самой плоскости.}}

Если $AB$ лежит в $P$, то $AB\z\parallel P$ по определению.
Значит можно предположить, что $AB$ не лежит в~$P$.

Проведём через $AB$ и $CD$ плоскость $R$ и предположим, что прямая $AB$ где-нибудь пересекается с плоскость~$P$.
Тогда точка пересечения, находясь на прямой $AB$, должна принадлежать также и плоскости $R$, на которой лежит прямая $AB$, в то же время точка пересечения, конечно, должна принадлежать и плоскости~$P$.
Значит, точка пересечения, находясь одновременно и на плоскости $R$, и на плоскости $P$, должна лежать на прямой $CD$, по которой пересекаются эти плоскости;
следовательно, прямая $AB$ пересекается с прямой $CD$.
Но это невозможно, так как по условию $AB\parallel CD$.
Значит, нельзя допустить, что прямая $AB$ пересекалась с плоскостью $P$, и потому $AB\parallel P$.

\paragraph{}\label{1938/s11}
\so{Теорема}. \textbf{\emph{Если плоскость}} ($R$, рис.~\ref{1938/s-ris-5}) \textbf{\emph{проходит через прямую}} ($AB$), \textbf{\emph{параллельную другой плоскости}} ($P$), \textbf{\emph{и пересекает эту плоскость, то линия пересечения}} ($CD$) \textbf{\emph{параллельна первой прямой}} ($AB$).

Действительно, во-первых, прямая $CD$ лежит в одной плоскости с прямой $AB$, во-вторых, эта прямая не может пересечься с прямой $AB$, потому что в противном случае прямая $AB$ пересекалась бы с плоскостью $P$, что невозможно.


{

\begin{wrapfigure}[21]{o}{34 mm}
\vskip-6mm
\centering
\includegraphics{mppics/s-ris-6}
\caption{}\label{1938/s-ris-6}
\bigskip
\includegraphics{mppics/s-ris-7}
\caption{}\label{1938/s-ris-7}
\end{wrapfigure}

\paragraph{}\label{1938/s12}
\mbox{\so{Следствие} 1.}
\emph{Если прямая} ($AB$, рис.~\ref{1938/s-ris-6}) \emph{параллельна каждой из двух пересекающихся плоскостей} ($P$ и $Q$), \emph{то она параллельна линии их пересечения} ($CD$).

Проведём плоскость через $AB$ и какую-нибудь точку $M$ прямой $CD$.
Эта плоскость должна пересечься с плоскостями $P$ и $Q$ по прямым, параллельным $AB$ и проходящим через точку $M$.
Но через точку $M$ можно провести только одну прямую, параллельную $AB$;
значит, две линии пересечения проведённой плоскости с плоскостями $P$ и $Q$ должны слиться в одну прямую.
Эта прямая, находясь одновременно на плоскости $P$ и на плоскости $Q$, должна совпадать с прямой $CD$, по которой плоскости $P$ и $Q$ пересекаются;
значит, $CD \parallel AB$.

\paragraph{}\label{1938/s13}
\mbox{\so{Следствие} 2.}
\emph{Если две прямые} ($AB$ и $CD$, рис.~\ref{1938/s-ris-7}) \emph{параллельны третьей прямой} ($EF$), \emph{то они параллельны между собой.}

Если прямые $AB$, $CD$ и $EF$ лежат в одной плоскости, то $AB\parallel CD$.
Остаётся рассмотреть случай когда прямые $AB$, $CD$ и $EF$ не лежат в одной плоскости.

Проведём плоскость $M$ через параллельные прямые $AB$ и $EF$.
Так как $CD \parallel EF$, то $CD \parallel M$ (§~\ref{1938/s10}).

}

Проведём также плоскость $N$ через $CD$ и некоторую точку $A$ прямой $AB$.
Так как $EF\z\parallel CD$, то $EF\parallel N$.
Значит, плоскость $N$ должна пересечься с плоскостью $M$ по прямой, параллельной $EF$ (§~\ref{1938/s11}) и в то же время проходящей через точку $A$.
Но в плоскости $M$ через $A$ проходит единственная прямая, параллельная $EF$, а именно прямая $AB$.
Следовательно, плоскость $N$ пересекается с $M$ по прямой $AB$, значит, $CD \parallel AB$.


\subsection*{Параллельные плоскости}

\paragraph{}\label{1938/s14}
\so{Определение}.
Две плоскости называются \rindex{параллельность}\textbf{параллельными}, если они не пересекаются, сколько бы их ни продолжали.
При этом плоскость считается параллельной самой себе.

\begin{figure}[!ht]
\centering
\includegraphics{mppics/s-ris-8}
\caption{}\label{1938/s-ris-8}
\end{figure}

\paragraph{}\label{1938/s15}
\so{Теорема}. \textbf{\emph{Если две пересекающиеся прямые}} ($AB$ и $AC$, рис.~\ref{1938/s-ris-8}) \textbf{\emph{одной плоскости}} ($P$) \textbf{\emph{соответственно параллельны двум прямым}} ($A_1B_1$ и $A_1C_1$) \textbf{\emph{другой плоскости}} ($Q$), \textbf{\emph{то эти плоскости параллельны}}.

Прямые $AB$ и $AC$ параллельны плоскости $Q$ (§~\ref{1938/s10}).

\begin{wrapfigure}{r}{34 mm}
\centering
\includegraphics{mppics/s-ris-9}
\caption{}\label{1938/s-ris-9}
\end{wrapfigure}

Допустим, что плоскости $P$ и $Q$ пересекаются по некоторой прямой $DE$ (рис.~\ref{1938/s-ris-8}).
В таком случае $AB \parallel DE$ и $AC \parallel DE$ (§~\ref{1938/s11}).
Таким образом, в плоскости $P$ через точку $A$ проходят две прямые $AB$ и $AC$, параллельные прямой $DE$, что невозможно.
Значит, плоскости $P$ и $Q$ не пересекаются.

\paragraph{}\label{1938/s16}
\mbox{\so{Теорема}.} \textbf{\emph{Если две параллельные плоскости}} ($P$ и $Q$, рис.~\ref{1938/s-ris-9}) \textbf{\emph{пересекаются третьей плоскостью}} ($R$), \textbf{\emph{то линии пересечения}} ($AB$ и $CD$) \textbf{\emph{параллельны.}}

Действительно, во-первых, прямые $AB$ и $CD$ находятся в одной плоскости ($R$);
во-вторых, они не могут пересечься, так как в противном случае пересекались бы плоскости $P$ и $Q$, что противоречит условию.


\paragraph{}\label{1938/s17} \mbox{\so{Теорема}.}
\textbf{\emph{Отрезки параллельных прямых}} ($AC$ и $BD$, рис.~\ref{1938/s-ris-9}), \textbf{\emph{заключённые между параллельными плоскостями}} ($P$ и $Q$), \textbf{\emph{равны.}}


Через параллельные прямые $AC$ и $BD$ проведём плоскость $R$;
она пересечёт плоскости $P$ и $Q$ по параллельным прямым $AB$ и $CD$;
следовательно, фигура $ABDC$ есть параллелограмм, и потому $AC = BD$.

\paragraph{}\label{1938/s18}
\mbox{\so{Теорема}.} \textbf{\emph{Два угла}} ($BAC$ и $B_1A_1C_1$, рис.~\ref{1938/s-ris-10}) \textbf{\emph{с соответственно параллельными и одинаково направленными сторонами равны и лежат в параллельных плоскостях}} ($P$ и $Q$).

\begin{wrapfigure}{o}{44 mm}
\centering
\includegraphics{mppics/s-ris-10}
\caption{}\label{1938/s-ris-10}
\end{wrapfigure}

Что плоскости $P$ и $Q$ параллельны, было доказано выше (§~\ref{1938/s15});
остаётся доказать, что углы $A$ и $A_1$ равны.

Отложим на сторонах углов произвольные, но равные отрезки $AB\z=A_1B_1$, $AC=A_1C_1$ и проведём прямые $AA_1$, $BB_1$, $CC_1$, $BC$ и $B_1C_1$.
Так как отрезки $AB$ и $A_1B_1$ равны и параллельны, то фигура $ABB_1A_1$ есть параллелограмм;
поэтому отрезки $AA_1$ и $BB_1$ равны и параллельны.
По той же причине равны и параллельны отрезки $AA_1$ и $CC_1$.
Следовательно, $BB_1\parallel CC_1$ и $BB_1=CC_1$ (§§~\ref{1938/s13} и \ref{1938/s17}).
Поэтому $BC=B_1C_1$ и $\triangle ABC=\triangle A_1B_1C_1$ (по трём сторонам);
значит, $\angle A=\angle A_1$.

\subsection*{Задачи на построение}

\begin{wrapfigure}{o}{30 mm}
\centering
\vskip-0mm
\includegraphics{mppics/s-ris-11}
\caption{}\label{1938/s-ris-11}
\end{wrapfigure}

\paragraph{}\label{1938/s19}
\emph{Через точку} ($A$, рис.~\ref{1938/s-ris-11}), \emph{расположенную вне данной прямой} ($a$), \emph{в пространстве провести прямую, параллельную данной прямой}~($a$).


\medskip

\mbox{\so{Решение}.}
Через прямую $a$ и точку $A$ проводим плоскость $M$.
В этой плоскости строим прямую $b$, параллельную прямой $a$.

Задача имеет единственное решение.
В самом деле, искомая прямая должна лежать с прямой $a$ в одной плоскости.
В этой же плоскости должна находиться точка $A$, через которую проходит искомая прямая.
Значит, эта плоскость должна совпадать с $M$.
Но в плоскости $M$ через точку $A$ можно провести только одну прямую, параллельную прямой~$a$.

\paragraph{}\label{1938/s20}
\emph{Через данную точку} ($A$, рис.~\ref{1938/s-ris-12}) \emph{провести плоскость, параллельную данной плоскости} ($P$), \emph{не проходящей через точку} $A$.

\begin{wrapfigure}{o}{44 mm}
\centering
\includegraphics{mppics/s-ris-12}
\caption{}\label{1938/s-ris-12}
\end{wrapfigure}

\mbox{\so{Решение}.}
Проводим на плоскости $P$ через какую-либо точку $B$ две какие-либо прямые $BC$ и $BD$.
Построим две вспомогательные плоскости: плоскость $M$ — через точку $A$ и прямую $BC$ и плоскость $N$ — через точку $A$ и прямую $BD$.
Искомая плоскость, параллельная плоскости $P$, должна пересечь плоскость $M$ по прямой, параллельной $BC$, а плоскость $N$ — по прямой, параллельной $BD$ (§~\ref{1938/s16}).

Отсюда вытекает такое построение: через точку $A$ проводим в плоскости $M$ прямую $AC_1\parallel BC$, а в плоскости $N$ прямую $AD_1\parallel BD$.
Через прямые $AC_1$ и $AD_1$ проводим плоскость $Q$.
Она и будет искомой.
В самом деле, стороны угла $D_1AC_1$, расположенного в плоскости $Q$, параллельны сторонам угла $DBC$, расположенного в плоскости~$P$.
Следовательно, $Q\parallel P$.

Так как в плоскости $M$ через точку $A$ можно провести лишь одну прямую, параллельную $BC$, а в плоскости $N$ через точку $A$ лишь одну прямую, параллельную $BD$, то задача имеет единственное решение.
Следовательно, через каждую точку пространства можно провести единственную плоскость, параллельную данной плоскости.

\paragraph{}\label{1938/s21}
\emph{Через данную прямую} ($a$, рис.~\ref{1938/s-ris-13}) \emph{провести плоскость, параллельную другой данной прямой} ($b$).

\begin{wrapfigure}{o}{54 mm}
\centering
\includegraphics{mppics/s-ris-13}
\caption{}\label{1938/s-ris-13}
\end{wrapfigure}

\mbox{\so{Решение}.}
1-й \so{случай}.
Прямые $a$ и $b$ не параллельны.
Через какую-нибудь точку $A$ прямой $a$ проводим прямую $b_1$, параллельную $b$;
через прямые $a$ и $b_1$ проводим плоскость.
Она и будет искомой (§~\ref{1938/s10}).
Задача имеет в этом случае единственное решение.

2-й \so{случай}.
Прямые $a$ и $b$ параллельны.
В этом случае задача имеет много решений: всякая плоскость, проходящая через прямую $a$, будет параллельна прямой $b$.

\paragraph{Пример более сложной задачи на построение.}\label{1938/s22}
Даны две скрещивающиеся прямые ($a$ и $b$, рис.~\ref{1938/s-ris-14}) и точка $A$, не лежащая ни на одной из данных прямых.
Провести через точку $A$ прямую, пересекающую обе данные прямые ($a$ и $b$).

\begin{wrapfigure}{o}{55 mm}
\centering
\includegraphics{mppics/s-ris-14}
\caption{}\label{1938/s-ris-14}
\end{wrapfigure}

\mbox{\so{Решение}.}
Так как искомая прямая должна проходить через точку $A$ и пересекать прямую $a$, то она должна лежать в плоскости, проходящей через прямую $a$ и точку $A$ (так как две её точки должны лежать в этой плоскости: точка $A$ и точка пересечения с прямой $a$).
Совершенно так же убеждаемся, что искомая прямая должна лежать в плоскости, проходящей через точку $A$ и прямую $b$.
Следовательно, она должна служить линией пересечения этих двух плоскостей.

Отсюда такое построение.
Через точку $A$ и прямую $a$ проводим плоскость $M$;
через точку $A$ и прямую $b$ проводим плоскость $N$.
Берём прямую $c$ пересечения плоскостей $M$ и $N$.
Если прямая $c$ не параллельна ни одной из данных прямых, то она пересечётся с каждой из данных прямых (так как с каждой из них она лежит в одной плоскости: $a$ и $c$ лежат в плоскости $M$, $b$ и $c$ — в плоскости $N$).
Прямая $c$ будет в этом случае искомой.
Если же $a\parallel c$ или $b\parallel c$, то задача не имеет решения.

\section{Перпендикуляр и наклонные к плоскости}

\paragraph{}\label{1938/s23}
Поставим задачу определить, в каком случае прямая может считаться перпендикулярной к плоскости.
Докажем предварительно следующее предложение:

\medskip

\so{Теорема}.
\textbf{\emph{Если прямая}} ($AA_1$, рис.~\ref{1938/s-ris-15}), \textbf{\emph{пересекающаяся с плоскостью}} ($M$), \textbf{\emph{перпендикулярна к каким-нибудь двум прямым}} ($OB$ и $OC$), \textbf{\emph{проведённым на этой плоскости через точку пересечения}} ($O$) \textbf{\emph{данной прямой и плоскости, то она перпендикулярна и ко всякой третьей прямой}} ($OD$), \textbf{\emph{проведённой на плоскости через ту же точку пересечения}} ($O$).

\begin{wrapfigure}{o}{40 mm}
\centering
\includegraphics{mppics/s-ris-15}
\caption{}\label{1938/s-ris-15}
\end{wrapfigure}

Отложим на прямой $AA_1$ произвольной длины, но равные отрезки $OA$ и $OA_1$ и проведём на плоскости какую-нибудь прямую, которая пересекала бы три прямые, исходящие из точки $O$, в каких-нибудь точках $C$, $D$ и $B$.
Эти точки соединим с точками $A$ и $A_1$.
Мы получим тогда несколько треугольников.
Рассмотрим их в такой последовательности.

Сначала возьмём треугольники $ACB$ и $A_1CB$;
они равны, так как у них $CB$ — общая сторона, $AC=A_1C$ как наклонные к прямой $AA_1$, одинаково удалённые от основания $O$ перпендикуляра $OC$;
по той же причине $AB=A_1B$.
Из равенства этих треугольников следует, что $\angle ABC\z=\angle A_1BC$.

После этого перейдём к треугольникам $ADB$ и $A_1DB$;
они равны, так как у них $DB$ — общая сторона, $AB=A_1B$ и $\angle ABD=\angle A_1BD$.
Из равенства этих треугольников выводим, что $AD=A_1D$.

Теперь возьмём треугольники $AOD$ и $A_1OD$;
они равны, так как имеют соответственно равные стороны.
Из их равенства выводим, что $\angle AOD=\angle A_1OD$;
а так как эти углы смежные, то, следовательно, $AA_1\perp OD$.

\paragraph{}\label{1938/s24}
\so{Определение}.
Прямая называется \rindex{перпендикулярность}\textbf{перпендикулярной к плоскости}, если она, пересекаясь с этой плоскостью, образует прямой угол с каждой прямой, проведённой на плоскости через точку пересечения.
В этом случае говорят также, что плоскость перпендикулярна к прямой.

Из предыдущей теоремы (§~\ref{1938/s23}) следует, что прямая перпендикулярна к плоскости, если она перпендикулярна к двум прямым, лежащим в данной плоскости и проходящим через точку пересечения данной прямой и плоскости.

Прямая, пересекающая плоскость, но не перпендикулярная к ней, называется \rindex{наклонная}\textbf{наклонной} к этой плоскости.
Точка пересечения прямой с плоскостью называется \rindex{основание!наклонной}\rindex{основание!перпендикуляра}\textbf{основанием}, перпендикуляра или наклонной.


\paragraph{Сравнительная длина перпендикуляра и наклонных.}\label{1938/s25}%
\footnote{Для краткости термины «перпендикуляр» и «наклонная» употребляются вместо «отрезок перпендикуляра, ограниченный данной точкой и основанием перпендикуляра», и «отрезок наклонной, ограниченный данной точкой и основанием наклонной».}
Когда из одной точки $A$ (рис.~\ref{1938/s-ris-16}) проведены к плоскости перпендикуляр $AB$ и наклонная $AC$, условимся называть \rindex{проекция наклонной}\textbf{проекцией} наклонной на плоскость $P$ отрезок $BC$, соединяющий основание перпендикуляра и основание наклонной.
Таким образом, отрезок $BC$ есть проекция наклонной $AC$, отрезок $BD$ есть проекция наклонной $AD$ и так далее.

\medskip

\so{Теорема}.
\textbf{\emph{Если из одной и той же точки}} ($A$, рис.~\ref{1938/s-ris-16}), \textbf{\emph{взятой вне плоскости}} ($P$), \textbf{\emph{проведены к этой плоскости перпендикуляр}} ($AB$) \textbf{\emph{и какие-нибудь наклонные}} ($AC$, $AD$, $AE,\dots$), \textbf{\emph{то:}}

1) \textbf{\emph{две наклонные, имеющие равные проекции, равны;}}

2) \textbf{\emph{из двух наклонных та больше, проекция которой больше.}} 

\begin{wrapfigure}{o}{40 mm}
\centering
\includegraphics{mppics/s-ris-16}
\caption{}\label{1938/s-ris-16}
\end{wrapfigure}

Вращая прямоугольные треугольники $ABC$ и $ABD$ вокруг катета $AB$, мы можем совместить их плоскости с плоскостью $\triangle ABE$.
Тогда все наклонные будут лежать в одной плоскости с перпендикуляром, а все проекции расположатся на одной прямой.
Таким образом, доказываемые теоремы приводятся к аналогичным теоремам планиметрии.

{\small
\medskip

\mbox{\so{Замечание}.}
Так как $AB$ есть катет прямоугольного треугольника, а каждая из наклонных $AC, AD, AE,\dots$ есть гипотенуза, то перпендикуляр $AB$ меньше всякой наклонной;
значит, перпендикуляр, опущенный из точки на плоскость, есть наименьший из всех отрезков, соединяющих данную точку с любой точкой плоскости, и потому он принимается за меру расстояния точки $A$ от плоскости~$P$.

}

\paragraph{}\label{1938/s27}
\so{Обратные теоремы}.
\textbf{\emph{Если из одной и той же точки, взятой вне плоскости, проведены перпендикуляр и какие-нибудь наклонные, то: }}

1) \textbf{\emph{равные наклонные имеют равные проекции;}}

2) \textbf{\emph{из двух проекций та больше, которая соответствует большей наклонной.}}

Доказательство (от противного) предоставляем самим учащимся.

\paragraph{}\label{1938/s28}\so{Теорема}.
\textbf{\emph{Прямая}} ($DE$, рис.~\ref{1938/s-ris-17}), \textbf{\emph{проведённая на плоскости}} ($P$) \textbf{\emph{через основание наклонной}} ($AC$) \textbf{\emph{перпендикулярно к её проекции}} ($BC$), \textbf{\emph{перпендикулярна и к самой наклонной.}}

\begin{wrapfigure}{o}{50 mm}
\centering
\includegraphics{mppics/s-ris-17}
\caption{}\label{1938/s-ris-17}
\end{wrapfigure}

Отложим произвольные, но равные отрезки $CD$ и $CE$ и соединим прямолинейными отрезками точки $A$ и $B$ с точками $D$ и $E$.
Тогда будем иметь: $BD=BE$ как наклонные к прямой $DE$, одинаково удалённые от основания $C$ перпендикуляра $BC$;
$AD=AE$ как наклонные к плоскости $P$, имеющие равные проекции $BD$ и $BE$.
Вследствие этого $\triangle ADE$ равнобедренный, и потому его медиана $AC$ перпендикулярна к основанию $DE$.

Эта теорема носит название \so{теоремы о трёх перпендикулярах}.
Действительно, в ней говорится о связи, соединяющей следующие три перпендикуляра: 1) $AB$ к плоскости $P$, 2) $BC$ к прямой $DE$ и 3) $AC$ к той же прямой $DE$.


\paragraph{}\label{1938/s29}
\so{Обратная теорема}.
\textbf{\emph{Прямая}} ($DE$, рис.~\ref{1938/s-ris-17}), \textbf{\emph{проведённая на плоскости}} ($P$) \textbf{\emph{через основание наклонной}} ($AC$) \textbf{\emph{перпендикулярно к этой наклонной, перпендикулярна и к её проекции.}}

Проделаем те же построения, что и при доказательстве прямой теоремы.
Отложим произвольные, но равные отрезки $CD$ и $CE$ и соединим прямолинейными отрезками точки $A$ и $B$ с точками $D$ и $E$, тогда будем иметь: $AD=AE$ как наклонные к прямой $DE$, одинаково удалённые от основания $C$ перпендикуляра $AC$;
$BD=BE$ как проекции равных наклонных $AD$ и $AE$.
Вследствие этого $\triangle BDE$ равнобедренный, и потому его медиана $BC$ перпендикулярна к основанию $DE$.



\section[Параллельность и перпендикулярность]{Параллельность и перпендикулярность\\ прямых и плоскостей}


\paragraph{Предварительное замечание.}\label{1938/s30}
Параллельность прямых и плоскостей в пространстве и перпендикулярность прямой к плоскости находятся в некоторой зависимости.
Именно наличие параллельности одних элементов влечёт перпендикулярность других, и, обратно, из перпендикулярности одних элементов можно сделать заключение о параллельности других.
Эта связь между параллельностью и перпендикулярностью прямых и плоскостей в пространстве выражается следующими теоремами.

\paragraph{}\label{1938/s31}
\so{Теорема}.
\textbf{\emph{Если плоскость}} ($P$, рис.~\ref{1938/s-ris-18}) \textbf{\emph{перпендикулярна к одной из параллельных прямых}} ($AB$), \textbf{\emph{то она перпендикулярна и к другой}} ($CD$).

Проведём через точку $B$ на плоскости $P$ две какие-нибудь прямые $BE$ и $BF$, а через точку $D$ проведём прямые $DG$ и $DH$, соответственно параллельные и одинаково направленные прямым $BE$ и $BF$.
Тогда будем иметь: $\angle ABE=\angle CDG$ и $\angle ABF = \angle CDH$ как углы с параллельными сторонами.
Но углы $ABE$ и $ABF$ прямые, так как $AB\perp P$, значит, углы $CDG$ и $CDH$ также прямые (§~\ref{1938/s18}).
Следовательно, $CD \perp P$ (§~\ref{1938/s24}).

\begin{figure}[!ht]
\begin{minipage}{.48\textwidth}
\centering
\includegraphics{mppics/s-ris-18}
\end{minipage}
\hfill
\begin{minipage}{.48\textwidth}
\centering
\includegraphics{mppics/s-ris-19}
\end{minipage}

\medskip

\begin{minipage}{.48\textwidth}
\centering
\caption{}\label{1938/s-ris-18}
\end{minipage}
\hfill
\begin{minipage}{.48\textwidth}
\centering
\caption{}\label{1938/s-ris-19}
\end{minipage}
\vskip-4mm
\end{figure}

\paragraph{}\label{1938/s32}
\mbox{\so{Обратная теорема}.}
\textbf{\emph{Если две прямые}} ($AB$ и $CD$, рис. \ref{1938/s-ris-19}) \textbf{\emph{перпендикулярны к одной и той же плоскости}} ($P$), \textbf{\emph{то они параллельны.}}

Предположим противное, то есть что прямые $AB$ и $CD$ не параллельны.
Проведём тогда через точку $D$ прямую, параллельную $AB$. %надо сказать, что D на P
При нашем предположении это будет какая-нибудь прямая $DC_1$, не сливающаяся с $DC$.
Согласно прямой теореме прямая $DC$ будет перпендикулярна к плоскости~$P$.
Проведём через $CD$ и $C_1D$ плоскость $Q$ и возьмём линию её пересечения $DE$ с плоскостью~$P$.
Так как (на основании предыдущей теоремы) $C_1D \perp P$, то $\angle C_1DE$ прямой, а так как по условию $CD \perp P$, то $\angle CDE$ также прямой.
Таким образом, окажется, что в плоскости $Q$ к прямой $DE$ из одной её точки $D$ восставлены два перпендикуляра $DC$ и $DC_1$.
Так как это невозможно, %ref
то нельзя допустить, что прямые $AB$ и $CD$ были не параллельны.

\paragraph{}\label{1938/s33}
\so{Теорема}.
\textbf{\emph{Если прямая}} ($BB_1$, рис.~\ref{1938/s-ris-20}) \textbf{\emph{перпендикулярна к одной из параллельных плоскостей}} ($P$), \textbf{\emph{то она перпендикулярна и к другой}} ($Q$).

\begin{wrapfigure}{r}{50 mm}
\vskip-4mm
\centering
\includegraphics{mppics/s-ris-20}
\caption{}\label{1938/s-ris-20}
\end{wrapfigure}

Проведём через прямую $BB_1$ какие-нибудь две плоскости $M$ и $N$, каждая из которых пересекается с $P$ и $Q$ по параллельным прямым: одна — по параллельным прямым $BC$ и $B_1C_1$, другая — по параллельным прямым $BD$ и $B_1D_1$.
Согласно условию прямая $BB_1$ перпендикулярна к прямым $BC$ и $BD$;
следовательно, она также перпендикулярна к параллельным им прямым $B_1C_1$ и $B_1D_1$, а потому перпендикулярна и к плоскости $Q$, на которой лежат прямые $B_1C_1$ и $B_1D_1$.

\paragraph{}\label{1938/s34}
\so{Обратная теорема}.
\textbf{\emph{Если две плоскости}} ($P$ и $Q$, рис.~\ref{1938/s-ris-21}) \textbf{\emph{перпендикулярны к одной и той же прямой}} ($AB$), \textbf{\emph{то они параллельны.}}

\begin{figure}[!ht]
\centering
\includegraphics{mppics/s-ris-21}
\caption{}\label{1938/s-ris-21}
\end{figure}

Предположим противное, то есть что две различные плоскости $P$ и $Q$ пересекаются.
Возьмём на линии их пересечения какую-нибудь точку $C$ и проведём плоскость $R$ через $C$ и прямую $AB$.
Плоскость $R$ пересечёт плоскости $P$ и $Q$ соответственно по прямым $AC$ и $BC$.
Так как $AB\perp P$, то $AB\perp AC$, и так как $AB\perp Q$, то $AB\perp BC$.
Таким образом, в плоскости $R$ мы будем иметь два перпендикуляра к прямой $AB$, проходящих через одну и ту же точку $C$, перпендикуляры $AC$ и $BC$.
Так как это невозможно, то предположение, что плоскости $P$ и $Q$ пересекаются, неверно.
Значит, они параллельны.

\subsection*{Задачи на построение}

\paragraph{}\label{1938/s35}
\emph{Через данную точку в пространстве провести плоскость, перпендикулярную к данной прямой} ($AB$).

\medskip

\so{Решение}.
1-й \so{случай}.
Данная точка $C$ лежит на прямой $AB$ (рис.~\ref{1938/s-ris-22}).

\begin{wrapfigure}{o}{50 mm}
\centering
\includegraphics{mppics/s-ris-22}
\caption{}\label{1938/s-ris-22}
\end{wrapfigure}

Проведём через прямую $AB$ какие-нибудь две плоскости $P$ и $Q$.
Искомая плоскость должна пересекать эти плоскости по прямым, перпендикулярным к прямой $AB$ (§~\ref{1938/s24}).

Отсюда построение: через $AB$ проводим две произвольные плоскости $P$ и $Q$.
В каждой из этих плоскостей восставляем перпендикуляр к прямой $AB$ в точке $C$ (в плоскости $P$ — перпендикуляр $CD$, в плоскости $Q$ — перпендикуляр $CE$).
Плоскость, проходящая через прямые $CD$ и $CE$, есть искомая.

2-й \so{случай}.
Данная точка $D$ не лежит на прямой $AB$ (рис.~\ref{1938/s-ris-22}).
Через точку $D$ и прямую $AB$ проводим плоскость $P$ и в этой плоскости строим прямую $DC$, перпендикулярную к $AB$.
Через прямую $AB$ проводим произвольно вторую плоскость $Q$ и в этой плоскости строим прямую $CE$, перпендикулярную к $AB$.
Искомая плоскость должна пересечь плоскости $P$ и $Q$ по прямым, перпендикулярным к $AB$.

Отсюда построение: через точку $D$ проводим в плоскости $P$ прямую $DC$, перпендикулярную к $AB$.
Прямая $DC$ пересечёт прямую $AB$ в некоторой точке $C$.
Через точку $C$ проводим в плоскости $Q$ прямую $CE$ перпендикулярно к $AB$.
Плоскость, проходящая через прямые $CD$ и $CE$, — искомая.

Так как в каждой из плоскостей $P$ и $Q$ через данную точку можно провести лишь одну прямую, перпендикулярную к данной, то задача в обоих случаях имеет одно решение, то есть через каждую точку в пространстве можно провести лишь одну плоскость, перпендикулярную к данной прямой.

\paragraph{}\label{1938/s36}
\emph{Через данную точку} ($O$) \emph{пространства провести прямую, перпендикулярную к данной плоскости} ($P$).

1-й \so{случай}.
Точка $O$ лежит на плоскости $P$ (рис.~\ref{1938/s-ris-23}).
Проведём на плоскости $P$ через точку $O$ две какие-либо взаимно перпендикулярные прямые $OA$ и $OB$.
Проведём, далее, через прямую $OA$ какую-либо новую плоскость $Q$ и на этой плоскости $Q$ построим прямую $OC$, перпендикулярную к $OA$.
Через прямые $OB$ и $OC$ проведём новую плоскость $R$ и построим в ней прямую $OM$, перпендикулярную к $OB$.
Прямая $OM$ и будет искомым перпендикуляром к плоскости~$P$.

Действительно, так как $OA \perp OB$ и $OA \perp OC$, то прямая $AO$ перпендикулярна к плоскости $R$ и, следовательно, $OA \perp OM$.
Таким образом, мы видим, что $OM\perp OA$ и $OM\perp OB$;
следовательно, $OM$ перпендикулярна к плоскости~$P$.

\begin{figure}[!ht]
\begin{minipage}{.48\textwidth}
\centering
\includegraphics{mppics/s-ris-23}
\end{minipage}
\hfill
\begin{minipage}{.48\textwidth}
\centering
\includegraphics{mppics/s-ris-24}
\end{minipage}

\medskip

\begin{minipage}{.48\textwidth}
\centering
\caption{}\label{1938/s-ris-23}
\end{minipage}
\hfill
\begin{minipage}{.48\textwidth}
\centering
\caption{}\label{1938/s-ris-24}
\end{minipage}
\vskip-4mm
\end{figure}

2-й \mbox{\so{случай}.}
Точка $O$ не лежит на плоскости $P$ (рис.~\ref{1938/s-ris-24}).
Возьмём на плоскости $P$ какую-нибудь точку $A$ и выполним для неё предыдущее построение.
Мы получим тогда прямую $AB$, перпендикулярную к плоскости~$P$.
После этого через точку $O$ проводим прямую, параллельную $AB$.
Эта прямая и будет искомой (§~\ref{1938/s31}).

\medskip

Задача в обоих случаях имеет одно решение.
В самом деле, так как два перпендикуляра к одной и той же плоскости параллельны, то через одну и ту же точку $O$ нельзя провести двух перпендикуляров к плоскости~$P$.
Следовательно, через каждую точку в пространстве можно провести одну и только одну прямую, перпендикулярную к данной плоскости.

\paragraph{Пример более сложной задачи.}\label{1938/s37}
\emph{Даны две скрещивающиеся прямые} ($a$ и $b$, рис.~\ref{1938/s-ris-25}).
\emph{Построить прямую, пересекающую обе данные прямые и перпендикулярную к ним обеим.}

\medskip

\mbox{\so{Решение}.}
Проведём через прямую $a$ плоскость $M$, параллельную прямой $b$ (§~\ref{1938/s21}).
Из двух каких-нибудь точек прямой $b$ опустим перпендикуляры $AA_1$ и $BB_1$ на плоскость $M$.
Соединим точки $A_1$ и $B_1$ отрезком прямой и найдём точку $C_1$ пересечения прямых $A_1B_1$ и $a$.
Через точку $C_1$ проведём прямую, перпендикулярную к плоскости $M$.

\begin{wrapfigure}{o}{55 mm}
\vskip-0mm
\centering
\includegraphics{mppics/s-ris-25}
\caption{}\label{1938/s-ris-25}
\end{wrapfigure}

Предоставляем самим учащимся доказать, что эта прямая 1) пересечётся с прямой $b$ в некоторой точке $C$ и 2) будет перпендикулярна как к прямой $a$, так и к прямой $b$.
Прямая $CC_1$ будет, следовательно, искомой прямой.


Заметим, что отрезок $CC_1$ меньше всех других отрезков, которые можно получить, соединяя точки прямой $a$ с точками прямой $b$.
В самом деле, возьмём на прямой а какую-нибудь точку $E$ и на прямой $b$ какую-нибудь точку $F$, соединим эти точки отрезком прямой и докажем, что $EF>CC_1$.
Опустим из точки $F$ перпендикуляр $FF_1$ на плоскость $M$.
Тогда будем иметь: $EF>FF_1$ (§~\ref{1938/s25}).
Но $FF_1=CC_1$, следовательно $EF>CC_1$.
На этом основании длина отрезка $CC_1$ называется \rindex{кратчайшее расстояние}\textbf{кратчайшим расстоянием} между данными прямыми $a$ и~$b$.

\section[Углы]{Двугранные углы,
угол прямой с плоскостью,
угол двух скрещивающихся прямых,
многогранные углы}

\subsection*{Двугранные углы}

\begin{wrapfigure}{r}{34 mm}
\vskip-0mm
\centering
\includegraphics{mppics/s-ris-26}
\caption{}\label{1938/s-ris-26}
\bigskip
\includegraphics{mppics/s-ris-27}
\caption{}\label{1938/s-ris-27}
\bigskip
\includegraphics{mppics/s-ris-28}
\caption{}\label{1938/s-ris-28}
\end{wrapfigure}


\paragraph{}\label{1938/s38}
\mbox{\so{Определения}.}
Часть плоскости, лежащая по одну сторону от какой-либо прямой, лежащей в этой плоскости, называется \rindex{полуплоскость}\textbf{полуплоскостью}.
Фигура, образованная двумя полуплоскостями ($P$ и $Q$, рис.~\ref{1938/s-ris-26}), исходящими из одной прямой ($AB$), называется \rindex{двугранный угол}\textbf{двугранным углом}.
Прямая $AB$ называется \rindex{ребро!двугранного угла}\textbf{ребром}, а полуплоскости $P$ и $Q$ — \rindex{грань!двугранного угла}\textbf{гранями} двугранного угла.

Такой угол обозначается обыкновенно двумя буквами, поставленными у его ребра (двугранный угол $AB$).
Но если при одном ребре лежат несколько двугранных углов, то каждый на них обозначают четырьмя буквами, из которых две средние стоят при ребре, а две крайние — у граней (например, двугранный угол $SCDR$, рис.~\ref{1938/s-ris-27}).

Если из произвольной точки $D$ ребра $AB$ (рис.~\ref{1938/s-ris-28}) проведём на каждой грани по перпендикуляру к ребру, то образованный ими угол $CDE$ называется \rindex{линейный угол}\textbf{линейным углом} двугранного угла.

Величина линейного угла не зависит от положения его вершины на ребре.
Так, линейные углы $CDE$ и $C_1D_1E_1$ равны, потому что их стороны соответственно параллельны и одинаково направлены.

Плоскость линейного угла перпендикулярна к ребру, так как она содержит две прямые, перпендикулярные к нему.
Поэтому для получения линейного угла достаточно грани данного двугранного угла пересечь плоскостью, перпендикулярной к ребру, и рассмотреть получившийся в этой плоскости угол.

\paragraph{Равенство и неравенство двугранных углов.}\label{1938/s39}
Два двугранных угла считаются \rindex{равные двугранные углы}\textbf{равными}, если они при вложении могут совместиться;
в противном случае тот из двугранных углов считается меньшим, который составит часть другого угла.

Подобно углам в планиметрии двугранные углы могут быть \rindex{смежные углы}\textbf{смежные}, \rindex{вертикальные углы}\textbf{вертикальные} и прочие.

Если два смежных двугранных угла равны между собой, то каждый из них называется прямым двугранным углом.

\medskip

\so{Теоремы}.
1) \textbf{\emph{Равным двугранным, углам, соответствуют равные линейные углы.}}

2) \textbf{\emph{Б\'{о}льшему двугранному углу соответствует больший линейный угол.}}

\begin{wrapfigure}{o}{64 mm}
\centering
\includegraphics{mppics/s-ris-29}
\caption{}\label{1938/s-ris-29}
\end{wrapfigure}

Пусть $PABQ$ и $P_1A_1B_1Q_1$ (рис.~\ref{1938/s-ris-29}) — два двугранных угла.
Вложим угол $A_1B_1$ в угол $AB$ так, чтобы ребро $A_1B_1$ совпало с ребром $AB$ и грань $P_1$ с гранью~$P$.
Тогда если эти двугранные углы равны, то грань $Q_1$ совпадёт с гранью $Q$;
если же угол $A_1B_1$ меньше угла $AB$, то грань $Q_1$ займёт некоторое положение внутри двугранного угла, например~$Q_2$.


Заметив это, возьмём на общем ребре какую-нибудь точку $B$ и проведём через неё плоскость $R$, перпендикулярную к ребру.
От пересечения этой плоскости с гранями двугранных углов получатся линейные углы.
Ясно, что если двугранные углы совпадут, то у них окажется один и тот же линейный угол $CBD$;
если же двугранные углы не совпадут, если, например, грань $Q_1$ займёт положение $Q_2$, то у б\'{о}льшего двугранного угла окажется больший линейный угол, а именно $\angle CBD > \angle C_2BD$.

\paragraph{}\label{1938/s40}
\so{Обратные теоремы}.
1) \textbf{\emph{Равным, линейным, углам, соответствуют равные двугранные углы.}}

2) \textbf{\emph{Б\'{о}льшему линейному углу соответствует больший двугранный угол.}}

Эти теоремы легко доказываются от противного.

\paragraph{}\label{1938/s41}
\so{Следствия}. 1) \emph{Прямому двугранному углу соответствует прямой линейный угол, и обратно.}

Пусть (рис.~\ref{1938/s-ris-30}) двугранный угол $PABQ$ прямой.
Это значит, что он равен смежному углу $QABP_1$.
Но в таком случае линейные углы $CDE$ и $CDE_1$ также равны;
а так как они смежные, то каждый из них должен быть прямой.

\begin{wrapfigure}{o}{60 mm}
\centering
\includegraphics{mppics/s-ris-30}
\caption{}\label{1938/s-ris-30}
\end{wrapfigure}

Обратно, если равны смежные линейные углы $CDE$ и $CDE_1$, то равны и смежные двугранные углы, то есть каждый из них должен быть прямой.

2) \emph{Все прямые двугранные углы равны}, потому что у них равны линейные углы.

Подобным же образом легко доказать, что:

3) \emph{Вертикальные двугранные углы равны.}

4) \emph{Двугранные углы с соответственно параллельными и одинаково (или противоположно) направленными гранями равны.}

5) Если за единицу двугранных углов возьмём такой двугранный угол, который соответствует единице линейных углов, то можно сказать, что \emph{двугранный угол измеряется его линейным углом}.


\subsection*{Перпендикулярные плоскости}

\paragraph{}\label{1938/s42}
\mbox{\so{Определение}.}
Две плоскости называются \rindex{перпендикулярность}\textbf{взаимно перпендикулярными}, если, пересекаясь, они образуют прямые двугранные углы.

\begin{wrapfigure}{r}{38 mm}
\vskip-5mm
\centering
\includegraphics{mppics/s-ris-31}
\caption{}\label{1938/s-ris-31}
\end{wrapfigure}

\paragraph{Признак перпендикулярности.}\label{1938/s43}\ 

\mbox{\so{Теорема}.}
\textbf{\emph{Если плоскость}} ($P$, рис. \ref{1938/s-ris-31}) \textbf{\emph{проходит через перпендикуляр}} ($AB$) \textbf{\emph{к другой плоскости}} ($Q$), \textbf{\emph{то она перпендикулярна к этой плоскости.}}

Пусть $DE$ будет линия пересечения плоскостей $P$ и $Q$.
На плоскости $Q$ проведём $BC \z\perp DE$.
Тогда угол $ABC$ будет линейным углом двугранного угла $PDEQ$.
Так как прямая $AB$ по условию перпендикулярна к $Q$, то $AB\perp BC$;
значит, угол $ABC$ прямой, а потому и двугранный угол прямой, то есть плоскость $P$ перпендикулярна к плоскости~$Q$.

\paragraph{}\label{1938/s44}
\so{Теорема}.
\textbf{\emph{Если две плоскости}} ($P$ и $Q$, рис.~\ref{1938/s-ris-31}) \textbf{\emph{взаимно перпендикулярны и к одной из них}} ($Q$) \textbf{\emph{проведён перпендикуляр}} ($AB$), \textbf{\emph{имеющий общую точку}} ($A$) \textbf{\emph{с другой плоскостью}} ($P$), \textbf{\emph{то этот перпендикуляр весь лежит в этой плоскости}}~($P$).

\begin{wrapfigure}{o}{45 mm}
\centering
\includegraphics{mppics/s-ris-32}
\caption{}\label{1938/s-ris-32}
\bigskip
\includegraphics{mppics/s-ris-33}
\caption{}\label{1938/s-ris-33}
\end{wrapfigure}

Предположим, что перпендикуляр $AB$ не лежит в плоскости $P$ (как изображено на рис.~\ref{1938/s-ris-32}).
Пусть $DE$ будет линия пересечения плоскостей $P$ и $Q$.
На плоскости $P$ проведём прямую $AC \z\perp DE$, а на плоскости $Q$ проведём прямую $CF \z\perp DE$.
Тогда угол $ACF$, как линейный угол прямого двугранного угла, будет прямой.
Поэтому линия $AC$, образуя прямые углы с $DE$ и $CF$, будет перпендикуляром к плоскости $Q$.
Мы будем иметь тогда два перпендикуляра, опущенные из одной и той же точки $A$ на плоскость $Q$, а именно $AB$ и $AC$.
Так как это невозможно (§~\ref{1938/s36}), то допущение неверно;
значит, перпендикуляр $AB$ лежит в плоскости~$P$.


\paragraph{}\label{1938/s45}
\mbox{\so{Следствие}.}
\emph{Линия пересечения} ($AB$ рис.~\ref{1938/s-ris-33}) \emph{двух плоскостей} ($P$ и $Q$), \emph{перпендикулярных к третьей плоскости} ($R$), \emph{есть перпендикуляр к этой плоскости}.

Действительно, если через какую-нибудь точку $A$ линии пересечения плоскостей $P$ и $Q$ проведём перпендикуляр к плоскости $R$, то этот перпендикуляр согласно предыдущей теореме должен лежать и в плоскости $Q$, и в плоскости $P$; значит, он сольётся с $AB$.



\subsection*{Угол двух скрещивающихся прямых}

\paragraph{}\label{1938/s46}
\mbox{\so{Определение}.}
Углом двух скрещивающихся прямых ($AB$ и $CD$, рис.~\ref{1938/s-ris-34}), для которых дано положение и направление, называется угол ($MON$), 
\begin{figure}[!ht]
\vskip-0mm
\centering
\includegraphics{mppics/s-ris-34}
\caption{}\label{1938/s-ris-34}
\end{figure}
который получится, если из произвольной точки пространства ($O$) проведём полупрямые ($OM$ и $ON$), соответственно параллельные данным прямым ($AB$ и $CD$) и одинаково с ними направленные.


Величина этого угла не зависит от положения точки $O$, так как если построим указанным путём угол $M_1O_1N_1$ с вершиной в какой-нибудь другой точке $O_1$, то $\angle MON = \angle M_1O_1N_1$, потому что эти углы имеют соответственно параллельные и одинаково направленные стороны.

\subsection*{Угол, образуемый прямой с плоскостью}

\begin{wrapfigure}{r}{34 mm}
\vskip-6mm
\centering
\includegraphics{mppics/s-ris-35}
\caption{}\label{1938/s-ris-35}
\bigskip
\includegraphics{mppics/s-ris-36}
\caption{}\label{1938/s-ris-36}
\end{wrapfigure}

\paragraph{Проекция точки и прямой на плоскость.}\label{1938/s47}
Мы говорили ранее (§~\ref{1938/s25}), что когда из одной точки проведены к плоскости перпендикуляр и наклонная, то проекцией этой наклонной на плоскость называется отрезок, соединяющий основание перпендикуляра с основанием наклонной.
Дадим теперь более общее определение проекции.

1) \rindex{ортогональная проекция}\textbf{Ортогональной} (или \textbf{прямоугольной}) \textbf{проекцией} какой-нибудь \textbf{точки} на данную плоскость (например, точки $M$ на плоскость $P$, рис.~\ref{1938/s-ris-35}) называется основание ($m$) перпендикуляра, опущенного на эту плоскость из взятой точки.

2) \textbf{Ортогональной проекцией} какой-нибудь \textbf{линии} на плоскость называется геометрическое место проекций всех точек этой линии.

В частности, если проектируемая линия есть прямая (например, $AB$, рис.~\ref{1938/s-ris-35}), не перпендикулярная к плоскости ($P$), то проекция её на эту плоскость есть также прямая.
В самом деле, если мы через прямую $AB$ и перпендикуляр $Mm$, опущенный на плоскость проекций из какой-нибудь одной точки $M$ этой прямой, проведём плоскость $Q$, то эта плоскость должна быть перпендикулярна к плоскости $P$;
поэтому перпендикуляр, опущенный на плоскость $P$ из любой точки прямой $AB$ (например, из точки $N$), должен лежать в этой плоскости $Q$ (§~\ref{1938/s44}) и, следовательно, проекции всех точек прямой $AB$ должны лежать на прямой $ab$, по которой пересекаются плоскости $P$ и $Q$.

Обратно, всякая точка этой прямой $ab$ есть проекция какой-нибудь точки прямой $AB$, так как перпендикуляр, восставленный из любой точки прямой $ab$, лежит на плоскости $Q$ и, следовательно, пересекается с $AB$ в некоторой точке.
Таким образом, прямая $ab$ представляет собой геометрическое место проекций всех точек данной прямой $AB$ и, следовательно, есть её проекция.

Для краткости вместо «ортогональная проекция» мы будем говорить просто «проекция».

\paragraph{Угол прямой с плоскостью.}\label{1938/s48}
Углом прямой ($AB$, рис.~\ref{1938/s-ris-36}) с плоскостью ($P$) в том случае, когда прямая наклонна к плоскости, называется острый угол ($ABC$), составленный этой прямой с её проекцией на плоскость.
Угол прямой с плоскостью считается прямым если прямая перпендикулярна к плоскости.

Угол этот обладает тем свойством, что он есть наименьший из всех углов, которые данная прямая образует с прямыми, проведёнными на плоскости $P$ через её основание $B$.
Докажем, например, что угол $ABC$ меньше угла $ABD$.
Для этого отложим отрезок $BD=BC$ и соединим $D$ с $A$.
У треугольников $ABC$ и $ABD$ две стороны одного равны соответственно двум сторонам другого, но третьи стороны не равны, а именно $AD>AC$ (§~\ref{1938/s25}).
Вследствие этого угол $ABD$ больше угла $ABC$.

\subsection*{Многогранные углы}

\paragraph{}\label{1938/s49}
\mbox{\so{Определения}.}
Возьмём несколько углов (рис.~\ref{1938/s-ris-37}): $ASB$, $BSD$, $CSD$, которые, примыкая последовательно один к другому, расположены в одной плоскости вокруг общей вершины $S$.
Повернём плоскость угла $ASB$ вокруг общей стороны $SB$ так, чтобы эта плоскость составила некоторый двугранный угол с плоскостью $BSD$.
Затем, не изменяя получившегося двугранного угла, повернём его вокруг прямой $SD$ так, чтобы плоскость $BSD$ составила некоторый двугранный угол с плоскостью $CSD$.
Продолжим такое последовательное вращение вокруг каждой общей стороны.
Если при этом последняя сторона $SF$ совместится с первой стороной $SA$, то образуется фигура (рис.~\ref{1938/s-ris-38}), 
\begin{figure}[!ht]
\begin{minipage}{.32\textwidth}
\centering
\includegraphics{mppics/s-ris-37}
\end{minipage}
\hfill
\begin{minipage}{.32\textwidth}
\centering
\includegraphics{mppics/s-ris-38}
\end{minipage}
\hfill
\begin{minipage}{.32\textwidth}
\centering
\includegraphics{mppics/s-ris-39}
\end{minipage}

\medskip

\begin{minipage}{.32\textwidth}
\centering
\caption{}\label{1938/s-ris-37}
\end{minipage}
\hfill
\begin{minipage}{.32\textwidth}
\centering
\caption{}\label{1938/s-ris-38}
\end{minipage}
\hfill
\begin{minipage}{.32\textwidth}
\centering
\caption{}\label{1938/s-ris-39}
\end{minipage}
\vskip-4mm
\end{figure}
которая называется \rindex{многогранный угол}\textbf{многогранным углом}.
Углы $ASB, BSD,\dots$ называются плоскими углами или \rindex{грань!многогранного угла}\textbf{гранями}, стороны их $SA, SB,\dots$ называются \rindex{ребро!многогранного угла}\textbf{рёбрами}, а общая вершина $S$ — \rindex{вершина!многогранного угла}\textbf{вершиной} многогранного угла.
Каждое ребро многогранного угла является также ребром его двугранного угла;
поэтому в многогранном угле столько двугранных углов и столько плоских, сколько в нём всех рёбер.
Наименьшее число граней в многогранном угле — три;
такой угол называется \rindex{трёхгранный угол}\textbf{трёхгранным}.
Могут быть углы четырёхгранные, пятигранные и так далее.

Многогранный угол обозначается или одной буквой $S$, поставленной у вершины, или же рядом букв $SABCDE$, из которых первая обозначает вершину, а прочие — рёбра по порядку их расположения.

Многогранный угол называется \rindex{выпуклый многогранный угол}\textbf{выпуклым}, если он весь расположен по одну сторону от плоскости каждой из его граней, неограниченно продолженной.
Таков, например, угол, изображённый на рис.~\ref{1938/s-ris-38}.
Наоборот, угол на рис.~\ref{1938/s-ris-39} нельзя назвать выпуклым, так как он расположен по обе стороны от грани $ASB$ или от грани $BSD$.
Если все грани многогранного угла пересечём плоскостью, то в сечении образуется многоугольник ($abcde$).
В выпуклом многогранном угле этот многоугольник тоже выпуклый.

Мы будем рассматривать \so{только выпуклые многогранные углы.}

{

\begin{wrapfigure}{r}{40 mm}
\vskip-9mm
\centering
\includegraphics{mppics/s-ris-40}
\caption{}\label{1938/s-ris-40}
\end{wrapfigure}

\paragraph{}\label{1938/s50}
\mbox{\so{Теорема}.}
\textbf{\emph{В трёхгранном угле каждый плоский угол меньше суммы двух других плоских углов.}}

Пусть в трёхгранном угле $SABC$ (рис. \ref{1938/s-ris-40}) наибольший из плоских углов есть угол $ASC$.
Отложим на этом угле угол $ASD$, равный углу $ASB$, и проведём какую-нибудь прямую $AC$, пересекающую $SD$ в некоторой точке $D$.
Отложим $SB=SD$.

}

Соединив $B$ с $A$ и $C$, получим $\triangle ABC$, в котором
\[AD + DC < AB + BC.\]
Треугольники $ASD$ и $ASB$ равны, так как они содержат по равному углу, заключённому между равными сторонами;
следовательно, $AD \z=AB$.
Поэтому если в выведенном неравенстве отбросить равные слагаемые $AD$ и $AB$, получим, что $DC<BC$.
Теперь замечаем, что у треугольников $SCD$ и $SCB$ две стороны одного равны двум сторонам другого, а третьи стороны не равны;
в таком случае против большей из этих сторон лежит больший угол;
значит,
\[\angle CSD < \angle CSB.\]

Прибавив к левой части этого неравенства угол $ASD$, а к правой равный ему угол $ASB$, получим то неравенство, которое требовалось доказать:
\[\angle ASC < \angle CSB + \angle ASB.\]

Мы доказали, что даже наибольший плоский угол меньше суммы двух других углов.
Значит, теорема доказана.

\medskip

\so{Следствие}.
Отнимем от обеих частей последнего неравенства по углу $ASB$ или по углу $CSB$;
получим:
\[\angle ASC - \angle ASB < \angle CSB;\]
\[\angle ASC - \angle CSB < \angle ASB.\]
Рассматривая эти неравенства справа налево и приняв во внимание, что угол $ASC$ как наибольший из трёх углов больше разности двух других углов, мы приходим к заключению, что \so{в трёхгранном угле каждый плоский угол больше разности двух других углов}.

\begin{wrapfigure}[9]{o}{40 mm}
\vskip-8mm
\centering
\includegraphics{mppics/s-ris-41}
\caption{}\label{1938/s-ris-41}
\end{wrapfigure}

\paragraph{}\label{1938/s51}
\mbox{\so{Теорема}.}
\textbf{\emph{В выпуклом, многогранном угле сумма всех плоских углов меньше $\bm{360\degree}$.}}

Пересечём грани (рис.~\ref{1938/s-ris-41}) выпуклого угла $SABCDE$ такой плоскостью, что в сечении получится выпуклый $n$-угольник $ABCDE$.
Такую плоскость можно получить немного повернув плоскость грани угла $ASB$ вокруг прямой $AB$.
Применяя теорему предыдущего параграфа к каждому из трёхгранных углов, вершины которых находятся в точках $A$, $B$, $C$, $D$ и $E$, находим:
\begin{align*}
\angle ABC &< \angle ABS + \angle SBC,
\\
\angle BCD &< \angle BCS + \angle SCD
\\
&\text{\dots}
\end{align*}
Сложим почленно все эти неравенства.
Тогда в левой части получим сумму всех углов многоугольника $ABCDE$, которая равна $180\degree\cdot n \z- 360\degree$, а в правой — сумму углов треугольников $ABS$, $SBC$ и так далее, кроме тех углов, которые лежат при вершине $S$.
Обозначив сумму этих последних углов буквой $x$, мы получим после сложения:
\[180\degree\cdot n - 360\degree< 180\degree\cdot n - x.\]

Так как в разностях $180\degree\cdot n-360\degree$ и $180\degree\cdot n-x$, уменьшаемые одинаковы, то, чтобы первая разность была меньше второй, необходимо, чтобы вычитаемое $360\degree$ было больше вычитаемого $x$;
значит, $360\degree> x$, то есть $x < 360\degree$.

\paragraph{Симметричные многогранные углы.}\label{1938/s53}
Как известно, вертикальные углы равны, если речь идёт об углах, образованных прямыми или плоскостями.
Посмотрим, справедливо ли это утверждение применительно к углам многогранным.

\begin{wrapfigure}[15]{r}{50 mm}
\vskip-0mm
\centering
\includegraphics{mppics/s-ris-43}
\caption{}\label{1938/s-ris-43}
\end{wrapfigure}

Продолжим (рис.~\ref{1938/s-ris-43}) все рёбра угла $SABCDE$ за вершину $S$, тогда образуется другой многогранный угол $SA_1B_1D_1E_1$, который можно назвать \rindex{вертикальные углы}\textbf{вертикальным} по отношению к первому углу.
Нетрудно видеть, что у обоих углов равны соответственно и плоские углы, и двугранные, но те и другие расположены в \rindex{зеркальный порядок}\textbf{зеркальном порядке}.
Действительно, если мы вообразим наблюдателя, который смотрит извне многогранного угла на его вершину, то рёбра $SA$, $SB$, $SC$, $SD$, $SE$ будут казаться ему расположенными в направлении против движения часовой стрелки, тогда как смотря на угол $SA_1B_1C_1D_1E_1$ он видит рёбра $SA_1, SB_1,\dots$ расположенными по движению часовой стрелки;
то есть если смотреть на один из углов в зеркало, то порядок их рёбер покажется тем же.

Очевидно, что равные многогранные углы должны иметь соответственно равные плоские и двугранные углы, которые \rindex{одинаково расположенные грани}\textbf{одинаково расположены}, то есть расположены в том же порядке.
Многогранные углы с соответственно равными плоскими и двугранными углами, но расположенными в зеркальном порядке, вообще не могут совместиться при вложении;
значит, они не равны.
Такие углы называются центрально симметричными относительно вершины $S$.
Подробнее о симметрии фигур в пространстве будет сказано ниже.


\subsection*{Равенство трёхгранных углов}

\paragraph{}\label{1938/s52} В следующей теореме приведены два простейших признака равенства трёхгранных углов; пара более сложных признаков приведена в §~\ref{1914/402}.

\begin{wrapfigure}[10]{r}{50 mm}
\vskip-6mm
\centering
\includegraphics{mppics/s-ris-42}
\caption{}\label{1938/s-ris-42}
\end{wrapfigure}

\medskip

\mbox{\so{Теоремы}}.
\textbf{\emph{Трёхгранные углы равны, если они имеют:}}

1) \textbf{\emph{по равному двугранному углу, заключённому между двумя соответственно равными и одинаково расположенными плоскими углами;}} или

2) \textbf{\emph{по равному плоскому углу, заключённому между двумя соответственно равными и одинаково расположенными двугранными углами;}} 

1) Пусть $S$ и $S_1$ — два трёхгранных угла (рис.~\ref{1938/s-ris-42}), у которых $\angle ASB\z=\angle A_1S_1B_1$,
$\angle ASC= \angle A_1S_1C_1$ (и эти равные углы одинаково расположены) и двугранный угол $AS$ равен двугранному углу $A_1S_1$.
Вложим угол $S_1$ в угол $S$ так, чтобы у них совпали точки $S_1$ и $S$, прямые $S_1A_1$ и $SA$ и плоскости $A_1S_1B_1$ и $ASB$.
Тогда ребро $S_1B_1$ пойдёт по $SB$ (в силу равенства углов $A_1S_1B_1$ и $ASB$), плоскость $A_1S_1C_1$ пойдёт по $ASC$ (по равенству двугранных углов) и ребро $S_1C_1$ пойдёт по ребру $SC$ (в силу равенства углов $A_1S_1C_1$ и $ASC$).
Таким образом, трёхгранные углы совместятся всеми своими рёбрами, то есть они будут равны.

2) Второй признак, подобно первому, доказывается вложением.

\begin{wrapfigure}{r}{42 mm}
\vskip0mm
\centering
\includegraphics{mppics/s-ris-348}
\caption{}\label{1914/s-ris-348}
\end{wrapfigure}

{\small

\paragraph{Дополнительный угол.}\label{1914/399}
Из вершины $S$ (рис.~\ref{1914/s-ris-348}) трёхгранного угла $SABC$ восставим к грани $ASB$ перпендикуляр $SC_1$ направляя его в ту сторону от этой грани, в которой расположено противоположное ребро $SC$.
Подобно этому проведём перпендикуляр $SA_1$ к грани $BSC$ и $SB_1$ к грани $ASC$. Трёхгранный угол, у которого рёбрами служат полупрямые $SA_1$, $SB_1$ и $SC_1$, наз. \so{дополнительным} для угла $SABC$.

Заметим, что \emph{если для угла $SABC$ дополнительным углом служит угол $SA_1B_1C_1$, то и наоборот: для угла $SA_1B_1C_1$ дополнительным углом будет $SABC$.}
Действительно, плоскость $SA_1B_1$, проходя через перпендикуляры к плоскостям $BSC$ и $ASC$, перпендикулярна к ним обеим, а следовательно, и к линии их пересечения $SC$; значит, ребро $SC$ есть перпендикуляр к грани $SA_1B_1$ и, кроме того, оно расположено по ту же сторону от этой грани, по которую лежит противоположное ребро $SC_1$.
Подобно этому убедимся, что рёбра $SB$ и $SA$ соответственно перпендикулярны к граням $SA_1C_1$ и $SB_1C_1$ и расположены по ту сторону от них, по которую лежат рёбра $SB_1$ и $SA_1$.
Значит, углы $SABC$ и $SA_1B_1C_1$ взаимно дополнительны.

\begin{wrapfigure}{r}{55 mm}
\vskip-4mm
\centering
\includegraphics{mppics/s-ris-349}
\caption{}\label{1914/s-ris-349}
\end{wrapfigure}

\paragraph{}\label{1914/400}
\mbox{\so{Лемма} 1.}
\textbf{\emph{Если два трёxгранные угла взаимно дополнительны, то плоские углы одного служат дополнением до $180\degree$ к противоположным двугранным углам другого.}}

Каждый плоский угол одного из взаимно дополнительных трёxгранных углов образован двумя перпендикулярами, восставленными к граням противоположного двугранного угла другого трёxгранного, из одной точки его ребра.

Заметив это, возьмём какой-нибудь двугранный угол $AB$ (рис. \ref{1914/s-ris-349}) и из произвольной точки $B$ его ребра построим перпендикуляры: $BE$ к грани $AD$ и $BF$ к грани $AC$.
Затем через $BE$ и $BF$ построим плоскость перпендикулярную к ребру $AB$ (§§~\ref{1938/s43}, \ref{1938/s45}).
Пусть пересечения этой плоскости с гранями угла $AB$ будут прямые $BC$ и $BD$.
Тогда угол $CBD$ должен быть линейным углом двугранного $AB$.
Заметим, что стороны угла $EBF$ соответственно перпендикулярны к сторонам угла $CBD$.
Приняв во внимание направления перпендикуляров, получим, что сумма углов $EBF$ и $CBD$ их равна $180\degree$ (§~\ref{1914/s-ris-349}); что и требовалось доказать.

\paragraph{}\label{1914/401} 
\so{Лемма} 2. \textbf{\emph{Равным трёxгранным углам соответствуют равные дополнительные углы и обратно.}}

Равные трёхгранные углы при вложении совмещаются; поэтому совмещаются и те перпендикуляры, которые образуют рёбра дополнительных углов; значит, дополнительные углы также совмещаются.
Обратно: если совмещаются дополнительные углы, то совмещаются и данные углы.

\paragraph{}\label{1914/402}
Следующая теорема даёт два признака равенства трёхгранных углов в дополнение к признакам приведённым в §~\ref{1938/s52}.

\medskip

\so{Теоремы}.
\textbf{\emph{Трёхгранные углы равны, если они имеют:}}

1) \textbf{\emph{по три соответственно равных и одинаково расположенных плоских угла;}}

2) \textbf{\emph{по три соответственно равных и одинаково расположенных двугранных угла.}} 

\begin{wrapfigure}{o}{60 mm}
\vskip-0mm
\centering
\includegraphics{mppics/s-ris-351}
\caption{}\label{1914/s-ris-351}
\end{wrapfigure}

1) Пусть $S$ и $S_1$ (рис.~\ref{1914/s-ris-351}) два трёxгранные угла, у которых плоские углы одного равны соответственно плоским углам другого и, кроме того, равные углы одинаково расположены.

Отложим на всех рёбрах произвольные, но равные, отрезки 
\begin{align*}SA &= SB = SC =
\\
 =S_1A_1 &=S_1B_1 = S_1C_1. 
\end{align*}
и построим треугольники $ABC$ и $A_1B_1C_1$.
Из равенства треугольников $ABS$ и $A_1B_1S_1$ находим: $AB \z= A_1B_1$.
Подобно этому из равенства других боковых треугольников выводим: $AC \z= A_1C_1$ и $BC\z=B_1C_1$.
Следовательно, $\triangle ABC\z=\triangle A_1B_1C_1$.

Опустим на плоскости этих треугольников перпендикуляры $SO$ и $S_1O_1$.
Так как наклонные $SA$, $SB$ и $SC$ равны, то должны быть равны их проекции $OA$, $OB$ и $OC$ (§~\ref{1938/s25}).
Значит, точка $O$ есть центр круга, описанного около треугольника $ABC$.
Точно так же точка $O_1$ есть центр круга, описанного около треугольника $A_1B_1C_1$.
У равных треугольников радиусы описанных кругов равны; значит, $OB = O_1B_1$.
Поэтому $\triangle SBO=\triangle SB_1O_1$ (по гипотенузе и катету), и, следовательно, $OS = O_1S_1$.

Вложим теперь фигуру $S_1A_1B_1C_1$ в фигуру $SABC$ так, чтобы равные треугольники $A_1B_1C_1$ и $ABC$ совместились.
Тогда совместятся описанные окружности, и, следовательно, их центры $O_1$ и $O$.
Поскольку плоские углы одинаково расположены, перпендикуляр $O_1S_1$ пойдёт по $OS$ и точка $S_1$ совпадёт с $S$.
Таким образом, трёхгранные углы совместятся всеми своими рёбрами, значит, они равны.

2) Четвёртый признак легко доказывается при помощи дополнительных углов.
Если у двух трёxгранных углов соответственно равны и одинаково расположены двугранные углы, то у дополнительных углов соответственно равны и одинаково расположены плоские углы (§~\ref{1914/400});
следовательно, дополнительные углы равны; а если равны дополнительные, то равны и данные углы (§~\ref{1914/401}).

}

{\small

\subsection*{Упражнения}

\so{Доказать теоремы:}

\begin{enumerate}[noitemsep]

\item
Две плоскости, параллельные третьей, параллельны между собой.

\item
Все прямые, параллельные данной плоскости и проходящие через одну точку, лежат в одной плоскости, параллельной данной.

\item Дана плоскость $P$ и параллельная ей прямая $a$.
Доказать, что все точки прямой $a$ находятся на одинаковом расстоянии от плоскости~$P$.

\item
Доказать, что все точки одной из двух параллельных плоскостей находятся на одинаковом расстоянии от другой плоскости.

\item
Две плоскости, проходящие через две данные параллельные прямые и не параллельные между собой, пересекаются по прямой, параллельной данным прямым.

\item
Если прямая $a$ параллельна какой-либо прямой $b$, лежащей на плоскости $M$, то всякая плоскость, проходящая через $a$, пересекает плоскость $M$ по прямой, параллельной $b$ (возможно сливающейся с $b$).

\item
Если прямая $a$ параллельна плоскости $M$, то всякая прямая, проходящая через точку, лежащую в плоскости $M$, и параллельная прямой $a$, лежит в плоскости $M$.

\item
Если даны две скрещивающиеся прямые $a$ и $b$ и через первую проведена плоскость, параллельная второй, а через вторую — плоскость, параллельная первой, то эти две плоскости параллельны.

\item
Все прямые, проходящие через какую-нибудь точку на прямой $a$ и перпендикулярные к этой прямой, лежат в одной плоскости, перпендикулярной к $a$.

\item
Если плоскость и прямая перпендикулярны к одной прямой, то они параллельны.

\item
Если прямая $a$, параллельная плоскости $M$, пересекает прямую $b$, перпендикулярную этой плоскости, то прямые $a$ и $b$ перпендикулярны.
\end{enumerate}

\so{Задачи на построение}

\begin{enumerate}[resume,noitemsep]
\item
Через данную точку провести плоскость, параллельную двум данным прямым $a$ и $b$.

\item
Через данную точку провести прямую, параллельную данной плоскости и пересекающую данную прямую.

\item
Построить прямую, пересекающую две данные прямые и параллельную третьей данной прямой.

\item
Построить какую-либо прямую, пересекающую две данные прямые и параллельную данной плоскости (задача может иметь много решений).

\item
Построить какую-либо прямую, пересекающую три данные прямые (задача может иметь много решений).

\item
Через данную точку провести прямую, перпендикулярную двум данным скрещивающимся прямым.

\item
Через данную прямую провести плоскость, перпендикулярную к данной плоскости.

\item
Даны плоскость $M$ и прямая $a\parallel M$.
Через прямую $a$ провести плоскость, пересекающую плоскость $M$ под данным углом.

\item
Дана плоскость $M$ и две точки $A$ и $B$ по одну сторону от неё.
Найти на плоскости $M$ такую точку $C$, что сумма $AC$ + $BC$ была наименьшей.

\end{enumerate}


}

