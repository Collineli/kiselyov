\chapter{Круглые тела}


\section{Цилиндр и конус}

\begin{wrapfigure}{r}{43 mm}
\vskip-0mm
\centering
\includegraphics{mppics/s-ris-121}
\caption{}\label{1938/s-ris-121}
\vskip-0mm
\end{wrapfigure}

\paragraph{Поверхность вращения.}\label{1938/s105}
Поверхностью вращения называется поверхность, которая получается от вращения какой-нибудь линии ($MN$, рис.~\ref{1938/s-ris-121}), называемой образующей, вокруг неподвижной прямой ($AB$), называемой осью, при этом предполагается, что образующая ($MN$) при своём вращении неизменно связана с осью ($AB$).

Возьмём на образующей какую-нибудь точку $P$ и опустим из неё на ось перпендикуляр $PO$.
Очевидно, что при вращении не изменяются ни длина этого перпендикуляра, ни величина угла $AOP$, ни положение точки $O$.
Поэтому каждая точка образующей описывает окружность, плоскость которой перпендикулярна к оси $AB$ и центр которой лежит на пересечении этой плоскости с осью.
Отсюда следует:

\emph{Плоскость, перпендикулярная к оси, пересекаясь с поверхностью вращения, даёт в сечении окружность (или набор окружностей).} 
Такие окружности называются  \rindex{параллель}\textbf{параллелями} поверхности вращения.

Всякая секущая плоскость, проходящая через ось, называется \rindex{меридиональная плоскость}\textbf{меридиональной} плоскостью, а линия её пересечения с поверхностью вращения — \rindex{меридиан}\textbf{меридианом}.
Все меридианы равны между собой, потому что при вращении каждый из них проходит через то положение, в котором ранее был всякий другой меридиан.


\begin{wrapfigure}{r}{38 mm}
\vskip-0mm
\centering
\includegraphics{mppics/s-ris-122}
\caption{}\label{1938/s-ris-122}
\bigskip
\includegraphics{mppics/s-ris-123}
\caption{}\label{1938/s-ris-123}
\bigskip
\includegraphics{mppics/s-ris-124}
\caption{}\label{1938/s-ris-124}
\vskip-0mm
\end{wrapfigure}

\paragraph{Цилиндрическая поверхность.}\label{1938/s106}
Цилиндрической поверхностью называется поверхность, производимая движением прямой ($AB$, рис.~\ref{1938/s-ris-122}), перемещающейся в пространстве параллельно данной прямой и пересекающей при этом данную линию ($MN$).
Прямая $AB$ называется \rindex{образующая поверхности}\textbf{образующей}, а линия $MN$ — \rindex{направляющая поверхности}\textbf{направляющей}.

\paragraph{Цилиндр.}\label{1938/s107}
Цилиндром называется тело, ограниченное цилиндрической поверхностью и двумя параллельными плоскостями (рис.~\ref{1938/s-ris-123}).

Часть цилиндрической поверхности, заключённая между плоскостями, называется \rindex{боковая поверхность цилиндра}\textbf{боковой поверхностью}, а части плоскостей, отсекаемые этой поверхностью, — \rindex{основание цилиндра}\textbf{основаниями} цилиндра.
Расстояние между плоскостями оснований есть \rindex{высота цилиндра}\textbf{высота} цилиндра.
Цилиндр называется \rindex{прямой цилиндр}\textbf{прямым} или \rindex{наклонный цилиндр}\textbf{наклонным}, смотря по тому, перпендикулярны или наклонны к основаниям его образующие.

Прямой цилиндр (рис.~\ref{1938/s-ris-124}) называется круговым, если его основания — круги.
Такой цилиндр можно рассматривать как тело, происходящее от вращения прямоугольника $OAA_1O_1$ вокруг стороны $OO_1$ как оси;
при этом сторона $AA_1$ описывает боковую поверхность, а стороны $OA$ и $O_1A_1$ — круги оснований.
Всякий отрезок $BC$, параллельный $OA$, описывает также круг, плоскость которого перпендикулярна к оси.
Отсюда следует:

\emph{Сечение прямого кругового цилиндра плоскостью, параллельной основаниям, есть круг.}

В элементарной геометрии рассматривается только прямой круговой цилиндр;
для краткости его называют просто цилиндром.
Иногда приходится рассматривать такие призмы, основания которых — многоугольники, вписанные в основания цилиндра или описанные около них, а высоты равны высоте цилиндра;
такие призмы называются \rindex{вписанная призма}\textbf{вписанными} в цилиндр или \rindex{описанная призма}\textbf{описанными} около него.

\paragraph{Коническая поверхность.}\label{1938/s108}
Конической поверхностью называется поверхность, производимая движением прямой ($AB$, рис.~\ref{1938/s-ris-125}), перемещающейся в пространстве так, что она при этом постоянно проходит через неподвижную точку ($S$) и пересекает данную линию ($MN$).
Прямая $AB$ называется \rindex{образующая конической поверхности}\textbf{образующей}, линия $MN$ — \rindex{направляющая конической поверхности}\textbf{направляющей}, а точка $S$ — \rindex{вершина конической поверхности}\textbf{вершиной конической поверхности}.

\begin{wrapfigure}{r}{38 mm}
\vskip-0mm
\centering
\includegraphics{mppics/s-ris-125}
\caption{}\label{1938/s-ris-125}
\bigskip
\includegraphics{mppics/s-ris-126}
\caption{}\label{1938/s-ris-126}
\bigskip
\includegraphics{mppics/s-ris-127}
\caption{}\label{1938/s-ris-127}
\bigskip
\includegraphics{mppics/s-ris-128}
\caption{}\label{1938/s-ris-128}
\vskip-0mm
\end{wrapfigure}

\paragraph{Конус.}\label{1938/s109}
Конусом называется тело, ограниченное частью конической поверхности, расположенной по одну сторону от вершины, и плоскостью, пересекающей все образующие по ту же сторону от вершины (рис.~\ref{1938/s-ris-126}).
Часть конической поверхности, ограниченная этой плоскостью, называется \rindex{боковая поверхность конуса}\textbf{боковой поверхностью}, а часть плоскости, отсекаемая боковой поверхностью, — основанием конуса.
Перпендикуляр, опущенный из вершины на плоскость основания, называется \rindex{высота конуса}\textbf{высотой конуса}.

Конус называется \rindex{прямой круговой конус}\textbf{прямым круговым}, если его основание есть круг, а высота проходит через центр основания (рис.~\ref{1938/s-ris-127}).
Такой конус можно рассматривать как тело, происходящее от вращения прямоугольного треугольника $SOA$ вокруг катета $SO$ как оси.
При этом гипотенуза $SA$ описывает боковую поверхность, а катет $OA$ — основание конуса.
Всякий отрезок $BO$\, параллельный $OA$, описывает при вращении круг, плоскость которого перпендикулярна к оси.
Отсюда следует:

\emph{Сечение прямого кругового конуса плоскостью, параллельной основанию, есть круг.}

В элементарной геометрии рассматривается только прямой круговой конус, который для краткости называется просто конусом.

Иногда приходится рассматривать такие пирамиды, основаниями которых являются многоугольники, вписанные в основание конуса или описанные около него, а вершина совпадает с вершиной конуса.
Такие пирамиды называются \rindex{вписанная пирамида}\textbf{вписанными} в конус или \rindex{описанная пирамида}\textbf{описанными} около него.

\paragraph{Усечённый конус.}\label{1938/s110}
Так называется часть полного конуса, заключённая между основанием и секущей плоскостью, \so{параллельной основанию}.

Круги, по которым параллельные плоскости пересекают конус, называются \rindex{основание усечённого конуса}\textbf{основаниями} усечённого конуса.

Усечённый конус (рис.~\ref{1938/s-ris-128}) можно рассматривать как тело, происходящее от вращения прямоугольной трапеции $OAA_1O_1$ вокруг стороны $OO_1$, перпендикулярной к основаниям трапеции.

\subsection*{Поверхность цилиндра и конуса}

\paragraph{Площадь кривой поверхности.}\label{1938/s111}
Боковые поверхности цилиндра и конуса принадлежат к поверхностям {}\textbf{кривым}, то есть к таким, никакая часть которых не может совместиться с плоскостью.
Поэтому мы должны особо оговорить, как измерить площадь \so{кривой} поверхности, используя \so{плоскую} единицу площади.

Напомним, что длину дуги мы определяли как предел длин вписанных в неё ломаных при условии, что ломаные приближают дугу.
Может показаться, что площадь кривой поверхности можно определить аналогично: рассмотреть вписанные в неё многогранные поверхности и перейти к пределу их площадей при условии, что многогранные поверхности всё точнее и точнее приближают исходную кривую поверхность.
Однако, как будет показано в §~\ref{fikhtengoltz/3-623b}, это определение непригодно даже для боковой поверхности прямого кругового цилиндра --- даже в этом случае предел о котором идёт речь не существует.

На помощь приходит то обстоятельство, что цилиндры, конусы, усечённые конусы (а также шары, которые будут рассматриваться в~§~\ref{1938/s136}) являются \so{выпуклыми телами}; то есть с любой парой своих точек они содержат и отрезок между ними.
Доказано, что если рассматривать приближения выпуклых тел только \so{выпуклыми} вписанными многогранниками, то независимо от выбора последовательности вписанных многогранников, площади их поверхностей стремятся к одному и тому же пределу.
Этот предел и принимается за площадь поверхности исходного тела. 
Более того, если выпуклое тело приближается последовательностью выпуклых тел,
то площади поверхностей последовательности выпуклых тел стремятся к площади поверхности предельного тела.%
\footnote{Доказательство можно найти например в книге «Круг и шар» В. Бляшке.}

Отсюда легко выводятся следующие два утверждения, которые можно принять за \so{определение} площади боковых поверхностей цилиндра и конуса:

1) \emph{За величину боковой поверхности цилиндра принимают предел, к которому стремится боковая поверхность вписанной в этот цилиндр правильной призмы, когда число сторон правильного многоугольника, вписанного в основание, неограниченно удваивается} (и, следовательно, площадь каждой боковой грани неограниченно убывает).

2) \emph{За величину боковой поверхности конуса} (полного или усечённого) \emph{принимается предел, к которому стремится боковая поверхность вписанной в этот конус правильной пирамиды} (полной или усечённой), \emph{когда число сторон правильного многоугольника, вписанного в основание, неограниченно удваивается} (и, следовательно, площадь каждой боковой грани неограниченно убывает).

\paragraph{Сапог Шварца.}\label{fikhtengoltz/3-623b}
Мы построим приближение боковой поверхности прямого кругового цилиндра \so{невыпуклыми} вписанными многогранными поверхностями, площади которых не имеют предела.
Этот пример был приведён Карлом Шварцем; он демонстрирует несостоятельность определения площади поверхности тела как предела площадей поверхностей вписанных в него многогранников (без дополнительного предположения, что многогранники выпуклы).

\begin{wrapfigure}{r}{42 mm}
\centering
\includegraphics{asy/schwarz}
\caption{}
\label{jpg/schwarzscher_stiefel}
\end{wrapfigure}

Разделим данный цилиндр на $m$ равных цилиндров плоскостями, параллельными его основаниям.
Так на поверхности данного цилиндра получится $m+1$ окружностей, включая и окружности обоих оснований цилиндра.
Каждую из этих окружностей разделим на $n$ равных дуг так, чтобы точки деления вышележащей окружности находились в точности над серединами дуг нижележащей окружности.
Рассмотрим все треугольники, образованные хордами всех этих дуг и отрезками, соединяющими концы хорд с теми точками деления выше- и нижележащих окружностей, которые расположены как раз над или под серединами соответствующих дуг.
Все эти $2mn$ равных треугольников и образуют многогранную поверхность; 
она и называется \so{сапогом Шварца}.
Модель сапога Шварца при $m=8$ и $n=6$ показана на рис.~\ref{jpg/schwarzscher_stiefel}.


\begin{wrapfigure}{O}{45 mm}
\centering
\includegraphics{mppics/s-ris-501}
\caption{}\label{treug-sapog}
\end{wrapfigure}

Предположим, что основанием цилиндра является круг радиуса $R$.
Рассмотрим один из треугольников $ABC$ из которых состоит сапог Шварца, рис.~\ref{treug-sapog}.
По построению, основание $AC$ лежит на горизонтальной плоскости,
а проекция $B'$ вершины $B$ на эту плоскость делит дугу $AC$ на две равные части.
Значит вершины треугольника $AB'C$ это три последовательные вершины правильного $2n$-угольника вписанного в окружность радиуса~$R$.
Обозначим через $s_n$ площадь треугольника $AB'C$; очевидно она зависит только от $R$ и $n$.
 
Оба треугольника $ABC$ и $AB'C$ равнобедренные и имеют общее основание $AC$;
при этом высота $BK$ больше высоты $B'K$ так как отрезок $BK$ длинней своей проекции $B'K$.
Отсюда получаем, что
\begin{align*}
\text{площадь}\, ABC&=\tfrac12 AC\cdot BK>
\\
&>\tfrac12 AC\cdot B'K=
\\
&=\text{площадь}\, AB'C=
\\
&=s_n.
\end{align*}
Значит общая площадь $S$ сапога Шварца превосходит $2mns_n$.

Заметим, что построенные многогранные поверхности приближают боковую поверхность цилиндра если $m$ и $n$ неограниченно возрастают.
Предположим, дополнительно, что $m$ возрастает существенно быстрее $n$,
а именно выполняется условие $m>\tfrac1{s_n}$.
Тогда
\[S> 2 m n s_n>2 \tfrac1{s_n} n s_n=2n.\]
Значит, $S$ неограниченно возрастает с ростом $n$.
В частности $S$ не имеет предела.%
\footnote{Подробный анализ зависимости площади $S$ в от $m$ и $n$ приведён в §~623, третьего тома  «Курса дифференциального и интегрального исчисления» Г.~М.~Фихтенгольца.}


\paragraph{}\label{1938/s112}
\so{Теорема}.
\textbf{\emph{Боковая поверхность цилиндра равна произведению длины окружности основания на высоту.}}

\begin{wrapfigure}{r}{38 mm}
\vskip-0mm
\centering
\includegraphics{mppics/s-ris-129}
\caption{}\label{1938/s-ris-129}
\vskip-0mm
\end{wrapfigure}

Впишем в цилиндр (рис.~\ref{1938/s-ris-129}) какую-нибудь правильную призму.
Обозначим буквами $p$ и $H$ числа, выражающие длины периметра основания и высоты этой призмы.
Тогда боковая поверхность её выразится произведением $p\cdot H$.
Предположим теперь, что число сторон вписанного в основание многоугольника неограниченно возрастает.

Тогда периметр $p$ будет стремиться к пределу, принимаемому за длину $C$ окружности основания, а высота $H$ останется без изменения;
следовательно, боковая поверхность призмы, равная всегда произведению $p\cdot H$, будет стремиться к пределу $C\cdot H$.
Этот предел и принимается за величину боковой поверхности цилиндра.
Обозначив боковую поверхность цилиндра буквой $S$, можем написать:
\[S = C\cdot H.\]

\paragraph{}\label{1938/s113}
\so{Следствия}. 1) Если $R$ обозначает радиус основания цилиндра, то $C= 2\pi R$, поэтому боковая поверхность цилиндра выразится формулой:
\[S = 2\pi R \cdot H.\]

2) Чтобы получить полную поверхность цилиндра, достаточно приложить к боковой поверхности сумму площадей двух оснований, поэтому, обозначая полную поверхность через $T$, будем иметь:
\[T= 2\pi RH + \pi R^2 + \pi R^2 = 2\pi R(H + R).\]

\paragraph{}\label{1938/s114}
\so{Теорема}.
\textbf{\emph{Боковая поверхность конуса равна произведению длины окружности основания на половину образующей.}}

\begin{wrapfigure}{r}{38 mm}
\vskip-0mm
\centering
\includegraphics{mppics/s-ris-130}
\caption{}\label{1938/s-ris-130}
\vskip-0mm
\end{wrapfigure}

Впишем в конус (рис.~\ref{1938/s-ris-130}) какую-нибудь правильную пирамиду и обозначим буквами $p$ и $l$ числа, выражающие длины периметра основания и апофемы этой пирамиды.
Тогда боковая поверхность её выразится произведением $\tfrac12 p\cdot l$.

Предположим теперь, что число сторон вписанного в основание многоугольника неограниченно возрастает.
Тогда периметр $p$ будет стремиться к пределу, принимаемому за длину $C$ окружности основания, а апофема $l$ будет иметь пределом образующую конуса (так как из $\triangle SAK$ следует, что $SA\z-SK\z<AK$).
Значит, если образующую конуса обозначим буквой $L$, то боковая поверхность вписанной пирамиды, равная $\tfrac12 p\cdot l$, будет стремиться к пределу $\tfrac12 C\cdot L$. 

Этот предел и принимается за величину боковой поверхности конуса.
Обозначив боковую поверхность конуса буквой $S$, можем написать:
\[S = \tfrac12 C\cdot L.\]

\paragraph{}\label{1938/s115}
\so{Следствия}. 1) Если $R$ обозначает радиус основания конуса, то $C= 2\pi R$, поэтому боковая поверхность конуса выразится формулой:
\[S
= \tfrac12 \cdot 2\pi R \cdot L
=\pi RL.\]

2) Полную поверхность конуса получим, если боковую поверхность сложим с площадью основания;
поэтому, обозначая полную поверхность через $T$:
\[T= \pi RL + \pi R^2 = \pi R(L + R).\]

\paragraph{}\label{1938/s116}
\so{Теорема}.
\textbf{\emph{Боковая поверхность усечённого конуса равна произведению полусуммы длин окружностей оснований на образующую.}}

\begin{wrapfigure}{r}{44 mm}
\vskip-0mm
\centering
\includegraphics{mppics/s-ris-131}
\caption{}\label{1938/s-ris-131}
\vskip-0mm
\end{wrapfigure}

Впишем в усечённый конус (рис.~\ref{1938/s-ris-131}) какую-нибудь правильную усечённую пирамиду и обозначим буквами $p$, $p_1$ и $l$ числа, выражающие в одинаковых линейных единицах длины периметров нижнего и верхнего оснований и апофемы этой пирамиды.
Тогда боковая поверхность вписанной пирамиды равна $\tfrac12 (p+p_1)l$.

При неограниченном возрастании числа боковых граней вписанной пирамиды периметры $p$ и $p_1$ стремятся к пределам, принимаемым за длины $C$ и $C_1$ окружностей оснований, а апофема $l$ имеет пределом образующую $L$ усечённого конуса.
Следовательно, величина боковой поверхности вписанной пирамиды стремится при этом к пределу, равному $\tfrac12(C + C_1)L$.
Этот предел и принимается за величину боковой поверхности усечённого конуса.
Обозначив боковую поверхность усечённого конуса буквой $S$, будем иметь:
\[S=\tfrac12 (C+C_1)L.\]

\paragraph{}\label{1938/s117}
\so{Следствия}. 1) Если $R$ и $R_1$ означают радиусы окружностей нижнего и верхнего оснований, то боковая поверхность усечённого конуса будет:
\[S = \tfrac12(2\pi R + 2\pi R_1)L = \pi (R + R_1)L.\]

2) Если в трапеции $OO_1A_1A$ (рис.~\ref{1938/s-ris-131}), от вращения которой получается усечённый конус, проведём среднюю линию $BC$, то получим:
\[BC = \tfrac12(OA + O_1A_1) = \tfrac12(R + R_1),\]
откуда
\[R + R_1=2\cdot BC\]
Следовательно,
\[S = 2\pi\cdot BC\cdot L,\]
то есть боковая поверхность усечённого конуса равна произведению длины окружности среднего сечения на образующую.

3) Полная поверхность $T$ усечённого конуса выразится так:
\[T = \pi (R^2 + R_1^2 + RL + R_1L).\]

\paragraph{Развёртка цилиндра и конуса.}\label{1938/s118}
Впишем в цилиндр (рис.~\ref{1938/s-ris-132}) какую-нибудь правильную призму и затем вообразим, что боковая её поверхность разрезана вдоль бокового ребра.
Очевидно, что, вращая её грани вокруг рёбер, мы можем \so{развернуть} эту поверхность в плоскую фигуру без разрыва и без складок.
Тогда получится то, что называется \rindex{развёртка поверхности}\textbf{развёрткой} боковой поверхности призмы.
\begin{figure}[h!]
\vskip-0mm
\centering
\includegraphics{mppics/s-ris-132}
\caption{}\label{1938/s-ris-132}
\vskip-0mm
\end{figure}
Она представляет собой прямоугольник $KLMN$, составленный из стольких отдельных прямоугольников, сколько в призме боковых граней.
Основание его $MN$ равно периметру основания призмы, а высота $KN$ есть высота призмы.

Вообразим теперь, что число боковых граней вписанной призмы неограниченно удваивается;
тогда её развёртка будет всё удлиняться, приближаясь к предельному прямоугольнику $KPQN$, у которого длина основания равна длине окружности основания цилиндра, а высота есть высота цилиндра.
Этот прямоугольник называется \rindex{развёртка поверхности}\textbf{развёрткой} боковой поверхности цилиндра.

Подобно этому вообразим, что в конус вписана какая-нибудь правильная пирамида (рис.~\ref{1938/s-ris-133}).
Мы можем разрезать её боковую поверхность по одному из рёбер и затем, поворачивая грани вокруг рёбер, получить её плоскую развёртку в виде многоугольного сектора $SKL$, составленного из стольких равнобедренных треугольников, сколько в пирамиде боковых граней.
\begin{figure}[h!]
\vskip-0mm
\centering
\includegraphics{mppics/s-ris-133}
\caption{}\label{1938/s-ris-133}
\vskip-0mm
\end{figure}
Отрезки $SK$, $Sa$, $Sb,\dots$ равны боковому ребру пирамиды (или образующей конуса), а длина ломаной $Kab\dots L$ равна периметру основания пирамиды.
При неограниченном удвоении числа боковых граней вписанной пирамиды развёртка её увеличивается, приближаясь к предельному сектору $SKM$, у которого длина дуги $KM$ равна длине окружности основания, а радиус $SK$ равен образующей конуса.
Этот сектор называется \rindex{развёртка поверхности}\textbf{развёрткой} боковой поверхности конуса.

Подобно этому можно получить развёртку боковой поверхности усечённого конуса (рис.~\ref{1938/s-ris-133}) в виде части кругового кольца $KMNP$.
Легко видеть, что боковая поверхность цилиндра или конуса равна площади соответствующей развёртки.



\subsection*{Объём цилиндра и конуса}

\paragraph{}\label{1914/470} \so{Лемма} 1. 
\textbf{\emph{Объём цилиндра есть общий предел объёмов правильных вписанных и описанных призм при неограниченном удвоении числа их боковых граней.}}


Впишем в цилиндр и опишем около него по правильной призме с правильным $n$-угольником как основание.
Обозначим объём, площадь основания и высоту соответственно; 
для цилиндра — $V$, $B$, $H$, 
для вписанной призмы — $v_n$, $b_n$, $H$ 
и для описанной призмы — $V_n$, $B_n$, $H$.
Тогда будем иметь (§~\ref{1938/s88}): 
\begin{align*}
v_n&=b_n\cdot H;
&
V_n&=B_n\cdot H.
\end{align*}
Откуда:
\[V_n-v_n=(B_n-b_n)H.\]

При неограниченном удвоении $n$ разность: $B_n-b_n$ стремится к нулю (§~\ref{1938/262}), а множитель $H$ есть число постоянное.
Поэтому правая часть последнего равенства, и следовательно, и его левая часть, стремятся к нулю.

Объём цилиндра, очевидно, больше объёма вписанной призмы, но меньше объёма описанной.
Поэтому каждая из разностей $V-v_n$ и $V_n-V$ меньше разности $V_n-v_n$. 
Но последняя, по доказанному, стремится к нулю.
Следовательно, и первые две разности стремятся к нулю.
Это, по определению предела, и означает, что $V$ есть предел обоих последовательностей $v_n$ и $V_n$.

\paragraph{}\label{1914/471} \so{Лемма} 2. 
\textbf{\emph{Объём полного конуса есть общий предел объёмов правильных вписанных и описанных пирамид при неограниченном удвоении числа их боковых граней.}}

Впишем в конус и опишем около него по какой-нибудь пирамиде с правильным $n$-угольником как основание.
Так же как в §~\ref{1914/470}, обозначим объём, площадь основания и высоту соответственно; 
для конуса — $V$, $B$, $H$, 
для вписанной пирамиды — $v_n$, $b_n$, $H$ 
и для описанной пирамиды — $V_n$, $B_n$, $H$.
Тогда будем иметь (§~\ref{1938/s91}): 
\begin{align*}
v_n&=\tfrac13 b_n\cdot H;
&
V_n&=\tfrac13 B_n\cdot H.\end{align*}
Откуда:
\[V_n-v_n=\tfrac13(B_n-b_n)H.\]

Отсюда, точно также как и в §~\ref{1914/470}, можно заключить, что $V$ есть предел обоих последовательностей $v_n$ и $V_n$.

\medskip

\so{Замечание}. В доказанных леммах вписанные и описанные призмы и пирамиды предполагаются правильными только ради простоты доказательства. Содержание этих лемм остается в полной силе и тогда, когда призмы и пирамиды будут неправильные, лишь бы боковые грани их неограниченно уменьшались. 


\paragraph{}\label{1938/s120}
\so{Теоремы}.
1) \textbf{\emph{Объём цилиндра равен произведению площади основания на высоту.}}

2) \textbf{\emph{Объём конуса равен произведению площади основания на треть высоты.}}

Впишем в цилиндр какую-нибудь правильную призму, а в конус — какую-нибудь правильную пирамиду с правильным $n$-угольником как основание.
Обозначим площадь основания призмы или пирамиды буквой $b_n$, высоту их буквой $H$ и объём — $v_n$, получим:

\medskip

\columnratio{0.5}
\setlength{\columnseprule}{.2pt}
\begin{paracol}{2}
для призмы $v_n = b_n\cdot H.$
\switchcolumn
для пирамиды $v_n= \tfrac13 b_n\cdot H.$
\end{paracol}

\medskip

Вообразим теперь, что число $n$ неограниченно удваивается.
Тогда $b_n$ будет иметь пределом площадь $B$ основания цилиндра или конуса, а высота $H$ остаётся без изменения;
значит, произведения $b_nH$ и $\tfrac13b_nH$ будут стремиться к пределам $BH$ и $\tfrac13BH$.
Согласно доказанным леммам (§~\ref{1914/470}, \ref{1914/471}), объём $V$ цилиндра или конуса будет:

\medskip

\columnratio{0.5}
\setlength{\columnseprule}{.2pt}
\begin{paracol}{2}
для цилиндра $V = B\cdot  H$.
\switchcolumn
для конуса $V= \tfrac13 B\cdot  H$.
\end{paracol}

\medskip

\paragraph{}\label{1938/s121}
\so{Следствие}.
Если радиус основания цилиндра или конуса обозначим через $R$, то $B= \pi R^2$, поэтому 
\[
\text{объём цилиндра}\ V = \pi R^2H;
\quad
\text{объём конуса}\ V = \tfrac13\pi R^2H.\]

\paragraph{}\label{1938/s122}
\so{Теорема}.
\textbf{\emph{Объём, усечённого конуса равен сумме объёмов трёх конусов, имеющих одинаковую высоту с усечённым конусом, а основаниями: один — нижнее основание этого конуса, другой — верхнее, третий — круг, площадь которого есть среднее геометрическое между площадями верхнего и нижнего оснований.}}

Теорему эту докажем совершенно так же, как раньше мы доказали теорему для объёма усечённой пирамиды (§~\ref{1938/s92}).

На верхнем основании усечённого конуса (рис.~\ref{1938/s-ris-134}) поместим такой малый конус (с высотой $h$), который дополняет данный усечённый конус до полного.
Тогда объём $V$ усечённого конуса можно рассматривать как разность объёмов полного конуса и дополнительного.
Поэтому
\[V =\tfrac13 \pi R^2(H + h)-\tfrac13 \pi r^2h=\tfrac13 \pi [R^2H+(R^2-r^2)h].\]
Из подобия треугольников находим:
\[\frac Rr=\frac{H+h}h\]
откуда получаем:
\[Rh = rH + rh;
\quad
(R-r)h = rH;
\quad
h= \frac{rH}{R-r}
\]

\begin{wrapfigure}[10]{o}{54 mm}
\vskip-0mm
\centering
\includegraphics{mppics/s-ris-134}
\caption{}\label{1938/s-ris-134}
\vskip-0mm
\end{wrapfigure}

Поскольку 
\[R^2-r^2=(R+r)(R-r),\] 
получаем
\begin{align*}
V &= \tfrac13 \pi [R^2H+(R^2-r^2)h]=
\\
&= \tfrac13\pi [R^2H + (R +r)rH]=
\\
&=
\tfrac13\pi H(R^2 + Rr + r^2)
= 
\\
&=\tfrac13\pi R^2 H+\tfrac13\pi Rr H+\tfrac13\pi r^2 H.
\end{align*}

Так как $\pi R^2$ выражает площадь нижнего основания, $\pi r^2$ — площадь верхнего основания и $\pi Rr \z= \sqrt{\pi R^2\cdot \pi r^2}$ есть среднее геометрическое между площадями верхнего и нижнего оснований, то полученная нами формула подтверждает теорему.

\subsection*{Подобные цилиндры и конусы}

\paragraph{}\label{1938/s123}
\so{Определение}.
Два цилиндра или конуса называются подобными, если они произошли от вращения подобных прямоугольников или прямоугольных треугольников вокруг сходственных сторон.

Пусть (рис.~\ref{1938/s-ris-135} и \ref{1938/s-ris-136}) $h$ и $h_1$ будут высоты двух подобных цилиндров или конусов, $r$ и $r_1$ — радиусы их оснований, $l$ и $l_1$ — образующие;
тогда согласно определению
\[\frac r{r_1}=\frac h{h_1}\quad\text{и}\quad\frac r{r_1}=\frac l{l_1}\]
откуда (по свойству равных пропорций) находим:
\[\frac {r+h}{r_1+h_1}=\frac r{r_1}
\quad\text{и}\quad
\frac {r+l}{r_1+l_1}=\frac r{r_1}\]

\begin{figure}[h!]
\begin{minipage}{.48\textwidth}
\centering
\includegraphics{mppics/s-ris-135}
\end{minipage}\hfill
\begin{minipage}{.48\textwidth}
\centering
\includegraphics{mppics/s-ris-136}
\end{minipage}

\medskip

\begin{minipage}{.48\textwidth}
\centering
\caption{}\label{1938/s-ris-135}
\end{minipage}\hfill
\begin{minipage}{.48\textwidth}
\centering
\caption{}\label{1938/s-ris-136}
\end{minipage}
\vskip-4mm
\end{figure}

Заметив эти пропорции, докажем следующую теорему:

\paragraph{}\label{1938/s124}
\so{Теорема}.
\textbf{\emph{Боковые и полные поверхности подобных цилиндров или конусов относятся, как квадраты радиусов или высот;
объёмы — как кубы радиусов или высот.}}

Пусть $S$, $T$ и $V$ будут соответственно боковая поверхность, полная поверхность и объём одного цилиндра или конуса;
$S_1$, $T_1$ и $V_1$ — те же величины для другого цилиндра или конуса, подобного первому.
Тогда будем иметь для цилиндров:
\begin{align*}
\frac S{S_1}&=
\frac{2\pi rh}{2\pi r_1h_1}=
\frac{rh}{r_1h_1}=
\frac{r}{r_1}\cdot \frac{h}{h_1}=
\frac{r^2}{r_1^2}=
\frac{h^2}{h_1^2};
\\
\frac T{T_1}&=
\frac{2\pi r(r+h)}{2\pi r_1(r_1+h_1)}=
\frac{r}{r_1}\cdot\frac{r+h}{r_1+h_1}=
\frac{r^2}{r_1^2}=
\frac{h^2}{h_1^2};
\\
\frac V{V_1}&=
\frac{\pi r^2h}{\pi r_1^2h_1}=
\frac{r^2}{r_1^2}\cdot \frac{h}{h_1}=
\frac{r^3}{r_1^3}=
\frac{h^3}{h_1^3};
\end{align*}
для конусов
\begin{align*}
\frac S{S_1}&=
\frac{\pi rl}{\pi r_1l_1}=
\frac{rl}{r_1l_1}=
\frac{r}{r_1}\cdot \frac{l}{l_1}=
\frac{r^2}{r_1^2}=
\frac{l^2}{l_1^2};
\\
\frac T{T_1}&=
\frac{\pi r(r+l)}{\pi r_1(r_1+l_1)}=
\frac{r}{r_1}\cdot\frac{r+l}{r_1+l_1}=
\frac{r^2}{r_1^2}=
\frac{l^2}{l_1^2};
\\
\frac V{V_1}&=
\frac{\frac13\pi r^2h}{\frac13\pi r_1^2h_1}=
\frac{r^2}{r_1^2}\cdot \frac{h}{h_1}=
\frac{r^3}{r_1^3}=
\frac{h^3}{h_1^3}.
\end{align*}

\section{Шар}

\subsection*{Сечение шара плоскостью}

\paragraph{}\label{1938/s125}
\so{Определение}.
Тело, происходящее от вращения полукруга вокруг диаметра, называется \rindex{шар}\textbf{шаром}, а поверхность, образуемая при этом полуокружностью, называется \rindex{сфера}\textbf{сферой} или шаровой поверхностью.
Можно также сказать, что сфера есть геометрическое место точек, одинаково удалённых от одной и той же точки (называемой \rindex{центр сферы}\textbf{центром} сферы).

Отрезок, соединяющий центр с какой-нибудь точкой сферы, называется \rindex{радиус сферы}\rindex{радиус шара}\textbf{радиусом}, а отрезок, соединяющий две точки сферы и проходящий через её центр, называется \rindex{диаметр сферы}\rindex{диаметр шара}\textbf{диаметром}.
Все радиусы одного шара равны между собой;
всякий диаметр равен двум радиусам.

Два шара одинакового радиуса равны, потому что при вложении они совмещаются.

\paragraph{}\label{1938/s126}
\so{Теорема}.
\textbf{\emph{Всякое сечение шара плоскостью есть круг.}}

\begin{wrapfigure}{r}{38 mm}
\vskip-0mm
\centering
\includegraphics{mppics/s-ris-137}
\caption{}\label{1938/s-ris-137}
\vskip-0mm
\end{wrapfigure}

1) Предположим сначала, что (рис.~\ref{1938/s-ris-137}) секущая плоскость $AB$ проходит через центр $O$ шара.
Все точки линии пересечения принадлежат сфере и поэтому одинаково удалены от точки $O$, лежащей в секущей плоскости;
следовательно, сечение есть круг с центром в точке $O$.

2) Положим теперь, что секущая плоскость $CD$ не проходит через центр.
Опустим на неё из центра перпендикуляр $OK$ и возьмём на линии пересечения какую-нибудь точку $M$.
Соединив её с $O$ и $K$, получим прямоугольный треугольник $MOK$, из которого находим:
\[MK = \sqrt{OM^2 - OK^2}. \eqno(1)\]

Так как длины отрезков $OM$ и $OK$ не изменяются при изменении положений точки $M$ на линии пересечения, то расстояние $MK$ есть величина постоянная для данного сечения;
значит, линия пересечения есть окружность, центр которой есть точка $K$.

\paragraph{}\label{1938/s127}
\so{Следствие}.
Пусть $R$ и $r$ будут длины радиуса шара и радиуса круга сечения, a $d$ — расстояние секущей плоскости от центра, тогда равенство (1) примет вид: 
\[r= \sqrt{R^2 - d^2}.\]

Из этой формулы выводим:

1) \emph{Наибольший радиус сечения получается при $d=0$, то есть когда секущая плоскость проходит через центр шара.}
В этом случае $r=R$.
Круг, получаемый в этом случае, называется \rindex{большой круг}\textbf{большим кругом}.

2) \emph{Наименьший радиус сечения получается при $d=R$.}
В этом случае $r=0$, то есть круг сечения обращается в точку.

3) \emph{Сечения, равноотстоящие от центра шара, равны.}

4) \emph{Из двух сечений, неодинаково удалённых от центра шара, то, которое ближе к центру, имеет больший радиус.}

\paragraph{}\label{1938/s128}
\so{Теорема}.
\textbf{\emph{Всякая плоскость}} ($P$, рис.~\ref{1938/s-ris-138}), \textbf{\emph{проходящая через центр сферы, делит её на две зеркально симметричные и равные части.}}

\begin{wrapfigure}{r}{53 mm}
\vskip-0mm
\centering
\includegraphics{mppics/s-ris-138}
\caption{}\label{1938/s-ris-138}
\vskip-0mm
\end{wrapfigure}

Возьмём на сфере какую-нибудь точку $A$;
пусть $AB$ есть перпендикуляр, опущенный из точки $A$ на плоскость~$P$.
Продолжим $AB$ до пересечения со сферой в точке $C$.
Проведя $BO$, мы получим два равных прямоугольных треугольника $AOB$ и $BOC$ (общий катет $BO$, а гипотенузы равны, как радиусы сферы);
следовательно, $AB=BC$, таким образом, всякой точке $A$ сферы соответствует другая точка $C$ сферы, зеркально симметричная относительно плоскости $P$ с точкой $A$.
Значит, плоскость $P$ делит сферу на две зеркально симметричные части.

Эти части не только зеркально симметричны, но и равны, так как, разрезав сферу по плоскости $P$, мы можем совместить эти части повернув дну из них на угол $180\degree$ вокруг любой прямой плоскости $P$ проходящей через~$O$.



\paragraph{}\label{1938/s129}
\so{Теорема}.
\textbf{\emph{Через две точки сферы, не лежащие на концах одного диаметра, можно провести окружность большого круга и только одну.}}

\begin{wrapfigure}{r}{43 mm}
\vskip-0mm
\centering
\includegraphics{mppics/s-ris-139}
\caption{}\label{1938/s-ris-139}
\vskip-0mm
\end{wrapfigure}

Пусть на сфере (рис.~\ref{1938/s-ris-139}), имеющей центр $O$, взяты какие-нибудь две точки, например $C$ и $N$, не лежащие на одной прямой с точкой $O$.
Тогда через точки $C$, $O$ и $N$ можно провести плоскость.
Эта плоскость, проходя через центр $O$, даст в пересечении со сферой окружность большого круга.

Другой окружности большого круга через те же две точки $C$ и $N$ провести нельзя.
Действительно, всякая окружность большого круга должна, по определению, лежать в плоскости, проходящей через центр сферы;
следовательно, если бы через $C$ и $N$ можно было провести ещё другую окружность большого круга, тогда выходило бы, что через три точки $C$, $N$ и $O$, не лежащие на одной прямой, можно провести две различные плоскости, что невозможно.

\paragraph{}\label{1938/s130}
\so{Теорема}.
\textbf{\emph{Окружности двух больших кругов при пересечении делятся пополам.}}

Центр $O$ (рис.~\ref{1938/s-ris-139}), находясь на плоскостях обоих больших кругов, лежит на прямой, по которой эти круги пересекаются;
значит, эта прямая есть диаметр того и другого круга, а диаметр делит окружность пополам.

\subsection*{Плоскость, касательная к сфере}

\paragraph{}\label{1938/s131}
\so{Определение}.
Плоскость, имеющая со сферой только одну общую точку, называется касательной плоскостью.
Возможность существования такой плоскости доказывается следующей теоремой.

\paragraph{}\label{1938/s132}
\so{Теорема}.
\textbf{\emph{Плоскость}} ($P$, рис.~\ref{1938/s-ris-140}), \textbf{\emph{перпендикулярная к радиусу}} ($AO$) \textbf{\emph{в конце его, лежащем на сфере, есть касательная плоскость.}}

\begin{wrapfigure}{r}{43 mm}
\vskip-2mm
\centering
\includegraphics{mppics/s-ris-140}
\caption{}\label{1938/s-ris-140}
\bigskip
\includegraphics{mppics/s-ris-504}
\caption{}\label{1938/s-ris-504}
\end{wrapfigure}

Возьмём на плоскости $P$ произвольную точку $B$ и проведём прямую $OB$.
Так как $OB$ — наклонная, а $OA$ — перпендикуляр к плоскости $P$, то $OB>OA$.
Поэтому точка $B$ лежит вне сферы;
следовательно, у плоскости $P$ есть только одна общая точка $A$ со сферой;
значит, эта плоскость касательная.


\paragraph{}\label{1938/s133}
\mbox{\so{Обратная теорема}.}
\textbf{\emph{Касательная плоскость}} ($P$, рис. \ref{1938/s-ris-140}) \textbf{\emph{перпендикулярна к радиусу}} ($OA$), \textbf{\emph{проведённому в точку касания.}}

Проведём плоскость $Q$ через центр $O$ и произвольную прямую $AB$ на плоскости~$P$.
Заметим, что $Q$ пересекает сферу по окружности большого круга и прямая $AB$ имеет с этим кругом ровно одну общую точку — точку $A$ (рис.~\ref{1938/s-ris-504}).
То есть прямая $AB$ касается окружности большого круга в плоскости $Q$ (§~\ref{1938/113}) и значит $AB\perp OA$.

Точно также докажем, что $AC\z\perp OA$ для другой прямой $AC$ на плоскости $P$
и значит $P\perp OA$ (§~\ref{1938/s23}).

Прямая, имеющая одну ровно общую точку со сферой, называется \so{касательной} к сфере.
Легко видеть, что существует бесчисленное множество прямых, касающихся сферы в данной точке.
Действительно, всякая прямая ($AC$, рис.~\ref{1938/s-ris-140}), лежащая в плоскости, касательной к сфере, в данной точке ($A$) и проходящая через точку касания ($A$), есть касательная к сфере в этой точке.

\subsection*{Сфера и её части}

\paragraph{}\label{1938/s134}
\so{Определения}.
1) Часть сферы (рис.~\ref{1938/s-ris-141}), отсекаемая от неё какой-нибудь плоскостью ($AA_1$), называется \rindex{сферический сегмент}\textbf{сферическим сегментом}. 

\begin{wrapfigure}{r}{34 mm}
\vskip-0mm
\centering
\includegraphics{mppics/s-ris-141}
\caption{}\label{1938/s-ris-141}
\bigskip
\includegraphics{mppics/s-ris-142}
\caption{}\label{1938/s-ris-142}
\vskip-0mm
\end{wrapfigure}

Окружность $AA_1$ называется \rindex{основание сферического сегмента}\textbf{основанием}, а отрезок $KM$ радиуса, перпендикулярного к плоскости сечения, — \rindex{высота сферического сегмента}\textbf{высотой} сферического сегмента.

2) Часть сферы, заключённая между двумя параллельными секущими плоскостями ($AA_1$ и $BB_1$), называется \rindex{сферический пояс}\textbf{сферическим поясом}.

Окружности сечения $AA_1$ и $BB_1$ называются \rindex{основание пояса}\textbf{основаниями}, а расстояние $KL$ между параллельными плоскостями — \rindex{высота пояса}\textbf{высотой} пояса.

Сферический пояс и сегмент можно рассматривать как поверхности вращения, в то время как полуокружность $MABN$, вращаясь вокруг диаметра $MN$, описывает сферу, часть её $AB$ описывает пояс, а часть $MA$ — сегмент.

Для нахождения площади сферы и её частей мы докажем следующую лемму:

\paragraph{}\label{1938/s135}
\mbox{\so{Лемма}.}
\textbf{\emph{Боковая поверхность каждого из трёх тел: конуса, усечённого конуса и цилиндра — равна произведению высоты тела на длину окружности, у которой радиус есть перпендикуляр, восставленный к образующей из её середины до пересечения с осью.}}

1) Пусть конус образуется (рис.~\ref{1938/s-ris-142}) вращением треугольника $ABC$ вокруг катета $AC$.
Если $D$ есть середина образующей $AB$, то (§~\ref{1938/s115})
\[\text{боковая поверхность конуса} = 2\pi \cdot BC\cdot AD. \eqno(1)\]

Проведя $DE\perp AB$, получим два подобных треугольника $ABC$ и $AED$ (они прямоугольные и имеют общий угол $A$);
из их подобия выводим:
\[\frac{BC}{ED} = \frac{AC}{AD},\]
откуда
\[BC\cdot AD = ED\cdot AC,\]
и равенство (1) даёт:
\[\text{боковая поверхность конуса} = 2\pi \cdot ED\cdot AC,\]
что и требовалось доказать.

\begin{wrapfigure}{r}{34 mm}
\vskip-0mm
\centering
\includegraphics{mppics/s-ris-143}
\caption{}\label{1938/s-ris-143}
\vskip-0mm
\end{wrapfigure}

2) Пусть усечённый конус (рис.~\ref{1938/s-ris-143}) образуется вращением трапеции $ABCD$ вокруг стороны $AD$.

Проведя среднюю линию $EF$, будем иметь (§~\ref{1938/s117}):
\[\text{боковая поверхность усечённого конуса} = 2\pi \cdot EF\cdot BC. \eqno(2)\]

Проведём $EG\perp BC$ и $BH\perp DC$.
Тогда получим два подобных треугольника $EFG$ и $BCH$ (стороны одного перпендикулярны к сторонам другого);
из их подобия выводим:
\[\frac{EF}{BH} = \frac{EG}{BC},\]
откуда
\[EF\cdot BC = BH\cdot EG = AD \cdot EG.\]
Поэтому равенство (2) можно записать так:
\[\text{боковая поверхность усечённого конуса} = 2\pi \cdot EG\cdot AD,\]
что и требовалось доказать.

3) Теорема остаётся верной и в применении к цилиндру, так как окружность, о которой говорится в теореме, равна окружности основания цилиндра.

\begin{wrapfigure}{r}{34 mm}
\vskip-0mm
\centering
\includegraphics{mppics/s-ris-144}
\caption{}\label{1938/s-ris-144}
\vskip-0mm
\end{wrapfigure}

\paragraph{}\label{1938/s136} 
Используя рассуждения в §~\ref{1938/s111}, легко прийти к следующему утверждению, которые можно принять за \so{определение} площади сферического пояса:

\emph{За площадь поверхности сферического пояса, образуемого вращением} (рис.~\ref{1938/s-ris-144}) \emph{какой-нибудь дуги} ($BE$) \emph{полуокружности вокруг диаметра} ($AF$), \emph{принимают предел, к которому стремится поверхность, образуемая вращением вокруг того же диаметра правильной вписанной ломаной линии} ($BCDE$), \emph{когда её стороны неограниченно уменьшаются} (и, следовательно, число сторон неограниченно увеличивается).

Это утверждение распространяется и на сферический сегмент, и на всю сферу;
в последнем случае ломаная линия вписывается в целую полуокружность.

\paragraph{}\label{1938/s137}
\so{Теоремы}.
1) \textbf{\emph{Площадь сферического сегмента равна произведению его высоты на длину окружности большого круга.}}

2) \textbf{\emph{Площадь сферического пояса равна произведению его высоты на длину окружности большого круга.}}

\begin{wrapfigure}{r}{34 mm}
\vskip-0mm
\centering
\includegraphics{mppics/s-ris-145}
\caption{}\label{1938/s-ris-145}
\vskip-0mm
\end{wrapfigure}

Впишем в дугу $AF$ (рис.~\ref{1938/s-ris-145}), образующую при вращении сферический сегмент, правильную ломаную линию $ACDEF$ с произвольным числом сторон.

Поверхность, получающаяся от вращения этой ломаной, состоит из частей, образуемых вращением сторон $AC$, $CD$, $DE$ и так далее.
Эти части представляют собой боковые поверхности или полного конуса (от вращения $AC$), или усечённого конуса (от вращения $CD$, $EF,\dots$), или цилиндра (от вращения $DE$), если $DE\parallel AB$.
Поэтому мы можем применить к ним лемму §~\ref{1938/s135}.
При этом заметим, что каждый из перпендикуляров, восставленных из середин образующих до пересечения с осью, равен апофеме ломаной линии.
Обозначив эту апофему буквой $a$, получим:
\begin{align*}
\text{поверхность, образованная вращением}&\ AC = Ac \cdot 2\pi a;
\\
\text{—\textquotedbl—\qquad\qquad—\textquotedbl—\qquad\qquad—\textquotedbl—\qquad}&\ CD = cd \cdot 2\pi a;
\\
\text{—\textquotedbl—\qquad\qquad—\textquotedbl—\qquad\qquad—\textquotedbl—\qquad}&\ DE = de \cdot 2\pi a;
\\
\text{и так далее\qquad\qquad}&
\end{align*}

Сложив эти равенства почленно, получим:
\[\text{поверхность, образованная вращением}\ ACDEF = Af \cdot 2\pi a.\]

При неограниченном увеличении числа сторон вписанной ломаной апофема $a$ стремится к пределу, равному радиусу сферы $R$, а отрезок $Af$ остаётся без изменения;
следовательно, предел поверхности, образованной вращением $ACDEF$ равен $Af\cdot 2\pi R$.
Но предел поверхности, образованной вращением $ACDEF$, принимают за площадь сферического сегмента, а отрезок $Af$ есть высота $H$ сегмента;
поэтому
\[\text{площадь сферического сегмента} = H\cdot 2\pi R = 2\pi RH.\]

2) Предположим, что правильная ломаная линия вписана не в дугу $AF$, образующую сферический сегмент, а в какую-нибудь дугу $CF$, образующую сферический пояс (рис.~\ref{1938/s-ris-145}).
Это изменение, как легко видеть, нисколько не влияет на ход предыдущих рассуждений, поэтому и вывод остаётся тот же, то есть что
\[\text{площадь сферического пояса} = H \cdot 2\pi R = 2\pi RH,\]
где буквой $H$ обозначена высота $cf$ сферического пояса.

\paragraph{}\label{1938/s138}
\so{Теорема}.
\textbf{\emph{Площадь сферы равна произведению длины окружности большого круга на диаметр,}} или: \textbf{\emph{площадь сферы равна учетверённой площади большого круга.}}

Площадь сферы, образуемой вращением полуокружности $ADB$ (рис.~\ref{1938/s-ris-145}), можно рассматривать как сумму поверхностей, образуемых вращением дуг $AD$ и $DB$.
Поэтому согласно предыдущей теореме можно написать:
\begin{align*}
\text{площадь сферы} &= 2\pi R\cdot Ad + 2\pi R\cdot dB =
\\
&=2\pi R(Ad + dB) =
\\
&= 2\pi R\cdot 2R = 
\\
&=4\pi R^2.
\end{align*}

\paragraph{}\label{1938/s139}
\so{Следствие}.
\emph{Площади сфер относятся, как квадраты их радиусов или диаметров}, потому что, обозначая через $R$ и $R_1$ радиусы, а через $S$ и $S_1$ площади двух сфер, будем иметь:
\begin{align*}
\frac{S}{S_1} &= \frac{4\pi R^2}{4\pi R^2_1} =\frac{R^2}{R^2_1} = 
\\&=\frac{4R^2}{4R_1^2} =\frac{(2R)^2}{(2R_1)^2}.
\end{align*}

\subsection*{Объём шара и его частей}

\begin{wrapfigure}{r}{38 mm}
\vskip-0mm
\centering
\includegraphics{mppics/s-ris-146}
\caption{}\label{1938/s-ris-146}
\vskip-0mm
\end{wrapfigure}

\paragraph{}\label{1938/s140}
\mbox{\so{Определение}.}
Тело, получаемое от вращения (рис.~\ref{1938/s-ris-146}) кругового сектора ($AOB$) вокруг одной из его сторон ($OA$) называется \rindex{шаровой сектор}\textbf{шаровым сектором}.
Это тело ограничено боковой поверхностью конуса и сферическим сегментом.


\paragraph{}\label{1938/s141}
Для нахождения объёма шарового сектора и целого шара мы предварительно докажем следующую лемму.

\begin{wrapfigure}{r}{33 mm}
\vskip-0mm
\centering
\includegraphics{mppics/s-ris-147}
\caption{}\label{1938/s-ris-147}
\bigskip
\includegraphics{mppics/s-ris-148}
\caption{}\label{1938/s-ris-148}
\vskip-0mm
\end{wrapfigure}

\medskip

\mbox{\so{Лемма}.}
\textbf{\emph{Если $\triangle ABC$}} (рис.~\ref{1938/s-ris-147}) \textbf{\emph{вращается вокруг оси $xy$, которая лежит в плоскости треугольника, проходит через его вершину $A$, но не пересекает стороны $BC$, то объём тела, получаемого при этом вращении, равен произведению поверхности, образуемой противоположной стороной $BC$, на одну треть высоты $h$, опущенной на эту сторону.}}

При доказательстве рассмотрим три случая:

1) Ось совпадает со стороной $AB$ (рис.~\ref{1938/s-ris-148}).
В этом случае искомый объём равен сумме объёмов двух конусов, получаемых вращением прямоугольных треугольников $BCD$ и $DCA$.
Первый объём равен $\tfrac13 \pi CD^2\cdot DB$, а второй $\tfrac13\pi CD^2\cdot DA$;
поэтому объём, образованный вращением $ABC$, равен 
\[\tfrac13\pi CD^2(DB + DA) = \tfrac13\pi CD\cdot CD\cdot BA.\]

Заметим, что $CD\cdot BA=BC\cdot h$, так как каждое из этих произведений выражает удвоенную площадь $\triangle ABC$;
поэтому
\[\text{объём}\, ABC = \tfrac13\pi CD\cdot BC\cdot h.\]

\begin{wrapfigure}{r}{33 mm}
\vskip-0mm
\centering
\includegraphics{mppics/s-ris-149}
\caption{}\label{1938/s-ris-149}
\vskip-0mm
\end{wrapfigure}

Но произведение $\pi CD\cdot BC$ равно боковой поверхности конуса $BDC$;
значит,
\[\text{объём}\, ABC = (\text{поверхность}\, BC)\cdot h.\]


2) Ось не совпадает с $AB$ и не параллельна $BC$ (рис.~\ref{1938/s-ris-149}).
В этом случае искомый объём равен разности объёмов тел, производимых вращением треугольников $AMC$ и $AMB$.
По доказанному в первом случае
\begin{align*}
\text{объём}\, AMC&= h\cdot (\text{поверхность}\, MC),
\\
\text{объём}\, AMB &= h\cdot (\text{поверхность}\, MB),
\intertext{следовательно,}
\text{объём}\, ABC &= h\cdot (\text{поверхность}\, MC-\text{поверхность}\, MB)=
\\&=h\cdot (\text{поверхность}\, BC).
\end{align*}

\begin{wrapfigure}{r}{33 mm}
\vskip-0mm
\centering
\includegraphics{mppics/s-ris-150}
\caption{}\label{1938/s-ris-150}
\vskip-0mm
\end{wrapfigure}

3) Ось параллельна стороне $BC$ (рис.~\ref{1938/s-ris-150}).
Тогда искомый объём равен объёму цилиндра, производимому вращением прямоугольника $DEBC$ без суммы объёмов конусов, производимых вращением треугольников $AEB$ и $ACD$;
первый из них равен $\pi DC^2\cdot ED$;
второй — $\tfrac13\pi EB^2\cdot EA$ 
и третий — $\tfrac13\pi DC^2\cdot AD$.
Приняв во внимание, что $EB=DC$, получим:
\begin{align*}
\text{объём}\,ABC &= \pi DC^2[ED-\tfrac13(EA + AD)]=
\\
&=\pi DC^2[ED-\tfrac13 ED]=
\\
&= \tfrac23\pi DC^2\cdot ED.
\end{align*}
Произведение $2\pi DC\cdot ED$ выражает боковую поверхность цилиндра, образуемую стороной $BC$;
поэтому
\begin{align*}
\text{объём}\, ABC&= (\text{поверхность}\, BC)\cdot \tfrac13 DC
\\&=(\text{поверхность}\, BC)\cdot \tfrac13 h.
\end{align*}

\paragraph{}\label{1938/s143} 
\so{Теорема}.
\textbf{\emph{Объём шарового сектора равен произведению площади его сферического сегмента на треть радиуса.}}

Пусть шаровой сектор производится вращением вокруг стороны $OA$ (рис.~\ref{1938/s-ris-151}) сектора $AOD$.

\begin{wrapfigure}{r}{33 mm}
\vskip-0mm
\centering
\includegraphics{mppics/s-ris-151}
\caption{}\label{1938/s-ris-151}
\vskip-0mm
\end{wrapfigure}

Впишем в дугу $AD$ правильную ломаную линию $ABCD$ с произвольным числом сторон равным $n$.
Затем, продолжив конечные радиусы $OA$ и $OD$, опишем около дуги $AD$ правильную ломаную $A_1B_1C_1D_1$ стороны которой параллельны сторонам вписанной ломаной.
Многоугольники $OABCD$ и $OA_1B_1C_1D_1$ произведут при вращении некоторые тела, объёмы которых обозначим: первого через $v_n$, а второго через $V_n$.

Докажем прежде всего, что при неограниченном удвоении числа $n$ разность $V_n-v_n$ стремится к нулю.

Объем $v_n$ есть сумма объёмов, получаемых вращением треугольников $OAB$, $OBC$, $OCD$ вокруг оси $OA$.
Объем $V_n$ есть сумма, объёмов, получаемых вращением вокруг той же оси треугольников $OA_1B_1$, $OB_1C_1$ $OC_1D_1$.
Применим к этим объёмам лемму в §~\ref{1938/s140}, при чем заметим, что высоты первых треугольников равны апофеме $a_n$ вписанной ломаной, а высоты вторых треугольников равны радиусу $R$ шара.
Согласно этой лемме будем иметь:
\begin{align*}
v_n&=
(\text{поверхность}\, AB)\cdot\tfrac {a_n}3
+
(\text{поверхность}\, BC)\cdot\tfrac {a_n}3
+
\dots=
\\
&=
(\text{поверхность}\, ABCD)\cdot \tfrac {a_n}3.
\\
V_n&=
(\text{поверхность}\, AB)\cdot\tfrac R3
+
(\text{поверхность}\, BC)\cdot\tfrac R3
+
\dots=
\\
&=
(\text{поверхность}\, ABCD)\cdot \tfrac R3.
\end{align*}


Вообразим теперь, что число $n$ неограниченно удваивается.
При этом условии поверхности $ABCD$ и $A_1B_1C_1D_1$ стремятся к общему пределу, именно к площади сферического сегмента $AD$ (§~\ref{1938/s136}), а апофема $a_n$ имеет пределом радиус $R$.
Следовательно, объёмы $v_n$ и $V_n$ стремятся при этом к общему пределу, именно к произведению $(\text{поверхность сегмента}\,  AD)\cdot \tfrac R3$.
То есть обе последовательности объёмов $v_n$ и $V_n$ приближаются к одной и той же постоянной величине как угодно близко; это возможно только тогда, когда разность $V_n-v_n$ стремится к $0$.

Обозначим буквою $V$ объём шарового сектора $OAD$.
Очевидно, что $V>v_n$ и $V<V_n$.
Значит, каждая из разностей $V_n-V$ и $V-v_n$ меньше разности $V_n-v_n$.
Ho эта разность, как мы видели, при неограниченном удвоении числа сторон ломанных стремится к $0$. Следовательно, разности $V_n-V$ и $V-v_n$ и подавно при этом стремятся к $0$.
Отсюда заключаем, что постоянная величина $V$ есть общий предел последовательностей объёмов $v_n$ и $V_n$.
Ho этот общий предел, как мы нашли, есть произведение $(\text{поверхность сегмента}\,  AD)\cdot \tfrac R3$.
Значит:
\[V=(\text{поверхность сегмента}\,  AD)\cdot \tfrac R3.\]

\begin{wrapfigure}{r}{48 mm}
\vskip-0mm
\centering
\includegraphics{mppics/s-ris-152}
\caption{}\label{1938/s-ris-152}
\vskip-0mm
\end{wrapfigure}

\paragraph{}\label{1938/s144}
\mbox{\so{Теорема}.}
\textbf{\emph{Объём шара равняется произведению площади его поверхности на треть радиуса.}}

Разбив полукруг $ABC$ (рис.~\ref{1938/s-ris-152}), производящий шар, на пару круговых секторов $AOB$ и $BOC$, мы заметим, что объём шара можно рассматривать как сумму объёмов шаровых секторов, производимых вращением этих круговых секторов.
Так как согласно предыдущей теореме
\begin{align*}
\text{объём}\, AOB &= (\text{поверхность}\, AB)\cdot \tfrac13 R,
\\
\text{объём}\, BOC &= (\text{поверхность}\, BC) \cdot \tfrac13 R,
\intertext{то}
\text{объём шара} &= (\text{поверхность}\, AB+\text{поверхность}\, BC)\cdot \tfrac13 R=
\\
&=(\text{поверхность}\, ABC)\cdot \tfrac13 R.
\end{align*}

\medskip

\so{Замечание}.
Можно и непосредственно рассматривать объём шара как объём тела, образованного вращением вокруг диаметра кругового сектора, центральный угол которого равен $180\degree$.

В таком случае объём шара можно получить как частный случай объёма шарового сектора, чей сферический сегмент составляет всю сферу.
В силу предыдущей теоремы \emph{объём шара будет при этом равен его поверхности, умноженной на одну треть радиуса.}

\paragraph{}\label{1938/s145}
\so{Следствие} 1 .
Обозначим высоту сферического пояса или сегмента через $H$, радиус сферы — через $R$, а диаметр — через $D$;
тогда площадь пояса или сегмента выразится, как мы видели (§~\ref{1938/s137}), формулой $2\pi RH$, а площадь сферы (§~\ref{1938/s138}) — формулой $4\pi R^2$;
поэтому

\begin{align*}
\text{объём шарового сектора}&= 2\pi RH\cdot \tfrac13 R =
\\
&= \tfrac23 \pi R^2 H;
\\
\text{объём шара} &= 4\pi R\cdot \tfrac13 R = 
\\
&=\tfrac43\pi R^3=
\\
&= \tfrac43\pi\left(\frac D2\right)^3=
\\
&= \tfrac16\pi D^3.
\end{align*}
Отсюда видно, что объёмы шаров относятся, как кубы их радиусов или диаметров.%
\footnote{Объём шара может быть выведен (не вполне, впрочем, строго) следующим простым рассуждением.
Вообразим, что вся поверхность шара разбита на очень малые участки и что все точки контура каждого участка соединены радиусами с центром шара.
Тогда шар разделится на очень большое число маленьких тел, из которых каждое можно рассматривать как пирамиду с вершиной в центре шара.
Так как объём пирамиды равен произведению поверхности основания на третью часть высоты (которую можно принять равной радиусу шара), то объём шара, равный, очевидно, сумме объёмов всех пирамид, выразится так:
\[\text{объём шара} = S \cdot \tfrac13 R,\]
где $S$ — сумма поверхностей оснований всех пирамид.
Но эта сумма поверхностей оснований должна составить площадь сферы, и, значит,
\[\text{объём шара} = 4\pi R^2\cdot \tfrac13 R = \tfrac13 \pi R^3.\]
Таким образом, объём шара может быть найден посредством формулы его поверхности.
Обратно, площадь сферы может быть найдена с помощью формулы его объёма из равенства:
\[S\cdot \tfrac13 R = \tfrac43 \pi R^3,\quad\text{откуда}\quad S = 4\pi R^2.\]
}

\paragraph{}\label{1938/s146}
\so{Следствие} 2.
\emph{Площадь поверхности шара и объём шара соответственно составляют $\tfrac23$ полной поверхности и объёма цилиндра, описанного около шара.}

Действительно, у цилиндра, описанного около шара, радиус основания равен радиусу шара, а высота равна диаметру шара;
поэтому для такого цилиндра
\begin{align*}
\text{полная поверхность описанного цилиндра} &= 2\pi R \cdot 2R + 2\pi R = 6\pi R,
\\
\text{объём описанного цилиндра} &= \pi R^2\cdot 2R = 2\pi R^3.
\end{align*}
Отсюда видно, что $\tfrac23$ полной поверхности этого цилиндра равны $4\pi R^2$, то есть равны поверхности шара, а $\tfrac23$ объёма цилиндра составляют $\tfrac43\pi R^3$, то есть объём шара.

Это предложение было доказано Архимедом (в III веке до начала нашей эры).
Архимед выразил желание, чтобы чертёж;
этой теоремы был изображён на его гробнице, что и было исполнено римским военачальником Марцеллом (Ф. Кэджори. История элементарной математики).

Предлагаем учащимся как полезное упражнение доказать, что поверхность и объём шара составляют $\tfrac49$ соответственно полной поверхности и объёма описанного конуса, у которого образующая равна диаметру основания.
Соединяя это предложение с указанным в следствии 2, мы можем написать такое равенство, где $Q$ обозначает поверхность или объём:
\[\frac{Q_{\text{шара}}}4
=\frac{Q_{\text{цилиндра}}}6
=\frac{Q_{\text{конуса}}}9.
\]

\paragraph{}\label{1938/s147}
\so{Замечание}.
Формулу для объёма шара можно весьма просто получить, основываясь на принципе Кавальери (§~\ref{1938/s89}), следующим образом.

\begin{figure}[h!]
\vskip-0mm
\centering
\includegraphics{mppics/s-ris-153}
\caption{}\label{1938/s-ris-153}
\vskip-0mm
\end{figure}

Пусть на одной и той же плоскости $H$ (рис.~\ref{1938/s-ris-153}) помещены шар радиуса $R$ и цилиндр, радиус основания которого равен $R$, а высота $2R$ (значит, это такой цилиндр, который может быть описан около шара радиуса $R$).
Вообразим далее, что из цилиндра вырезаны и удалены два конуса, имеющие общую вершину на середине $a$ оси цилиндра, и основания — у одного верхнее основание цилиндра, у другого нижнее.
От цилиндра останется тогда некоторое тело, объём которого, как мы сейчас увидим, равен объёму нашего шара.

Проведём какую-нибудь плоскость, параллельную плоскости $H$ и которая пересекалась бы с обоими телами.
Пусть расстояние этой плоскости от центра шара будет $d$, а радиус круга, полученного в сечении плоскости с шаром, пусть будет $r$.
Заметим, что $R$, $d$ и $r$ являются сторонами прямоугольного треугольника.
По теореме Пифагора, имеем $r^2=R^2-d^2$.
Значит площадь этого круга окажется равной 
\[\pi r^2 = \pi (R^2 - d^2).\]

Та же секущая плоскость даст в сечении с телом, оставшимся от цилиндра, круговое кольцо (оно на чертеже покрыто штрихами), у которого радиус внешнего круга равен $R$, а внутреннего $d$ (прямоугольный треугольник, образованный этим радиусом и отрезком $am$, равнобедренный, так как каждый острый угол его равен $45\degree$).
Значит, площадь этого кольца равна 
\[\pi R^2 - \pi d^2 = \pi(R^2 - d^2).\]

Мы видим, таким образом, что секущая плоскость, параллельная плоскости $H$, даёт в сечении с шаром и телом, оставшимся от цилиндра, фигуры одинаковой площади, следовательно, согласно принципу Кавальери, объёмы этих тел равны.
Но объём тела, оставшегося от цилиндра, равен объёму цилиндра без удвоенного объёма конуса, то есть он равен
\[\pi R^2\cdot 2R - 2 \cdot \tfrac13\pi R^2\cdot R = 2\pi R^3 - \tfrac23\pi R^3 = \tfrac43\pi R^3,\]
значит, это и будет объём шара.

\begin{wrapfigure}{r}{38 mm}
\vskip-0mm
\centering
\includegraphics{mppics/s-ris-154}
\caption{}\label{1938/s-ris-154}
\vskip-0mm
\end{wrapfigure}

\paragraph{}\label{1938/s148}
\mbox{\so{Определения}.}
1) Часть шара ($ACC'$, рис.~\ref{1938/s-ris-154}), отсекаемая от него какой-нибудь плоскостью ($CC'$), называется \rindex{шаровой сегмент}\textbf{шаровым сегментом}.
Круг сечения называется \rindex{основание шарового сегмента}\textbf{основанием сегмента}, а отрезок $Am$ радиуса, перпендикулярного к основанию, — высотой сегмента.

2) Часть шара, заключённая между двумя параллельными секущими плоскостями ($CC'$ и $DD'$), называется \rindex{шаровой слой}\textbf{шаровым слоем}.
Круги параллельных сечений называются основаниями слоя, а расстояние $mn$ между ними — его \rindex{высота!шарового слоя}\textbf{высотой}.

Оба эти тела можно рассматривать как происходящие от вращения вокруг диаметра $AB$ части круга $AmC$ или части $CmnD$.

\paragraph{}\label{1938/s149}
\so{Теорема}.
\textbf{\emph{Объём шарового сегмента равен объёму цилиндра, у которого радиус основания есть высота сегмента, а высота равна радиусу шара, уменьшенному на треть высоты сегмента,}} то есть
\[ V = \pi H^2(R-\tfrac13 H)\]
где $H$ есть высота сегмента, a $R$ — радиус шара.


\begin{wrapfigure}{r}{38 mm}
\vskip-0mm
\centering
\includegraphics{mppics/s-ris-155}
\caption{}\label{1938/s-ris-155}
\vskip-0mm
\end{wrapfigure}

Объём шарового сегмента, получаемого вращением вокруг диаметра $AD$ (рис.~\ref{1938/s-ris-155}) части круга $ACB$, найдётся, если из объёма шарового сектора, получаемого вращением кругового сектора $AOB$, вычтем объём конуса, получаемого вращением $\triangle COB$.
Первый из них равен $\tfrac23\pi R^2H$, а второй $\tfrac13\pi CB^2\cdot CO$.
Так как $CB$ есть средняя пропорциональная между $AC$ и $CD$, то $CB^2 = H(2R - H)$, поэтому
\begin{align*}
CB^2\cdot CO &= H(2R - H)(R - H)=
\\
&=2R^2H - RH^2 - 2RH^2 + H^3 =
\\
&= 2R^2H-3H^2R + H^3;
\end{align*}
следовательно, 
\begin{align*}
\text{объём}\, ABB_1
&= \text{объёму}\, OBAB_1- \text{объём}\, OBB_1=
\\
&=\tfrac23\pi R^2H -\tfrac13 \pi CB^2\cdot CO = 
\\
&=\tfrac23 R^2H - \tfrac23\pi R^2H + \pi RH^2 - \tfrac13\pi H^3 = 
\\
&= \pi H^2(R-\tfrac13H).
\end{align*}

\subsection*{Упражнения}

\begin{enumerate}

\item 
Объём цилиндра, у которого высота вдвое более диаметра основания, равен 1 м$^2$.
Вычислить его высоту.

\item
Вычислить боковую поверхность и объём усечённого конуса, у которого радиусы оснований равны 27 см и 18 см, а образующая равна 21см.

\item
На каком расстоянии от центра шара, радиус которого равен 2,425 м, следует провести секущую плоскость, чтобы отношение поверхности меньшего сегмента к боковой поверхности конуса, имеющего общее с сегментом основание, а вершину в центре шара, равнялось 7:4?

\item
Найти объём тела, происходящего от вращения правильного шестиугольника со стороной $a$ вокруг одной из его сторон.

\item
Вычислить радиус шара, описанного около куба, ребро которого равно 1 м.

\item
Вычислить объём тела, происходящего от вращения правильного треугольника со стороной $a$ вокруг оси, проходящей через его вершину и параллельной противоположной стороне.

\item
Дан равносторонний $\triangle ABC$ со стороной $a$;
на $AC$ строят квадрат $BCDE$, располагая его в противоположную сторону от треугольника.
Вычислить объём тела, происходящего от вращения пятиугольника $ABEDC$ вокруг стороны $AB$.

\item
Дан квадрат $ABCD$ со стороной $a$.
Через вершину $A$ проводят прямую $AM$, перпендикулярную к диагонали $AC$, и вращают квадрат вокруг $AM$.
Вычислить поверхность, образуемую контуром квадрата, и объём, образуемый площадью квадрата.

\item
Дан правильный шестиугольник $ABCDEF$ со стороной $a$.
Через вершину $A$ проводят прямую $AM$, перпендикулярную к радиусу $OA$, и вращают шестиугольник вокруг $AM$.
Вычислить поверхность, образуемую контуром, и объём, образуемый площадью правильного шестиугольника.

\item
В шаре, радиус которого равен 2, просверлено цилиндрическое отверстие вдоль его диаметра.
Вычислить объём оставшейся части, если радиус цилиндрического отверстия равен 1.

\item
Вычислить объём шара, который, будучи вложен в коническую воронку с радиусом основания $r = 5$ см и с образующей $l = 13$ см, касается основания воронки.

\item
Около круга радиуса $r$ описан равносторонний треугольник.
Найти отношение объёмов тел, которые производятся вращением круга и площади треугольника вокруг высоты треугольника.

\item
В цилиндрический сосуд, у которого диаметр основания равен 6 см, а высота 36 см, налита вода до половины высоты сосуда.
На сколько поднимается уровень воды в сосуде, если в него погрузить шар диаметром 5 см.

\item
Железный пустой шар, внешний радиус которого равен 0,154 м, плавает в воде, погружаясь в неё наполовину.
Вычислить толщину оболочки этого шара, зная, что плотность железа в 7,7 раза выше плотности воды.

\item
Диаметр Марса составляет половину земного.
Во сколько раз поверхность и объём Марса меньше, чем соответственные величины для Земли?

\item
Диаметр Юпитера в 11 раз больше земного.
Во сколько раз Юпитер превышает Марс по поверхности и объёму? (Используйте условие предыдущей задачи.)
\end{enumerate}
