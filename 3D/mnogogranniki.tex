\chapter{Многогранники}


\section{Параллелепипед и пирамида}

\paragraph{Многогранник.}\label{1938/s67}
\so{Многогранником} называется тело, ограниченное плоскими многоугольниками.
Общие стороны смежных многоугольников называются \rindex{ребро!многогранника}\textbf{рёбрами} многогранника.
Многоугольники, которые ограничивают многогранник, называются его \rindex{грань!многогранника}\textbf{гранями}.
Грани многогранника, сходящиеся в одной точке, образуют многогранный угол;
вершины таких многогранных углов называются \rindex{вершина!многогранника}\textbf{вершинами многогранника}.
Прямые, соединяющие две какие-нибудь вершины, не лежащие на одной грани, называются диагоналями многогранника.

Мы будем рассматривать только \so{выпуклые} многогранники, то есть такие, которые расположены по одну сторону от плоскости каждой из его граней.

Наименьшее число граней в многограннике — четыре;
такой многогранник получается от пересечения трёхгранного угла какой-нибудь плоскостью.

\paragraph{Призма.}\label{1938/s68}
\so{Призмой} называется многогранник, у которого две грани — равные многоугольники с соответственно параллельными сторонами, а все остальные грани — параллелограммы.

Чтобы убедиться в существовании такого многогранника, возьмём (рис.~\ref{1938/s-ris-73}) какой-нибудь многоугольник $ABCDE$ и через его вершины проведём ряд параллельных прямых, не лежащих в его плоскости.
Взяв затем на одной из этих прямых произвольную точку $A_1$, проведём через неё плоскость, параллельную плоскости $ABCDE$;
через каждые две соседние параллельные прямые также проведём по плоскости.
Эти плоскости образуют продолжение граней многогранника $ABCDEA_1B_1C_1D_1E_1$, удовлетворяющий определению призмы.
Действительно, параллельные плоскости $ABCDE$ и $A_1B_1C_1D_1E_1$ пересекаются боковыми плоскостями по параллельным прямым (§~\ref{1938/s16});
поэтому фигуры $AA_1E_1E$, $EE_1D_1D$ и так далее — параллелограммы.
С другой стороны, у многоугольников $ABCDE$ и $A_1B_1C_1D_1E_1$ равны соответственно стороны (как противоположные стороны параллелограммов) и углы (как углы с параллельными и одинаково направленными сторонами);
следовательно, эти многоугольники равны.

{\sloppy

Многоугольники $ABCDE$ и $A_1B_1C_1D_1E_1$, лежащие в параллельных плоскостях, называются \rindex{основание!призмы}\textbf{основаниями} призмы, перпендикуляр $OO_1$, опущенный из какой-нибудь точки одного основания на плоскость другого, называется \rindex{высота!призмы}\textbf{высотой} призмы.
Параллелограммы $AA_1B_1B$, $BB_1C_1C$ и так далее называются \rindex{боковая грань!призмы}\textbf{боковыми гранями} призмы, а их стороны $AA_1$, $BB_1$ и так далее, соединяющие соответственные вершины оснований, — \rindex{боковое ребро призмы}\textbf{боковыми рёбрами}.
У призмы все боковые рёбра равны как отрезки параллельных прямых, заключённые между параллельными плоскостями.

}

\begin{wrapfigure}{o}{35 mm}
\vskip-0mm
\centering
\includegraphics{mppics/s-ris-73}
\caption{}\label{1938/s-ris-73}
\end{wrapfigure}

Отрезок прямой, соединяющий какие-нибудь две вершины, не прилежащие к одной грани, называется диагональю призмы.
Таков, например, отрезок $AD_1$ (рис.~\ref{1938/s-ris-73}).

Плоскость, проведённая через какие-нибудь два боковых ребра, не прилежащих к одной боковой грани призмы (например, через рёбра $AA_1$ и $CC_1$, рис.~\ref{1938/s-ris-73}), называется диагональной плоскостью (на рисунке не показанной).

Призма называется \rindex{прямая призма}\textbf{прямой} или \rindex{наклонная призма}\textbf{наклонной}, смотря по тому, будут ли её боковые рёбра перпендикулярны или наклонны к основаниям.
У прямой призмы боковые грани — прямоугольники.
За высоту такой призмы можно принять боковое ребро.

Прямая призма называется \rindex{правильная призма}\textbf{правильной}, если её основания — правильные многоугольники.
У такой призмы все боковые грани — равные прямоугольники.

Призмы бывают треугольные, четырёхугольные и так далее, смотря по тому, что является основанием: треугольник, четырёхугольник и так далее.


\paragraph{Параллелепипед.}\label{1938/s69}\rindex{параллелепипед}
Параллелепипедом называют призму, у которой основаниями служат параллелограммы (рис.~\ref{1938/s-ris-74}).
Параллелепипеды, как и всякие призмы, могут быть прямые и наклонные.
Прямой параллелепипед называется \rindex{прямоугольный параллелепипед}\textbf{прямоугольным}, если его основание — прямоугольник (рис.~\ref{1938/s-ris-75}).
Из этих определений следует:

1) у параллелепипеда все шесть граней — параллелограммы;

\begin{wrapfigure}[43]{o}{35 mm}
\vskip-0mm
\centering
\includegraphics{mppics/s-ris-74}
\caption{}\label{1938/s-ris-74}
\bigskip
\includegraphics{mppics/s-ris-75}
\caption{}\label{1938/s-ris-75}
\bigskip
\includegraphics{mppics/s-ris-76}
\caption{}\label{1938/s-ris-76}
\bigskip
\includegraphics{mppics/s-ris-77}
\caption{}\label{1938/s-ris-77}
\end{wrapfigure}

2) у прямого параллелепипеда четыре боковые грани — прямоугольники, а два основания — параллелограммы;

3) у прямоугольного параллелепипеда все шесть граней — прямоугольники.

Три ребра прямоугольного параллелепипеда, сходящиеся к одной вершине, называются его \rindex{измерения!параллелепипеда}\textbf{измерениями};
одно из них можно рассматривать как длину, другое — как ширину, а третье — как высоту.

Прямоугольный параллелепипед, имеющий равные измерения, называется кубом.
У куба все грани — квадраты.

\paragraph{Пирамида.}\label{1938/s70}
Пирамидой называется многогранник, у которого одна грань, называемая основанием, есть какой-нибудь многоугольник, а все остальные грани, называемые боковыми, — треугольники, имеющие общую вершину.

Чтобы получить пирамиду, достаточно какой-нибудь многогранный угол $S$ (рис.~\ref{1938/s-ris-76}) пересечь произвольной плоскостью $ABCD$ и взять отсечённую часть $SABCD$.

Общая вершина $S$ боковых треугольников называется \rindex{вершина!пирамиды}\textbf{вершиной} пирамиды, а перпендикуляр $SO$, опущенный из вершины на плоскость основания, — \rindex{высота!пирамиды}\textbf{высотой}.

Обыкновенно, обозначая пирамиду буквами, пишут сначала ту, которой обозначена вершина, например $SABCD$ (рис.~\ref{1938/s-ris-76}).

Плоскость, проведённая через вершину пирамиды и через какую-нибудь диагональ основания (например, через диагональ $BD$, рис.~\ref{1938/s-ris-78}), называется \rindex{диагональная плоскость пирамиды}\textbf{диагональной плоскостью}.

Пирамиды бывают треугольные (рис.~\ref{1938/s-ris-77}), четырёхугольные и так далее, смотря по тому, что является основанием — треугольник, четырёхугольник и так далее.

Пирамида называется \rindex{правильная пирамида}\textbf{правильной} (рис.~\ref{1938/s-ris-78}), если, во-первых, её основание есть правильный многоугольник и, во-вторых, высота проходит через центр этого многоугольника.
В правильной пирамиде все боковые рёбра равны между собой (как наклонные с равными проекциями).

Поэтому все боковые грани правильной пирамиды — равные равнобедренные треугольники.
Высота $SM$ (рис.~\ref{1938/s-ris-78}) каждого из этих треугольников называется \rindex{апофема пирамиды}\textbf{апофемой}.
Все апофемы в правильной пирамиде равны.

\begin{figure}[h!]
\begin{minipage}{.48\textwidth}
\centering
\includegraphics{mppics/s-ris-78}
\end{minipage}
\hfill
\begin{minipage}{.48\textwidth}
\centering
\includegraphics{mppics/s-ris-79}
\end{minipage}

\medskip

\begin{minipage}{.48\textwidth}
\centering
\caption{}\label{1938/s-ris-78}
\end{minipage}
\hfill
\begin{minipage}{.48\textwidth}
\centering
\caption{}\label{1938/s-ris-78}
\end{minipage}
\vskip-4mm
\end{figure}

\paragraph{Усечённая пирамида.}\label{1938/s71}
Часть пирамиды (рис.~\ref{1938/s-ris-79}), заключённая между основанием ($ABCDE$) и секущей плоскостью ($A_1B_1C_1D_1E_1$), параллельной основанию, называется усечённой пирамидой.
Параллельные грани называются основаниями, а отрезок перпендикуляра $OO_1$, опущенного из какой-нибудь точки $O_1$ основания $A_1B_1C_1D_1E_1$ на другое основание, — высотой усечённой пирамиды.
Усечённая пирамида называется правильной, если она составляет часть правильной пирамиды.


\subsection*{Свойства граней и диагоналей параллелепипеда}

\paragraph{}\label{1938/s72}
\mbox{\so{Теорема}.}
\textbf{\emph{В параллелепипеде:}}

1) \textbf{\emph{противоположные грани равны и параллельны;}}

2) \textbf{\emph{все четыре диагонали пересекаются в одной точке и делятся в ней пополам.}}

1) Грани (рис.~\ref{1938/s-ris-80}) $BB_1C_1C$ и $AA_1D_1D$ параллельны, потому что две пересекающиеся прямые $BB_1$ и $B_1C_1$ одной грани параллельны двум пересекающимся прямым $AA_1$ и $A_1D_1$ другой (§~\ref{1938/s15});
эти грани и равны, так как $B_1C_1=A_1D_1$, $B_1B=A_1A$ (как противоположные стороны параллелограммов) и $\angle BB_1C_1=\angle AA_1D_1$.

\begin{wrapfigure}[35]{o}{40 mm}
\vskip-0mm
\centering
\includegraphics{mppics/s-ris-80}
\caption{}\label{1938/s-ris-80}
\bigskip
\includegraphics{mppics/s-ris-81}
\caption{}\label{1938/s-ris-81}
\bigskip
\includegraphics{mppics/s-ris-82}
\caption{}\label{1938/s-ris-82}
\end{wrapfigure}

2) Возьмём (рис.~\ref{1938/s-ris-81}) какие-нибудь две диагонали, например $AC_1$ и $BD_1$, и проведём вспомогательные прямые $AD_1$ и $BC_1$.
Так как рёбра $AB$ и $D_1C_1$ соответственно равны и параллельны ребру $DC$, то они равны и параллельны между собой;
вследствие этого фигура $AD_1C_1B$ есть параллелограмм, в котором прямые $C_1A$ и $BD_1$ — диагонали, а в параллелограмме диагонали делятся в точке пересечения пополам.

Возьмём теперь одну из этих диагоналей, например $AC_1$, с третьей диагональю, положим, с $B_1D$.
Совершенно так же мы можем доказать, что они делятся в точке пересечения пополам.
Следовательно, диагонали $B_1D$ и $AC_1$ и диагонали $AC_1$ и $BD_1$ (которые мы раньше брали) пересекаются в одной и той же точке, а именно в середине диагонали $AC_1$.
Наконец, взяв эту же диагональ $AC_1$ с четвёртой диагональю $A_1C$, мы также докажем, что они делятся пополам.
Значит, точка пересечения и этой пары диагоналей лежит в середине диагонали $AC_1$.
Таким образом, все четыре диагонали параллелепипеда пересекаются в одной и той же точке и делятся этой точкой пополам.

\paragraph{}\label{1938/s73}
\mbox{\so{Теорема}.}
\textbf{\emph{В прямоугольном параллелепипеде квадрат любой диагонали}} ($AC_1$, рис.~\ref{1938/s-ris-82}) \textbf{\emph{равен сумме квадратов трёх его измерений.}}

Проведя диагональ основания $AC$, получим треугольники $AC_1C$ и $ACB$.
Оба они прямоугольные: первый потому, что параллелепипед прямой и, следовательно, ребро $CC_1$ перпендикулярно к основанию;
второй потому, что параллелепипед прямоугольный и, значит, в основании его лежит прямоугольник.
Из этих треугольников находим:
\begin{align*}
AC_1^2 &= AC^2 + CC_1^2
\intertext{и}
AC^2 &= AB^2 + BC^2.
\end{align*}
Следовательно,
\begin{align*}
AC_1^2 &= AB^2 + BC^2 + CC_1^2 = 
\\
&=AB^2 + AD^2 + AA_1^2.
\end{align*}

\medskip

\so{Следствие}.
\emph{В прямоугольном параллелепипеде все диагонали равны.}

\subsection*{Свойства параллельных сечений в пирамиде}

\paragraph{}\label{1938/s74}
\so{Теоремы}.
\textbf{\emph{Если пирамида}} (рис.~\ref{1938/s-ris-83}) \textbf{\emph{пересечена плоскостью, параллельной основанию, то:}}

1) \textbf{\emph{боковые рёбра и высота делятся этой плоскостью на пропорциональные части;}}

2) \textbf{\emph{в сечении получается многоугольник}} ($abcde$), \textbf{\emph{подобный основанию}} ($ABCDE$);

3) \textbf{\emph{площади сечения и основания относятся, как квадраты их расстояний от вершины.}}

1) Прямые $ab$ и $AB$ можно рассматривать как линии пересечения двух параллельных плоскостей (основания и секущей) третьей плоскостью $ASB$;
поэтому $ab\parallel AB$ (§~\ref{1938/s16}).
По этой же причине $bc\parallel BC$, $cd\z\parallel CD\dots$ и $am\parallel AM$;
вследствие этого
\[\frac{Sa}{aA}=\frac{Sb}{bB}=\frac{Sc}{cC}=\dots=\frac{Sm}{mM}.\]


\begin{wrapfigure}{o}{40 mm}
\vskip-0mm
\centering
\includegraphics{mppics/s-ris-83}
\caption{}\label{1938/s-ris-83}
\vskip-0mm
\end{wrapfigure}

2) Из подобия треугольников $ASB$ и $aSb$, затем $BSC$ и $bSc$ и так далее выводим:
\[\frac{AB}{ab}=\frac{BS}{bS};\quad \frac{BS}{bS}=\frac{BC}{bc},\]
откуда
\[\frac{AB}{ab}=\frac{BC}{bc}.\]
Также
\[\frac{BC}{bc}=\frac{CS}{cS};\quad \frac{CS}{cS}=\frac{CD}{cd},\]
откуда
\[\frac{BC}{bc}=\frac{CD}{cd}.\]
Так же докажем пропорциональность остальных сторон многоугольников $ABCDE$ и $abcde$.
Так как, сверх того, у этих многоугольников равны соответственные углы (как образованные параллельными и одинаково направленными сторонами), то они подобны.
Площади подобных многоугольников относятся, как квадраты сходственных сторон;
поэтому
\[\frac{\text{площадь\,}ABCDE}{\text{площадь\,} abcde}=\frac{AB^2}{ab^2}=\left(\frac{AB}{ab}\right)^2.\]
но
\[\frac{AB}{ab}=\frac{AS}{aS}=\frac{MS}{mS},\]
значит
\[\frac{\text{площадь\,}ABCDE}{\text{площадь\,} abcde}=\frac{MS^2}{mS^2}=\left(\frac{MS}{mS}\right)^2.\]

\paragraph{}\label{1938/s75}
\so{Следствие}.
\emph{У правильной усечённой пирамиды верхнее основание есть правильный многоугольник, подобный нижнему основанию, а боковые грани — равные и равнобочные трапеции} (рис.~\ref{1938/s-ris-83}).
Высота любой из этих трапеций называется \rindex{апофема усечённой пирамиды}\textbf{апофемой} правильной усечённой пирамиды.

\paragraph{}\label{1938/s76}
\so{Теорема}.
\textbf{\emph{Если две пирамиды с равными высотами рассечены на одинаковом расстоянии от вершины плоскостями, параллельными основаниям, то площади сечений пропорциональны площадям оснований.}}

\begin{figure}[h!]%{r}{50 mm}
\vskip-0mm
\centering
\includegraphics{mppics/s-ris-84}
\caption{}\label{1938/s-ris-84}
\vskip-0mm
\end{figure}

Пусть (рис.~\ref{1938/s-ris-84}) $B$ и $B_1$ — площади оснований двух пирамид, $H$ — высота каждой из них, $b$ и $b_1$ — площади сечений плоскостями, параллельными основаниям и удалёнными от вершин на одно и то же расстояние $h$.

Согласно предыдущей теореме мы будем иметь:
\[\frac{b}{B}=\frac{h^2}{H^2}\quad\text{и}\quad\frac{b_1}{B_1}=\frac{h^2}{H^2},\]
откуда
\[\frac{b}{B}=\frac{b_1}{B_1}\quad\text{и}\quad\frac{b}{b_1}=\frac{B}{B_1}.\]

\paragraph{}\label{1938/s77}
\so{Следствие}.
\emph{Если $B=B_1$, то и $b=b_1$, то есть если у двух пирамид с равными высотами основания равновелики, то равновелики и сечения, равноотстоящие от вершины.}


\subsection*{Боковая поверхность призмы и пирамиды}

\begin{wrapfigure}{r}{53 mm}
\vskip-0mm
\centering
\includegraphics{mppics/s-ris-85}
\caption{}\label{1938/s-ris-85}
\vskip-0mm
\end{wrapfigure}

\paragraph{}\label{1938/s78} \rindex{перпендикулярное сечение}\textbf{Перпендикулярным сечением} (рис.~\ref{1938/s-ris-85}) называется многоугольник $abcde$, получаемый от пересечения боковых граней призмы (или их предложений за основания) с плоскостью, перпендикулярной к боковому ребру.

Заметим, что стороны этого многоугольника перпендикулярны к рёбрам (§§~\ref{1938/s31}, \ref{1938/s24}).

\medskip

\mbox{\so{Теорема}.}
\textbf{\emph{Площадь\footnote{В дальнейшем ради краткости термин «поверхность» может употребляется вместо «площадь поверхности».} боковой поверхности призмы равна произведению периметра перпендикулярного сечения на боковое ребро.}}

{\sloppy

Боковая поверхность призмы состоит из параллелограммов;
в каждом из них за основание можно взять боковое ребро, а за высоту — сторону перпендикулярного сечения.
Поэтому площадь боковой поверхности призмы равна:
\[AA_1\cdot ab+ BB_1\cdot bc+CC_1\cdot cd+DD_1\cdot de+EE_1\cdot ea
=
(ab+ bc+ cd+ de+ ea)\cdot AA_1.\]

}

\paragraph{}\label{1938/s79}
\so{Следствие}.
\emph{Площадь боковой поверхности прямой призмы равна произведению периметра основания на высоту} потому, что в такой призме за перпендикулярное сечение можно взять само основание, а боковое ребро её равно высоте.

\paragraph{}\label{1938/s80}
\so{Теорема}.
\textbf{\emph{Площадь боковой поверхности правильной пирамиды равна произведению периметра основания на половину апофемы.}}

\begin{wrapfigure}{o}{50 mm}
\vskip-0mm
\centering
\includegraphics{mppics/s-ris-86}
\caption{}\label{1938/s-ris-86}
\vskip-0mm
\end{wrapfigure}

Пусть (рис.~\ref{1938/s-ris-86}) $SABCDE$ — правильная пирамида и $SM$ — её апофема.
Площадь боковой поверхности этой пирамиды есть сумма площадей равных равнобедренных треугольников.
Площадь одного из них, например $ASB$, равна $AB\cdot\tfrac12SM$.
Если всего треугольников $n$, то площадь боковой поверхности равна 
\[AB\cdot\tfrac12SM\cdot n= AB\cdot n\cdot\tfrac12SM,\]
где $AB\cdot n$ есть периметр основания, a $SM$ — апофема.

\paragraph{}\label{1938/s81}
\mbox{\so{Теорема}.}
\textbf{\emph{Площадь боковой поверхности правильной усечённой пирамиды равна произведению полусуммы периметров обоих оснований на апофему.}}

Боковая поверхность правильной усечённой пирамиды сотоит из равных трапеций.
Площадь одной трапеции, например $AabB$ (рис.~\ref{1938/s-ris-86}), равна $\tfrac12(AB + ab)\cdot Mm$.
Если число всех трапеций есть $n$, то площадь боковой поверхности равна:
\[\frac{AB+ab}{2}\cdot Mm\cdot n=\frac{AB\cdot n+ab\cdot n}{2}\cdot Mm\]
где $AB\cdot n$ и $ab\cdot n$ — периметры нижнего и верхнего оснований.

\subsection*{Упражнения}



\begin{enumerate}

\item
Высота прямой призмы, основание которой есть правильный треугольник, равна 12 м, сторона основания 3 м.
Вычислить полную поверхность призмы.

\item
Площадь полной поверхности прямоугольного параллелепипеда равна 1714~м$^2$, а неравные стороны основания равны 25~м и 14~м.
Вычислить площадь боковой поверхности и боковое ребро.

\item
В прямоугольном параллелепипеде с квадратным основанием и высотой $h$ проведена секущая плоскость через два противоположных боковых ребра.
Вычислить площадь полной поверхности параллелепипеда, зная, что площадь сечения равна $S$.

\item
Правильная шестиугольная пирамида имеет сторону основания $a$ и высоту $h$.
Вычислить боковое ребро, апофему, площади боковой поверхности и полной поверхности.

\item
Вычислить площадь полной поверхности и высоту треугольной пирамиды, у которой каждое ребро равно $a$.

\item
Правильная шестиугольная пирамида, у которой высота 25~см, а сторона основания 5~см, рассечена плоскостью, параллельной основанию.
Вычислить расстояние этой плоскости от вершины пирамиды, зная, что площадь сечения равна $\tfrac23\sqrt{3}$~см$^2$.

\item
Высота усечённой пирамиды с квадратным основанием равна $h$, сторона нижнего основания $a$, а верхнего $b$.
Найти площадь полной поверхности усечённой пирамиды.

\item
Высота усечённой пирамиды равна 6, а площади оснований 18 и~8.
Пирамида рассечена плоскостью, параллельной основаниям и делящей высоту пополам.
Вычислить площадь сечения.
\end{enumerate}

\section{Объём призмы и пирамиды}

\paragraph{Основные допущения в объёмах.}\label{1938/s82}
Величина части пространства, занимаемого геометрическим телом, называется \rindex{объём}\textbf{объёмом} этого тела.

Мы ставим задачу — выразить эту величину положительным числом.
При этом мы будем руководствоваться следующими исходными положениями.

1) \emph{Равные тела имеют равные объёмы.}

2) \emph{Объём какого-нибудь тела}
(например, каждого параллелепипеда, изображённого на рис.~\ref{1938/s-ris-87}),
\emph{состоящего из частей}
($P$ и $Q$),
\emph{равен сумме объёмов этих частей.}

\begin{wrapfigure}{o}{50 mm}
\vskip-0mm
\centering
\includegraphics{mppics/s-ris-87}
\caption{}\label{1938/s-ris-87}
\vskip-0mm
\end{wrapfigure}

Так как объём измеряется положительным числом, получаем, что 
объём тела (например, одного из тел состоящего из частей $P$ и $Q$ на рис.~\ref{1938/s-ris-87})
больше объёма его любой его части ($P$ и $Q$).
Это свойство можно выразить как \emph{объём части меньше объёма целого}.

Два тела, имеющие одинаковые объёмы, называются \rindex{равновеликие тела}\textbf{равновеликими}.
Заметим, что равновеликие тела (например, две тела на рис.~\ref{1938/s-ris-87}) могут быть неравны, то есть одно из них невозможно совместить с другим.

\paragraph{Единица объёма.}\label{1938/s83}
За единицу объёма берут объём такого куба, у которого каждое ребро равно линейной единице.
Так, употребительны кубические метры (м$^3$), кубические сантиметры (см$^3$) и так далее.

\begin{wrapfigure}{o}{40 mm}
\vskip-0mm
\centering
\includegraphics{mppics/s-ris-91}
\caption{}\label{1938/s-ris-91}
\vskip-0mm
\end{wrapfigure}

Отношение двух кубических единиц разных названий равно третьей степени отношения тех линейных единиц, которые служат рёбрами для этих кубических единиц.
Так, отношение кубического метра к кубическому дециметру равно $10^3$, то есть $1000$.
Поэтому, например, если мы имеем куб с ребром длиной $a$ линейных единиц и другой куб с ребром длиной $3a$ линейных единиц, то отношение их объёмов будет равно $3^3$, то есть $27$, что ясно видно из рис.~\ref{1938/s-ris-91}.

\paragraph{Замечание о числе, измеряющем объём.}\label{1930/366}
Относительно числа, измеряющего данный объём в кубических единицах, можно сделать разъяснение, аналогичное тому, какое было нами приведено в §~\ref{1938/244}
относительно числа, измеряющего данную площадь в квадратных единицах.
Повторим вкратце это разъяснение в применении к объёмам.%
\footnote{Этот материал подробно изложен в статье «Площадь и объём» В. А. Рохлина (Энциклопедия элементарной математики, книга пятая, Геометрия).}

\begin{wrapfigure}{o}{40 mm}
\vskip-0mm
\centering
\includegraphics{mppics/s-ris-392}
\caption{}\label{1914/s-ris-392}
\vskip-0mm
\end{wrapfigure}

Возьмём три взаимно перпендикулярные прямые (рис. \ref{1914/s-ris-392}) $OA$, $OB$ и $OC$ и через каждые две из них проведём плоскость.
Мы получим тогда три взаимно перпендикулярные плоскости $AOC$, $COB$ и $BOA$.
Вообразим теперь три ряда параллельных плоскостей: ряд плоскостей, параллельных плоскости $AOC$ другой ряд плоскостей, параллельных плоскости $BOA$, и третий ряд плоскостей, параллельных плоскости $BOC$.
Допустим, кроме того, что соседние плоскости каждого ряда отстоят одна от другой на одно и то же расстояние, равное какой-нибудь $\tfrac1k$ доле линейной единицы.
Тогда от взаимного пересечения этих трёх рядов плоскостей образуется пространственная сеть кубов, из которых каждый представляет собой $(\tfrac1k)^3$ часть кубической единицы.

Вообразим, что в эту сеть мы поместили то тело, объём которого желаем измерить.
Тогда все кубы сети мы можем подразделить на три рода:
1) кубы, которые расположены полностью внутри тела, 
2) кубы, которые некоторой частью выступают из тела (которые, другими словами, пересекаются поверхностью тела), и  
3) кубы, расположенные полностью вне тела.
Если кубов 1-го рода будет $m$, а 2-го рода $n$, то объём данного тела не может быть меньше $\tfrac{m}{k^3}$ и не может быть больше $\tfrac{m+n}{k^3}$ кубических единиц.
То есть, эти два числа будут приближённые меры данного объёма с точностью до $\tfrac{n}{k^3}$ кубических единиц, первое число с недостатком, второе — с избытком.
Уменьшая всё более и более расстояние между параллельными плоскостями, мы будем заполнять пространство всё меньшими и меньшими кубами и можем получать приближённые результаты измерения всё с большей и большей точностью.
Если точность измерения $\tfrac{n}{k^3}$ стремится к нулю при неограниченном возрастании $k$,
то число $V$ равное общему пределу приближённых мер (с недостатком и с избытком) принимается за точную меру данного объёма. 

Доказано, что такое число $V$ существует для всякого многогранника и что оно не зависит от выбора тех трёх прямых $OA$, $OB$ и $OC$ (рис.~\ref{1914/s-ris-392}), которые были взяты для построения пространственной сети кубов.
(То же выполняется для тел ограниченных кривыми поверхностями — цилиндров, конусов, шаров, шаровых сегментов и шаровых слоёв, о которых речь пойдёт далее.)
Более того, число это обладает свойствами указанными в §~\ref{1938/s82}.

\subsection*{Объём параллелепипеда}

\paragraph{}\label{1938/s84}
\so{Теорема}.
\textbf{\emph{Объём прямоугольного параллелепипеда равен произведению трёх его измерений.}}

\begin{wrapfigure}{o}{35 mm}
\vskip-0mm
\centering
\includegraphics{mppics/s-ris-88}
\caption{}\label{1938/s-ris-88}
\vskip-0mm
\end{wrapfigure}

В таком кратком выражении теорему эту надо понимать так: число, выражающее объём прямоугольного параллелепипеда в кубической единице, равно произведению чисел, выражающих три его измерения в соответствующей линейной единице, то есть в единице, являющейся ребром куба, объём которого принят за кубическую единицу.
Так, если $x$ есть число, выражающее объём прямоугольного параллелепипеда в кубических сантиметрах, и $a$, $b$ и с — числа, выражающие три его измерения в линейных сантиметрах, то теорема утверждает, что $x=abc$.
При доказательстве рассмотрим особо следующие три случая:


1) Измерения выражаются \so{целыми числами}.

Пусть, например, измерения, будут (рис.~\ref{1938/s-ris-88}) $AB=a$, $BC=b$ и $BD=c$, где $a$, $b$ и $c$ — какие-нибудь целые числа (например, как изображено у нас на рисунке: $a=4$, $b=2$ и $c=5$).
Тогда основание параллелепипеда содержит $ab$ таких квадратов, из которых каждый представляет собой соответствующую квадратную единицу.
На каждом из этих квадратов, очевидно, можно поместить по одной кубической единице.
Тогда получится слой (изображённый на рисунке), состоящий из $ab$ кубических единиц.
Так как высота этого слоя равна одной линейной единице, а высота всего параллелепипеда содержит $c$ таких единиц, то внутри параллелепипеда можно поместить $c$ таких слоёв.
Следовательно, объём этого параллелепипеда равен $abc$ кубических единиц.

2) Измерения выражаются \so{дробными числами}.

Пусть измерения параллелепипеда будут:
\[\frac mn, \frac pq, \frac rs\]
(некоторые из этих дробей могут равняться целому числу).
Приведя дроби к одинаковому знаменателю, будем иметь:
\[\frac {mqs}{nqs}, \frac {pns}{qns}, \frac {rnq}{snq}.\]

Примем $\frac 1{nqs}$ долю линейной единицы за новую (вспомогательную) единицу длины.
Тогда в этой новой единице измерения данного параллелепипеда выразятся целыми числами, а именно: $mqs$, $pns$ и $rnq$
и потому по доказанному (в случае 1) объём параллелепипеда равен произведению
\[(mqs)\cdot (pns)\cdot (rnq),\]
 если измерять этот объём новой кубической единицей, соответствующей новой линейной единице.
Таких кубических единиц в одной кубической единице, соответствующей прежней линейной единице, содержится $(nqs)^3$; значит, новая кубическая единица составляет $\tfrac1{(nqs)^3}$ прежней.
Поэтому раллелепипеда, выраженный в прежних единицах равен
\begin{align*}\frac1{(nqs)^3}\cdot(mqs)\cdot (pns)\cdot (rnq)&=\frac{mqs}{nqs}\cdot \frac{pns}{nqs}\cdot \frac{rnq}{nqs}= 
\\&=\frac mn\cdot \frac pq\cdot \frac rs.
\end{align*}


3) Измерения выражаются \so{иррациональными числами}.

Пусть у данного параллелепипеда (рис.~\ref{1938/s-ris-89}), который для краткости мы обозначим одной буквой $Q$, измерения будут:
\[AB=\alpha,\quad AC=\beta,\quad AD=\gamma,\]
где все числа $\alpha$, $\beta$ и $\gamma$ или только некоторые из них иррациональные.

\begin{wrapfigure}{o}{60 mm}
\vskip-0mm
\centering
\includegraphics{mppics/s-ris-89}
\caption{}\label{1938/s-ris-89}
\vskip-0mm
\end{wrapfigure}

Каждое из чисел $a$, $\beta$ и $\gamma$ может быть представлено в виде бесконечной десятичной дроби.
Возьмём приближённые значения этих дробей с $n$ десятичными знаками сначала с недостатком, а затем с избытком.
Значения с недостатком обозначим $\alpha_n$, $\beta_n$, $\gamma_n$, значения с избытком $\alpha_n'$, $\beta_n'$, $\gamma_n'$.
Отложим на ребре $AB$, начиная от точки $A$, два отрезка $AB_1 = \alpha_n$ и $AB_2=\alpha_n'$.
На ребре $AC$ от той же точки $A$ отложим отрезки $AC_1=\beta_n$ и $AC_2=\beta_n'$ и на ребре $AD$ от той же точки — отрезки $AD_1=\gamma_n$ и $AD_2=\gamma_n'$.
При этом мы будем иметь
\begin{align*}
AB_1&\le AB<AB_2;
\\
AC_1&\le AC<AC_2;
\\ 
AD_1&\le AD<AD_2.
\end{align*}

Построим теперь два вспомогательных параллелепипеда: один (обозначим его $Q_1$) с измерениями $AB_1$, $AC_1$ и $AD_1$ и другой (обозначим его $Q_2$) с измерениями $AB_2$, $AC_2$ и $AD_2$.
Параллелепипед $Q_1$ будет весь помещаться внутри параллелепипеда $Q$, а параллелепипед $Q_2$ будет содержать внутри себя параллелепипед $Q$.

По доказанному (в случае 2) будем иметь:
\[\text{объём}\, Q_1 = \alpha_n\beta_n\gamma_n, \]
\[\text{объём}\, Q_2 = \alpha_n'\beta_n'\gamma_n'.\]

Поскольку $Q_1$ лежит в $Q$, а $Q$ лежит в $Q_2$, выполняется следующее двойное неравенство
\[\text{объём}\, Q_1 < \text{объём}\, Q <\text{объём}\, Q_2,\]
или 
\[\alpha_n\beta_n\gamma_n < \text{объём}\, Q <\alpha_n'\beta_n'\gamma_n'.\]
Это двойное неравенство остаётся верным при всякой степени точности, с которою мы находим приближённые значения чисел $\alpha$, $\beta$ и $\gamma$.
Значит, неравенство это мы можем высказать так: \emph{число, измеряющее объём данного параллелепипеда, должно быть больше произведения любых приближенных значений чисел $\alpha$, $\beta$ и $\gamma$, если эти значения взяты с недостатком, но меньше произведения любых приближенных значений тех же чисел, если эти значения взяты с избытком.}
Такое число, как известно из алгебры, равно произведению $\alpha\beta\gamma$.
Значит, и в этом случае
\[\text{объём}\, Q=\alpha\beta\gamma.\]

\paragraph{}\label{1938/s85}
\so{Следствие}.
Пусть измерения прямоугольного параллелепипеда, служащие сторонами его основания, выражаются числами $a$ и $b$, а третье измерение (высота) — числом $c$.
Тогда, обозначая объём его в соответствующих кубических единицах буквой $V$, можем написать
\[V = abc.\]
Так как произведение $ab$ выражает площадь основания, то можно сказать, что \emph{объём, прямоугольного параллелепипеда равен произведению площади основания на высоту.}

\paragraph{}\label{1938/s86}
\so{Лемма}.
\textbf{\emph{Наклонная призма равновелика такой прямой призме, основание которой равно перпендикулярному сечению наклонной призмы, а высота — её боковому ребру.}}

Пусть дана наклонная призма $ABCDEA_1B_1C_1D_1E_1$ (рис.~\ref{1938/s-ris-92}).
Продолжим все её боковые рёбра и боковые грани в одном направлении.

Возьмём на продолжении одного какого-нибудь ребра произвольную точку $a$ и проведём через неё перпендикулярное сечение $abcde$.
Затем, отложив $aa_1=AA_1$, проведём через $a_1$ перпендикулярное сечение $a_1b_1c_1d_1e_1$.
Так как плоскости обоих сечений параллельны, то 
\[bb_1=cc_1=dd_1=ee_1=aa_1=AA_1\]
(§~\ref{1938/s17}).
Вследствие этого многогранник $abcdea_1b_1c_1d_1e_1$, у которого за основания приняты проведённые нами сечения, есть прямая призма, о которой говорится в теореме.

\begin{wrapfigure}{o}{45 mm}
\vskip-4mm
\centering
\includegraphics{mppics/s-ris-92}
\caption{}\label{1938/s-ris-92}
\vskip-0mm
\end{wrapfigure}

Докажем, что данная наклонная призма равновелика этой прямой.
Для этого предварительно убедимся, что многогранники $aD$ и $a_1D_1$ равны.
Основания их $abcde$ и $a_1b_1c_1d_1e_1$ равны как основание призмы $ad_1$;
с другой стороны, прибавив к обеим частям равенства 
\[A_1A=a_1a\] по одному и тому же отрезку прямой $A_1a$, получим 
\[aA=a_1A_1.\] 
Подобно этому получим $bB=b_1B_1$, $cC\z=c_1C_1$ и так далее.
Вообразим теперь, что многогранник $aD$ вложен в многогранник $a_1D_1$ так, что основания их совпали;
тогда боковые рёбра, будучи перпендикулярны к основаниям и соответственно равны, также совпадут;
поэтому многогранник $aD$ совместится с многогранником $a_1D_1$;
значит, эти тела равны.

Теперь заметим, что если к прямой призме $a_1d$ добавим многогранник $aD$,
a к наклонной призме $A_1D$ добавим многогранник $a_1D_1$, равный $aD$, то получим один и тот же многогранник $a_1D$.
Из этого следует, что две призмы $A_1D$ и $a_1d$ равновелики.

\paragraph{}\label{1938/s87}
\so{Теорема}.
\textbf{\emph{Объём, параллелепипеда равен произведению площади основания на высоту.}}

Ранее мы доказали эту теорему для параллелепипеда прямоугольного, теперь докажем её для параллелепипеда прямого, а потом и наклонного.

\begin{wrapfigure}{O}{60 mm}
\vskip-0mm
\centering
\includegraphics{mppics/s-ris-93}
\caption{}\label{1938/s-ris-93}
\vskip-0mm
\end{wrapfigure}

1) Пусть (рис.~\ref{1938/s-ris-93}) $AC_1$ — прямой параллелепипед, то есть такой, у которого основание $ABCD$ — какой-нибудь параллелограмм, а все боковые грани — прямоугольники.
Возьмём в нём за основание боковую грань $AA_1B_1B$;
тогда параллелепипед будет наклонный.
Рассматривая его как частный случай наклонной призмы, мы на основании леммы предыдущего параграфа можем утверждать, что этот параллелепипед равновелик такому прямому параллелепипеду, у которого основание есть перпендикулярное сечение $MNPQ$, а высота $BC$.

Четырёхугольник $MNPQ$ — прямоугольник, потому что его углы служат линейными углами прямых двугранных углов;
поэтому прямой параллелепипед, имеющий основанием прямоугольник $MNPQ$, должен быть прямоугольным и, следовательно, его объём равен произведению трёх его измерений, за которые можно принять отрезки $MN$, $MQ$ и $BC$.
Таким образом,
\[\text{объём}\, AC_1 = MN\cdot MQ\cdot BC = MN\cdot (MQ\cdot BC).\]
Но произведение $MQ\cdot BC$ выражает площадь параллелограмма $ABCD$, поэтому
\[
\text{объём}\, AC_1
= (\text{площади}\, ABCD)\cdot MN
= (\text{площади}\, ABCD)\cdot BB_1.
\]

\begin{figure}[h!]%{o}{60 mm}
\centering
\includegraphics{mppics/s-ris-94}
\caption{}\label{1938/s-ris-94}
\vskip-0mm
\end{figure}

2) Пусть (рис.~\ref{1938/s-ris-94}) $AC_1$ — наклонный параллелепипед.
Он равновелик такому прямому, у которого основанием служит перпендикулярное сечение $MNPQ$ (то есть перпендикулярное к рёбрам $AD$, $BC,\dots$), а высотой — ребро $BC$.
Но, по доказанному, объём прямого параллелепипеда равен произведению площади основания на высоту;
значит,
\[\text{объём}\, AC_1 = (\text{площади}\, MNPQ)\cdot BC.\]

Если $RS$ есть высота сечения $MNPQ$, то площадь $MNPQ$ равна $MQ\cdot RS$, поэтому
\[\text{объём}\, AC_1 = MQ\cdot RS\cdot BC = (BC\cdot MQ)\cdot RS.\]
Произведение $BC\cdot MQ$ выражает площадь параллелограмма $ABCD$;
следовательно, 
\[\text{объём}\, AC_1 =(\text{площади}\, ABCD) \cdot RS.\]

Остаётся доказать, что отрезок $RS$ представляет собой высоту параллелепипеда.
Действительно, сечение $MNPQ$, будучи перпендикулярно к рёбрам $BC$, $B_1C_1,\dots$, должно быть перпендикулярно к граням $ABCD$, $BB_1C_1C,\dots$, проходящим через эти рёбра (§~\ref{1938/s43}).
Поэтому если мы из точки $S$ восставим перпендикуляр к плоскости $ABCD$, то он должен лежать весь в плоскости $MNPQ$ (§~\ref{1938/s44}) и, следовательно, должен слиться с прямой $SR$, лежащей в этой плоскости и перпендикулярной к $MQ$.
Значит, отрезок $SR$ есть высота параллелепипеда.
Таким образом, объём и наклонного параллелепипеда равен произведению площади основания на высоту.

\medskip

\so{Следствие}.
Если $V$, $B$ и $H$ — числа, выражающие в соответствующих единицах объём, площадь основания и высоту параллелепипеда, то можно написать:
\[V = B\cdot H.\]

\subsection*{Объём призмы}

\paragraph{}\label{1938/s88}
\so{Теорема}.
\textbf{\emph{Объём, призмы равен произведению площади основания на высоту.}}

Сначала докажем эту теорему для треугольной призмы, а потом и для многоугольной.

\begin{wrapfigure}{o}{45 mm}
\vskip-4mm
\centering
\includegraphics{mppics/s-ris-95}
\caption{}\label{1938/s-ris-95}
\vskip-0mm
\end{wrapfigure}

{\sloppy

1) Проведём через ребро $AA_1$ треугольной призмы $ABCA_1B_1C_1$ (рис.~\ref{1938/s-ris-95}) плоскость, параллельную грани $BB_1C_1C$, а через ребро $CC_1$ — плоскость, параллельную грани $AA_1B_1B$;
затем продолжим плоскости обоих оснований призмы до пересечения с проведёнными плоскостями.
Тогда мы получим параллелепипед $BD_1$, который диагональной плоскостью $AA_1C_1C$ делится на две треугольные призмы (одна из них данная).

}

Докажем, что эти призмы равновелики.
Для этого проведём перпендикулярное сечение $abcd$.
В сечении получится параллелограмм, который диагональю $ac$ делится на два равных треугольника.
Данная призма равновелика такой прямой призме, у которой основание есть $\triangle abc$, а высота — ребро $AA_1$ (§~\ref{1938/s86}).
Другая треугольная призма равновелика такой прямой, у которой основание есть $\triangle adc$, а высота — ребро $AA_1$.
Но две прямые призмы с равными основаниями и равными высотами равны (потому что при вложении они совмещаются), значит, призмы $ABCA_1B_1C_1$ и $ADCA_1D_1C_1$ равновелики.
Из этого следует, что объём данной призмы составляет половину объёма параллелепипеда $BD_1$;
поэтому, обозначив высоту призмы через $H$, получим:
\begin{align*}
\text{объём треугольной призмы}
&=
\frac{(\text{площади}\,ABCD)\cdot H}2=
\\
&=
\frac{(\text{площади}\,ABCD)}2\cdot H=
\\
&=
(\text{площади}\,ABC)\cdot H.
\end{align*}

\begin{wrapfigure}[12]{o}{30 mm}
\vskip0mm
\centering
\includegraphics{mppics/s-ris-96}
\caption{}\label{1938/s-ris-96}
\vskip-0mm
\end{wrapfigure}

2) Проведём через ребро $AA_1$ многоугольной призмы (рис.~\ref{1938/s-ris-96}) диагональные плоскости $AA_1C_1C$ и $AA_1D_1D$.
Тогда данная призма рассечётся на несколько треугольных призм.
Сумма объёмов этих призм составляет искомый объём.
Если обозначим площади их оснований через $b_1,b_2,b_3$, а общую высоту через $H$, то получим: 
\begin{align*}
&\text{объём многоугольной призмы}=
\\
&= b_1\cdot H + b_2 \cdot H+ b_3\cdot H=
\\
&=(b_1+ b_2+b_3)\cdot H = 
\\
&=(\text{площади}\, ABCDE)\cdot H.
\end{align*}

\mbox{\so{Следствие}.}
Если $V$, $B$ и $H$ будут числа, выражающие в соответствующих единицах объём, площадь основания и высоту призмы, то, по доказанному, можно написать:
\[V = B\cdot H.\]



\subsection*{Объём пирамиды}

\paragraph{}\label{1938/s90}
\so{Лемма}.
\textbf{\emph{Треугольные пирамиды с равновеликими основаниями и равными высотами равновелики.}}

Доказательство наше будет состоять из трёх частей.
В первой части мы докажем равновеликость не самих пирамид, а вспомогательных тел, составленных из ряда треугольных призм, поставленных друг на друга.
Во второй части мы докажем, что объёмы этих вспомогательных тел при увеличении числа составляющих их призм приближаются к объёмам пирамид как угодно близко.
Наконец, в третьей части мы убедимся, что сами пирамиды должны быть равновелики.

\begin{figure}[h!]
\vskip-0mm
\centering
\includegraphics{mppics/s-ris-99}
\caption{}\label{1938/s-ris-99}
\vskip-0mm
\end{figure}

I.
Вообразим, что пирамиды поставлены основаниями на некоторую плоскость (как изображено на рис.~\ref{1938/s-ris-99}), тогда их вершины будут находиться на одной прямой, параллельной плоскости оснований, и высота пирамид может быть изображена одним и тем же отрезком прямой $H$.
Разделим эту высоту на какое-нибудь целое число $n$ равных частей (например, на 4, как это указано на рис.~\ref{1938/s-ris-99}) и через точки деления проведём ряд плоскостей, параллельных плоскости оснований.
Плоскости эти, пересекаясь с пирамидами, дают в сечениях ряд треугольников, причём треугольники пирамиды $S$ будут равновелики соответствующим треугольникам пирамиды $S_1$ (§~\ref{1938/s77}).
Поставим внутри каждой пирамиды ряд таких призм, чтобы верхними основаниями у них были треугольники сечений, боковые рёбра были параллельны ребру $SA$ в одной пирамиде и ребру $S_1A_1$ в другой, а высота каждой призмы равнялась бы $H/n$.
Таких призм в каждой пирамиде окажется $n-1$;
они образуют некоторое ступенчатое тело, объём которого, очевидно, меньше объёма той пирамиды, в которой призмы построены.
Обозначим объёмы призм пирамиды $S$ по порядку, начиная от вершины буквами $p_1,p_2,p_3,\dots,p_{n-1}$, а объёмы призм пирамиды $S_1$ — также по порядку от вершины буквами $q_1, q_2, q_3, \dots ,q_{n-1}$.
Тогда, принимая во внимание, что у каждой пары соответствующих призм (у $p_1$ и $q_1$, у $p_2$ и $q_2$ и так далее) основания равновелики и высоты равны, мы можем написать ряд равенств на их объёмы: 
\[p_1=q_1,
\quad
p_2=q_2,
\quad
p_3=q_3,
\quad \dots,\quad p_{n-1}=q_{n-1}.
\]
Сложив все равенства почленно, получим:
\[p_1+p_2+p_3+\dots+p_{n-1}=q_1+q_2+q_3+\dots+q_{n-1}\eqno(1)\]
Мы доказали, таким образом, что объёмы построенных нами вспомогательных ступенчатых тел равны между собой (при всяком числе $n$, на которое мы делим высоту $H$).

II. Обозначив объёмы пирамид $S$ и $S_1$ соответственно буквами $V$
и $V_1$, положим, что
\[V - (p_1+p_2+p_3 + \dots+ p_{n-1}) = x\] 
и
\[V - (q_1+q_2+q_3 + \dots+ q_{n-1}) = y,\] 
откуда
\[p_1+p_2+p_3 + \dots+ p_{n-1} = V - x\] 
и
\[q_1+q_2+q_3 + \dots+ q_{n-1} = V - y,\]

Тогда равенство (1) мы можем записать так:
\[V-x=V_1-y.\eqno(2)\]

Предположим теперь, что число $n$ равных частей, на которое мы делим высоту $H$, неограниченно возрастает;
например, предположим, что, вместо того чтобы делить высоту на 4 равные части, мы разделим её на 8 равных частей, потом на 16, на 32 и так далее, и пусть каждый раз мы строим указанным образом ступенчатые тела в обеих пирамидах.
Как бы ни возросло число призм, составляющих ступенчатые тела, равенство (1), а следовательно, и равенство (2) остаются в силе.
При этом объёмы $V$ и $V_1$, конечно, не будут изменяться, тогда как величины $x$ и $y$, показывающие, на сколько объёмы пирамид превосходят объёмы соответствующих ступенчатых тел, будут, очевидно, всё более и более уменьшаться.
Докажем, что величины $x$ и $y$ могут сделаться как угодно малы (другими словами, что они стремятся к нулю).
Это достаточно доказать для какой-нибудь одной из двух величин $x$ и $y$, например для~$x$.

\begin{wrapfigure}{o}{43 mm}
\vskip-0mm
\centering
\includegraphics{mppics/s-ris-100}
\caption{}\label{1938/s-ris-100}
\vskip-0mm
\end{wrapfigure}

С этой целью построим для пирамиды $S$ (рис.~\ref{1938/s-ris-100}) ещё другой ряд призм, который составит тоже ступенчатое тело, но по объёму большее пирамиды.
Призмы эти мы построим так же, как строили внутренние призмы, с той только разницей, что треугольники сечении мы теперь примем не за верхние основания призм, а за нижние.
Вследствие этого мы получим ряд призм, которые некоторой своей частью будут выступать из пирамид наружу, и потому они образуют новое ступенчатое тело с объёмом, б\'{о}льшим, чем объём пирамиды.
Таких призм будет теперь не $n-1$, как внутренних призм, а $n$.
Обозначим их объёмы по порядку, начиная от вершины, буквами: $p_1',p_2',p_3',\dots,p_n'$.
Рассматривая чертёж, мы легко заметим, что
\[p_1'=p_1,\quad p_2'=p_2,\quad p_3'=p_3,\quad\dots, \quad p_{n-1}'=p_{n-1}.\]
Поэтому
\[(p_1'+p_2'+p_3'+\dots+p_{n-1}'+p_n')-(p_1+p_2+p_3+\dots+p_{n-1})=p_n'\]
Так как
\[p_1'+p_2'+p_3'+\dots+p_{n-1}'+p_n'>V,\]
а
\[p_1+p_2+p_3+\dots+p_{n-1}<V,\]
то
\[V-(p_1+p_2+p_3+\dots+p_{n-1})<p_n'\]
то есть
\[x<p_n'.\]

Но 
\[p_n'=\text{площади}\, ABC\cdot \frac Hn\]
(если $ABC$ есть основание); поэтому
\[x<(\text{площади}\, ABC)\cdot \frac Hn.\]

При неограниченном возрастании числа $n$ величина $\frac Hn$, очевидно, может быть сделана как угодно малой (стремится к нулю).
Поэтому и произведение ($\text{площадь}\, ABC\cdot \frac Hn$), в котором множимое не изменяется, а множитель стремится к нулю, тоже стремится к нулю, и так как положительная величина $x$ меньше этого произведения, то она и подавно стремится к нулю.

То же самое рассуждение можно было бы повторить и о величине~$y$.

Мы доказали, таким образом, что при неограниченном увеличении числа призм объёмы вспомогательных ступенчатых тел приближаются к объёмам соответствующих пирамид как угодно близко.

III.
Заметив это, возьмём написанное выше равенство (2) и придадим ему такой вид: 
\[V - V_1 = x - y.\]
Докажем теперь, что это равенство возможно только тогда, когда $V\z=V_1$ и $x=y$.
Действительно, разность $V-V_1$, как всякая разность постоянных величин, должна равняться постоянной величине, разность же $x-y$, как всякая разность между переменными величинами, стремящимися к нулю, должна или равняться некоторой переменной величине (стремящейся к нулю), или равняться нулю.
Так как постоянная величина не может равняться переменной, то из двух возможностей надо оставить только одну: разность $x-y=0$;
но тогда $V=V_1$ и $x=y$.

Мы доказали, таким образом, что рассматриваемые пирамиды равновелики.

\medskip

\so{Замечание}
Необходимость столь сложного доказательства этой леммы объясняется тем фактом, что два равновеликих многогранника нельзя так легко преобразовывать один в другой, как это можно было делать с равновеликими многоугольниками на плоскости.
А именно, если даны два равновеликих многогранника, то в общем случае оказывается невозможным разбить один из них на такие части из которых можно было бы составить другой.
В частности, это невозможно для двух произвольных треугольных пирамид с равновеликими основаниями и равными высотами.
Эти утверждения были строго доказаны немецким математиком Максом Деном.\footnote{Доказательство приводится в книге «Третья проблема Гильберта» В. Г. Болтянского.}

Метод использованный нами в доказательстве называется \so{методом исчерпывания}; он является основным методом нахождения объёмов.
Этот метод лежит в основе понятия интеграла, которое изучается в курсе алгебры и начал анализа.

\paragraph{Принцип Кавальери.}\label{1938/s89} 
Следующее предложение высказано итальянским математиком Кавальери:
\emph{Если два тела} (ограниченные плоскостями или кривыми поверхностями — всё равно) \emph{могут быть помещены в такое положение, при котором всякая плоскость, параллельная какой-нибудь данной плоскости либо даёт в сечении с ними равновеликие фигуры либо вовсе не пересекает оба тела, то объёмы таких тел одинаковы.}

\begin{wrapfigure}{o}{60 mm}
\vskip-0mm
\centering
\includegraphics{mppics/s-ris-97}
\caption{}\label{1938/s-ris-97}
\bigskip
\includegraphics{mppics/s-ris-101}
\caption{}\label{1938/s-ris-101}%нумерация
\vskip-0mm
\end{wrapfigure}

Это предложение может быть строго доказано тем же методом которым мы воспользовались в §~\ref{1938/s90}.
Но это доказательство сложнее и потому мы ограничимся только формулировкой.

Принцип Кавальери можно проверить на отдельных простых примерах.
Условиям принципа удовлетворяют, например, две прямые призмы (треугольные или многоугольные — всё равно) с равновеликими основаниями и равными высотами (рис.~\ref{1938/s-ris-97}).
Такие призмы, как мы знаем, равновелики.
Вместе с тем, если поставим такие призмы основаниями на какую-нибудь плоскость, то всякая плоскость, параллельная основаниям и пересекающая одну из призм, пересечёт и другую, причём в сечениях получатся равновеликие фигуры, так как фигуры эти равны основаниям, а основания равновелики.
Значит, принцип Кавальери подтверждается в этом частном случае.

Условиям принципа удовлетворяют также пирамиды с равновеликими основаниями и разными высотами.
Действительно, вообразим, что две такие пирамиды поставлены основаниями на какую-нибудь плоскость $P$ (рис.~\ref{1938/s-ris-101}), тогда всякая секущая плоскость $Q$, параллельная $P$, даёт в сечении с пирамидами равновеликие треугольники (§~\ref{1938/s77});
следовательно, пирамиды эти удовлетворяют условиям принципа Кавальери, и потому объёмы их должны быть одинаковы.
Заметим, что мы привели доказательство леммы в §~\ref{1938/s90} используя принцип Кавальери;
однако это доказательство нельзя считать строгим, так как принцип Кавальери нами не был доказан.

В §~\ref{1938/s147} будет представлено ещё один пример применения принципа Кавальери.

\begin{wrapfigure}{o}{55 mm}
\vskip-0mm
\centering
\includegraphics{mppics/s-ris-98}
\caption{}\label{1938/s-ris-98}
\vskip-0mm
\end{wrapfigure}

В применении к площадям для фигур на плоскости, также выполняется принцип Кавальери, а именно: 
\emph{если две фигуры могут быть помещены в такое положение, что всякая прямая, параллельная какой-нибудь данной прямой, либо даёт в сечении с ними равные отрезки либо не пересекает обе фигуры, то такие фигуры равновелики.}
Примером могут служить два параллелограмма или два треугольника с равными основаниями и равными высотами (рис.~\ref{1938/s-ris-98}).

\paragraph{}\label{1938/s91}
\so{Теорема}.
\textbf{\emph{Объём пирамиды равен произведению площади её основания на треть её высоты.}}

Сначала докажем эту теорему для пирамиды треугольной, а затем и многоугольной.


\begin{wrapfigure}{o}{60 mm}
\vskip-0mm
\centering
\includegraphics{mppics/s-ris-102}
\caption{}\label{1938/s-ris-102}
\vskip-0mm
\end{wrapfigure}

1) На основании треугольной пирамиды $SABC$ (рис.~\ref{1938/s-ris-102}) построим такую призму $ABCDES$, у которой высота равна высоте пирамиды, а одно боковое ребро совпадает с ребром $SB$.
Докажем, что объём пирамиды составляет третью часть объёма этой призмы.
Отделим от призмы данную пирамиду.
Тогда останется четырёхугольная пирамида $SADEC$ (которая для ясности изображена отдельно).
Проведём в ней секущую плоскость через вершину $S$ и диагональ основания $DC$.
Получившиеся от этого две треугольные пирамиды имеют общую вершину $S$ и равные основания $\triangle DEC$ и $\triangle DAC$, лежащие в одной плоскости;
значит, согласно доказанной выше лемме пирамиды эти равновелики.
Сравним одну из них, а именно $SDEC$, с данной пирамидой.
За основание пирамиды $SDEC$ можно взять $\triangle SDE$;
тогда вершина её будет в точке $C$ и высота равна высоте данной пирамиды.
Так как $\triangle SDE=\triangle ABC$, то согласно той же лемме пирамиды $SDEC$ и $SABC$ равновелики.

Призма $ABCDES$ нами разбита на три равновеликие пирамиды: $SABC$, $SDEC$ и $SDAC$.
(Такому разбиению, очевидно, можно подвергнуть всякую треугольную призму.) Таким образом, сумма объёмов трёх пирамид, равновеликих данной, составляет объём призмы;
следовательно,
\begin{align*}
\text{объём}\,SABC &= \tfrac13 \text{объёма}\, SDEABC =
\\
&=\frac{(\text{площади}\,ABC)\cdot H}3=
\\
&=(\text{площади}\,ABC)\cdot \frac{H}3,
\end{align*}
где $H$ есть высота пирамиды.

\begin{wrapfigure}{o}{35 mm}
\vskip-8mm
\centering
\includegraphics{mppics/s-ris-103}
\caption{}\label{1938/s-ris-103}
\vskip-0mm
\end{wrapfigure}

2) Через какую-нибудь вершину $E$ (рис.~\ref{1938/s-ris-103}) основания многоугольной пирамиды $SABCDE$ проведём диагонали $EB$ и $EC$.
Затем через ребро $SE$ и каждую из этих диагоналей проведём секущие плоскости.
Тогда многоугольная пирамида разобьётся на несколько треугольных, имеющих высоту, общую с данной пирамидой.
Обозначив площади оснований треугольных пирамид через $b_1$, $b_2$, $b_3$, высоту через $H$, будем иметь:
\begin{align*}
\text{объём}\,SABCDE &= \tfrac13b_1\cdot H + \tfrac13b_2\cdot H + \tfrac13b_3 \cdot H =
\\
&=(b_1+b_2+b_3)\cdot \frac H3=
\\
&=(\text{площади}\,ABCDE)\cdot \frac H3
\end{align*}
\so{Следствие}.
Если $V$, $B$ и $H$ означают числа, выражающие в соответствующих единицах объём, площадь основания и высоту какой угодно пирамиды, то
\[V = \tfrac13 B\cdot H.\]

\paragraph{}\label{1938/s92}
\so{Теорема}.
\textbf{\emph{Объём, усечённой пирамиды равен сумме объёмов трёх пирамид, имеющих высоту, одинаковую с высотой усечённой пирамиды, а основаниями: одна — нижнее основание данной пирамиды, другая — верхнее основание, а площадь основания третьей пирамиды равна среднему геометрическому площадей верхнего и нижнего оснований.}}

\begin{wrapfigure}{o}{38 mm}
\vskip-0mm
\centering
\includegraphics{mppics/s-ris-104}
\caption{}\label{1938/s-ris-104}
\vskip-0mm
\end{wrapfigure}

Пусть площади оснований усечённой пирамиды (рис.~\ref{1938/s-ris-104}) будут $B$ и $b$, высота $H$ и объём $V$ (усечённая пирамида может быть треугольная или многоугольная — всё равно).
Требуется доказать, что
\[V = \tfrac13B\cdot H + \tfrac13 b \cdot H + \tfrac13 H\cdot \sqrt{Bb} = \tfrac13 H(B + b + \sqrt{Bb}),\]
где $\sqrt{Bb}$ есть среднее геометрическое между $B$ и $b$.

Для доказательства на меньшем основании поместим малую пирамиду, дополняющую данную усечённую пирамиду до полной.
Тогда объём усечённой пирамиды $V$ мы можем рассматривать как разность двух объёмов полной пирамиды и верхней дополнительной.

Обозначив высоту дополнительной пирамиды буквой $x$, мы получим, что
\[V = \tfrac13B(H + x)- \tfrac13bx = \tfrac13(BH + Bx-bx) = \tfrac13 [BH + (B - b)x].\]

Для нахождения высоты $x$ воспользуемся теоремой §~\ref{1938/s74}, согласно которой мы можем написать уравнение:
\[\frac Bb=\frac{(H+x)^2}{x^2}.\]

Для упрощения этого уравнения извлечём из обеих частей его квадратный корень:
\[\frac {\sqrt{B}}{\sqrt{b}}=\frac{H+x}{x}.\]

Из этого уравнения (которое можно рассматривать как пропорцию) получим:
\[x\sqrt{B}=(H+x)\sqrt{b},\]
откуда
\[x(\sqrt{B}-\sqrt{b})=H\sqrt{b}\]
и, следовательно,
\[x=\frac{H\sqrt{b}}{\sqrt{B}-\sqrt{b}}.\]
Подставив это выражение в формулу, выведенную нами для объёма $V$, получим:
\[V=\frac13\left[BH+\frac{(B-b)H\sqrt{b}}{\sqrt{B}-\sqrt{b}}\right].\]
Так как $B - b= (\sqrt{B} + \sqrt{b})(\sqrt{B} - \sqrt{b})$, то по сокращении дроби на разность $\sqrt{B} - \sqrt{b}$ получим:
\begin{align*}
V&=\frac13\left[BH+(\sqrt{B}+\sqrt{b})H\sqrt{b}\right]=
\\
&=\frac13\left[BH+H\sqrt{Bb}+bH\right]=
\\
&=\frac13 H\left[B+\sqrt{Bb}+b\right],
\end{align*}
то есть получим ту формулу, которую требовалось доказать.

\medskip

\so{Замечание.}
Если основания усечённой пирамиды являются квадратами со сторонами $x$ и $y$, то согласно доказанной теореме, 
\[V=H(x^2+xy+y^2).\]
Эта формула для объёма квадратной усечённой пирамиды была известна в древнем Египте около 2 тысяч лет до нашей эры; в Московском математическом папирусе рассмотрен пример при $x=2$, $y=4$ и $H=6$.

\section{Правильные многогранники}

Многогранник называется правильным, если все его грани — равные правильные многоугольники и все многогранные углы равны (таков, например, куб).
Из этого определения следует, что в правильных многогранниках равны все плоские углы, все двугранные углы и все рёбра.

\paragraph{Перечисление правильных многогранников.}\label{1938/s97}
Примем во внимание, что в многогранном угле наименьшее число граней три, и что сумма плоских углов выпуклого многогранного угла меньше $360\degree$ (§~\ref{1938/s51}).

\begin{wrapfigure}{o}{35 mm}
\vskip-0mm
\centering
\includegraphics{mppics/s-ris-107}
\caption{}\label{1938/s-ris-107}
\bigskip
\includegraphics{mppics/s-ris-108}
\caption{}\label{1938/s-ris-108}
\bigskip
\includegraphics{mppics/s-ris-109}
\caption{}\label{1938/s-ris-109}
\vskip-0mm
\end{wrapfigure}

Каждый угол правильного треугольника равен $60\degree$.
Если повторим $60\degree$ слагаемым 3, 4 и 5 раз, то получим суммы, меньшие $360\degree$, а если повторим $60\degree$ слагаемым 6 раз или более, то получим в сумме $360\degree$ или более.
Поэтому из плоских углов, равных углам правильного треугольника, можно образовать выпуклые многогранные углы только трёх видов: трёхгранные, четырёхгранные и пятигранные.
Следовательно, если гранями правильного многогранника служат правильные треугольники, то в вершине многогранника могут сходиться или 3 ребра, или 4 ребра, или 5 рёбер.
Соответственно с этим имеется три вида правильных многогранников с треугольными гранями:

1) \rindex{тетраэдр}\textbf{Тетраэдр}, или правильный четырёхгранник, поверхность которого составлена из четырёх правильных треугольников (рис.~\ref{1938/s-ris-107}).
Он имеет 4 грани, 4 вершины и 6 рёбер.

2) \rindex{октаэдр}\textbf{Октаэдр}, или правильный восьмигранник, поверхность которого составлена из восьми правильных треугольников (рис.~\ref{1938/s-ris-108}).
Он имеет 8 граней, 6 вершин и 12 рёбер.

3) \rindex{икосаэдр}\textbf{Икосаэдр}, или правильный 20-гранник, образованный двадцатью правильными треугольниками (рис.~\ref{1938/s-ris-109}).
Он имеет 20 граней, 12 вершин и 30 рёбер.

Угол квадрата равен $90\degree$, а угол правильного пятиугольника равен $108\degree$, повторяя эти углы слагаемым 3 раза, получаем суммы, меньшие $360\degree$, а повторяя их 4 раза или более, получаем $360\degree$ или более.
Поэтому из плоских углов, равных углам квадрата или правильного пятиугольника, можно образовать только трёхгранные углы.

\begin{wrapfigure}{o}{35 mm}
\vskip-0mm
\centering
\includegraphics{mppics/s-ris-110}
\caption{}\label{1938/s-ris-110}
\bigskip
\includegraphics{mppics/s-ris-111}
\caption{}\label{1938/s-ris-111}
\vskip-0mm
\end{wrapfigure}

А поэтому, если гранями многогранника служат квадраты, то в каждой вершине могут сходиться лишь 3 ребра.
Имеется единственный правильный многогранник этого рода — это \rindex{куб}\textbf{куб}, или правильный шестигранник (рис.~\ref{1938/s-ris-110}).
Он имеет 6 граней, 8 вершин и 12 рёбер.

Если гранями правильного многогранника служат правильные пятиугольники, то в каждой вершине могут сходиться лишь 3 ребра.

Существует единственный правильный многогранник этого рода — правильный 12-гранник, или \rindex{додекаэдр}\textbf{додекаэдр}.
Он имеет 12 граней, 20 вершин и 30 рёбер (рис.~\ref{1938/s-ris-111}).

Угол правильного шестиугольника равен $120\degree$, поэтому из таких углов нельзя образовать даже трёхгранного угла.
Из углов правильных многоугольников, имеющих более 6 сторон, подавно нельзя образовать никакого выпуклого многогранного угла.

Отсюда следует, что гранями правильного многогранника могут служить лишь правильные треугольники, квадраты и правильные пятиугольники.

Таким образом, всего может существовать лишь пять видов правильных многогранников, указанных выше.

\paragraph{Построение правильных многогранников.}\label{1938/s98}
Изложенные выше рассуждения о возможных видах правильных многогранников доказывают, что может существовать не более пяти видов правильных многогранников.

Но из этих рассуждений ещё не вытекает, что все эти пять видов правильных многогранников действительно существуют, то есть что можно проведением плоскостей в пространстве осуществить построение каждого из этих пяти правильных многогранников.
Чтобы убедиться в существовании всех правильных многогранников, достаточно указать способ построения каждого из них.

Способ построения куба указать весьма легко.
Действительно, берём произвольную плоскость $P$ и в ней какой-либо квадрат;
через стороны этого квадрата проводим плоскости, перпендикулярные к плоскости~$P$.
Таких плоскостей будет четыре.
Далее проводим плоскость $Q$, параллельную $P$ и отстоящую от неё на расстоянии, равном стороне квадрата.
Шесть полученных плоскостей образуют грани куба;
двенадцать прямых — пересечения каждой пары пересекающихся плоскостей — являются рёбрами куба, а восемь точек пересечения каждой тройки пересекающихся плоскостей служат вершинами куба.

\begin{wrapfigure}{o}{35 mm}
\vskip-0mm
\centering
\includegraphics{mppics/s-ris-112}
\caption{}\label{1938/s-ris-112}
\bigskip
\includegraphics{mppics/s-ris-113}
\caption{}\label{1938/s-ris-113}
\bigskip
\includegraphics{mppics/s-ris-502}
\caption{}\label{1938/s-ris-502}
\bigskip
\includegraphics{mppics/s-ris-503}
\caption{}\label{1938/s-ris-503}
\vskip-0mm
\end{wrapfigure}

Умея построить куб, легко найти способ построения всех других правильных многогранников.

\textbf{Построение тетраэдра.}
Пусть дан куб (рис.~\ref{1938/s-ris-112}).
Возьмём какую-нибудь его вершину, например $A$.
В ней сходятся три грани куба, имеющие форму квадратов.
В каждом из этих квадратов берём вершину, противоположную точке $A$.
Пусть это будут вершины куба $B$, $C$ и $D$.

Точки $A$, $B$, $C$ и $D$ служат вершинами тетраэдра.
Действительно, каждый из отрезков $AB$, $BC$, $CD$, $AD$, $BD$ и $AC$, очевидно, служит диагональю одной из граней куба.
А потому все эти отрезки равны между собой.
Отсюда следует, что в треугольной пирамиде с вершиной $A$ и основанием $BCD$ все грани — правильные треугольники; следовательно, эта пирамида является тетраэдром.
Этот тетраэдр вписан в данный куб.

Полезно заметить, что оставшиеся четыре вершины куба служат вершинами другого тетраэдра, равного первому и также вписанного в данный куб.


\textbf{Построение октаэдра.}
Если в данном кубе построить центры всех его граней, то шесть полученных точек служат вершинами октаэдра (рис.~\ref{1938/s-ris-113}).

\textbf{Построение додекаэдра и икосаэдра.}
Если через каждое из 12 рёбер куба провести плоскость, не имеющую с поверхностью куба других общих точек, кроме точек того ребра, через которое она проведена, то полученные 12 плоскостей образуют грани некоторого 12-гранника.
Более подробное изучение формы этого многогранника показывает, что можно так подобрать наклон этих плоскостей к граням куба (рис.~\ref{1938/s-ris-502}), что полученный 12-гранник будет додекаэдром.
При этом куб окажется вписанным в додекаэдр; то есть 8 из 20 вершин додекаэдра являются вершинами куба.

Наконец, если мы умеем построить додекаэдр, то построение икосаэдра не представляет затруднений: центры граней додекаэдра служат вершинами икосаэдра (рис.~\ref{1938/s-ris-503}).

\section{Понятие о симметрии}

\begin{wrapfigure}{r}{35 mm}
\vskip-0mm
\centering
\includegraphics{mppics/s-ris-114}
\caption{}\label{1938/s-ris-114}
\vskip-0mm
\end{wrapfigure}

\paragraph{Центральная симметрия.}\label{1938/s99}
Две фигуры называются центрально симметричными относительно какой-либо точки $O$ пространства, если каждой точке $A$ одной фигуры соответствует в другой фигуре точка $A'$, расположенная на прямой $OA$ по другую сторону от точки $O$, на расстоянии, равном расстоянию точки $A$ от точки $O$ (рис.~\ref{1938/s-ris-114}).
Точка $O$ называется центром симметрии фигур.

Пример таких центрально симметричных фигур в пространстве мы уже встречали (§~\ref{1938/s53}), когда, продолжая за вершину ребра и грани многогранного угла, получали многогранный угол, центрально симметричный данному.
Соответственные отрезки и углы, входящие в состав двух центрально симметричных фигур, равны между собой.
Тем не менее фигуры в целом не могут быть названы равными: их нельзя совместить одну с другой вследствие того, что порядок расположения частей в одной фигуре иной, чем в другой, как это мы видели на примере центрально симметричных многогранных углов.

В отдельных случаях центрально симметричные фигуры могут совмещаться, но при этом будут совпадать несоответственные их части.

\begin{wrapfigure}{o}{40 mm}
\vskip-0mm
\centering
\includegraphics{mppics/s-ris-115}
\caption{}\label{1938/s-ris-115}
\vskip-0mm
\end{wrapfigure}

Например, возьмём прямой трёхгранный угол (рис.~\ref{1938/s-ris-115}) с вершиной в точке $O$ и рёбрами $OX$, $OY$, $OZ$.
Построим ему центрально симметричный угол $OX'Y'Z'$.
Угол $OXYZ$ можно совместить с $OX'Y'Z'$ так, чтобы ребро $OX$ совпало с $OY'$, а ребро $OY$ с $OX'$.
Если же совместить соответственные рёбра $OX$ с $OX'$ и $OY$ с $OY'$, то рёбра $OZ$ и $OZ'$ окажутся направленными в противоположные стороны.

Если центрально симметричные фигуры составляют в совокупности одно геометрическое тело, то говорят, что это геометрическое тело имеет центр симметрии.
Таким образом, если данное тело имеет центр симметрии, то всякой точке, принадлежащей этому телу, соответствует симметричная точка, тоже принадлежащая данному телу.
Из рассмотренных нами геометрических тел центр симметрии имеют, например: 1) параллелепипед, 2) призма, имеющая в основании правильный многоугольник с чётным числом сторон.

Тетраэдр не имеет центра симметрии.

\paragraph{Зеркальная симметрия.}\label{1938/s100}
Две пространственные фигуры называются зеркально симметричными относительно плоскости $P$, если каждой точке $A$ в одной фигуре соответствует в другой точка $A'$, причём отрезок $AA'$ перпендикулярен к плоскости $P$ и в точке пересечения с этой плоскостью делится пополам.

\medskip

\so{Теорема}.
\textbf{\emph{Всякие два соответственных отрезка в двух зеркально симметричных фигурах равны между собой.}}

\begin{wrapfigure}{o}{40 mm}
\vskip-0mm
\centering
\includegraphics{mppics/s-ris-116}
\caption{}\label{1938/s-ris-116}
\vskip-0mm
\end{wrapfigure}

Пусть даны две фигуры, зеркально симметричные относительно плоскости~$P$.
Выделим две какие-нибудь точки $A$ и $B$ первой фигуры, пусть $A'$ и $B'$ — соответствующие им точки второй фигуры (рис.~\ref{1938/s-ris-116}, на рисунке фигуры не изображены).
Пусть далее $C$ — точка пересечения отрезка $AA'$ с плоскостью $P$, $D$ — точка пересечения отрезка $BB'$ с той же плоскостью.
Соединив прямолинейным отрезком точки $C$ и $D$, получим два четырёхугольника $ABDC$ и $A'B'DC$.
Так как \[AC = A'C,\quad BD = B'D\] 
и 
\[\angle ACD = \angle A'CD=90\degree,\quad  \angle BBC = \angle B'DC=90\degree,\] то эти четырёхугольники равны (в чём легко убеждаемся наложением).
Следовательно, $AB=A'B'$.
Из этой теоремы непосредственно вытекает, что соответствующие плоские и двугранные углы двух зеркально симметричных фигур равны между собой.

Тем не менее совместить эти две фигуры одну с другой так, чтобы совместились их соответственные части, невозможно, так как порядок расположения частей в одной фигуре зеркальный тому, который имеет место в другой (это будет доказано ниже, §~\ref{1938/s102}).
Простейшим примером двух зеркально симметричных фигур являются: любой предмет и его отражение в плоском зеркале;
всякая фигура, симметричная со своим зеркальным отражением относительно плоскости зеркала.

Если какое-либо геометрическое тело можно разбить на две части, симметричные относительно некоторой плоскости, то эта плоскость называется плоскостью симметрии данного тела.

Геометрические тела, имеющие плоскость симметрии, чрезвычайно распространены в природе и в обыденной жизни.
Тело человека и животного имеет плоскость симметрии, разделяющую его на правую и левую части.

На этом примере особенно ясно видно, что зеркально симметричные фигуры нельзя совместить.
Так, кисти правой и левой рук симметричны, но совместить их нельзя, что можно видеть хотя бы из того, что одна и та же перчатка не может подходить и к правой и к левой руке.
Большое число предметов домашнего обихода имеет плоскость симметрии: стул, обеденный стол, книжный шкаф, диван и др.
Некоторые, как например обеденный стол, имеют даже не одну, а две плоскости симметрии (рис.~\ref{1938/s-ris-117}).

\begin{wrapfigure}{r}{43 mm}
\vskip-4mm
\centering
\includegraphics{mppics/s-ris-117}
\caption{}\label{1938/s-ris-117}
\vskip-0mm
\end{wrapfigure}

Если, рассматривая предмет, имеющий плоскость симметрии, занять по отношению к нему такое положение, чтобы плоскость симметрии нашего тела, или по крайней мере нашей головы, совпала с плоскостью симметрии самого предмета, то симметричная форма предмета становится особенно заметной.

\paragraph{Осевая симметрия.}\label{1938/s101}
Две фигуры называются симметричными относительно оси $l$ (ось — прямая линия), если каждой точке впервой фигуры соответствует точка $A'$ второй фигуры, так что отрезок $AA'$ перпендикулярен к оси, пересекается с нею и в точке пересечения делится пополам.
Сама прямая $l$ называется осью симметрии второго порядка.

Из этого определения непосредственно следует, что если два геометрических тела, симметричных относительно какой-либо оси, пересечь плоскостью, перпендикулярной к этой оси, то в сечении получатся две плоские фигуры, центрально симметричные относительно точки пересечения плоскости с осью симметрии тел.

Отсюда далее легко вывести, что два тела, симметричных относительно оси, можно совместить одно с другим, вращая одно из них на $180\degree$ вокруг оси симметрии.
В самом деле, вообразим все возможные плоскости, перпендикулярные к оси симметрии.
Каждая такая плоскость, пересекающая оба тела, содержит две фигуры, центрально симметричные относительно точки встречи плоскости с осью симметрии тел.
Если заставить скользить секущую плоскость саму по себе, вращая её вокруг оси симметрии тела на $180\degree$, то первая фигура совпадает со второй.
Это справедливо для любой секущей плоскости.
Вращение же всех сечений тела на $180\degree$ равносильно повороту всего тела на $180\degree$ вокруг оси симметрии.
Отсюда и вытекает справедливость нашего утверждения.

Если после вращения пространственной фигуры вокруг некоторой прямой на $180\degree$ она совпадает сама с собой, то говорят, что фигура имеет эту прямую своею осью симметрии второго порядка.

Название «ось симметрии второго порядка» объясняется тем, что при полном обороте вокруг этой оси тело будет в процессе вращения дважды принимать положение, совпадающее с исходным (считая и исходное).
Примерами геометрических тел, имеющих ось симметрии второго порядка, могут служить:

1) правильная пирамида с чётным числом боковых граней;
осью её симметрии служит её высота;

2) прямоугольный параллелепипед;
он имеет три оси симметрии: прямые, соединяющие центры его противоположных граней;

3) правильная призма с чётным числом боковых граней.
Осью её симметрии служит каждая прямая, соединяющая центры любой пары противоположных граней (боковых граней и двух оснований призмы).
Если число боковых граней призмы равно $2k$, то число таких осей симметрии будет $k+1$.
Кроме того, осью симметрии для такой призмы служит каждая прямая, соединяющая середины её противоположных боковых рёбер.
Таких осей симметрии призма имеет $k$.

Таким образом, правильная $2k$-гранная призма имеет $2k+1$ осей симметрии.

\paragraph{Зависимость между различными видами симметрии в пространстве.}\label{1938/s102}
Между различными видами симметрии в пространстве — осевой, плоскостной и центральной — существует зависимость, выражаемая следующей теоремой.

\medskip

\so{Теорема}.
\textbf{\emph{Если фигура $F$ зеркально симметрична с фигурой $F'$ относительно плоскости $P$ и в то же время центрально симметрична с фигурой $F''$ относительно точки $O$, лежащей в плоскости $P$, то фигуры $F'$ и $F''$ симметричны относительно оси, проходящей через точку $O$ и перпендикулярной к плоскости~$P$.}}

\begin{wrapfigure}{o}{33 mm}
\vskip-0mm
\centering
\includegraphics{mppics/s-ris-118}
\caption{}\label{1938/s-ris-118}
\vskip-0mm
\end{wrapfigure}

Возьмём какую-нибудь точку $A$ фигуры $F$ (рис.~\ref{1938/s-ris-118}).
Ей соответствует точка $A'$ фигуры $F'$ и точка $A''$ фигуры $F''$ (сами фигуры $F$, $F'$ и $F''$ на чертеже не изображены).

Пусть $B$ — точка пересечения отрезка $AA'$ с плоскостью~$P$.
Проведём плоскость через точки $A$, $A'$ и $O$.
Эта плоскость будет перпендикулярна к плоскости $P$, так как проходит через прямую $AA'$, перпендикулярную к этой плоскости.
В плоскости $AA'O$ проведём прямую $OH$, перпендикулярную к $OB$.
Эта прямая $OH$ будет перпендикулярна и к плоскости~$P$.
Пусть далее $C$ — точка пересечения прямых $AA''$ и $OH$.

В треугольнике $AA'A''$ отрезок $BO$ соединяет середины сторон $AA'$ и $AA''$, следовательно, $BO\parallel A'A''$, но $BO\perp OH$, значит, $A'A''\perp OH$.
Далее, так как $O$ — середина стороны $AA''$ и $CO\parallel AA'$, то $A'C = A''C$.
Отсюда заключаем, что точки $A'$ и $A''$ симметричны относительно оси $OH$.
То же самое справедливо и для всех других точек фигуры.
Значит, наша теорема доказана.

Из этой теоремы непосредственно следует, что две зеркально симметричные фигуры не могут быть совмещены так, чтобы совместились их соответственные части.
В самом деле, фигура $F'$ совмещается с $F''$ путём вращения вокруг оси $OH$ на $180\degree$.
Но фигуры $F''$ и $F$ не могут быть совмещены как центрально симметричные, следовательно, фигуры $F$ и $F'$ также не могут быть совмещены.

\begin{wrapfigure}{r}{33 mm}
\vskip-0mm
\centering
\includegraphics{mppics/s-ris-119}
\caption{}\label{1938/s-ris-119}
\vskip-0mm
\end{wrapfigure}

\paragraph{Оси симметрии высших порядков.}\label{1938/s103}
Фигура, имеющая ось симметрии, совмещается сама с собой после поворота вокруг оси симметрии на угол в $180\degree$.
Но возможны случаи, когда фигура приходит к совмещению с исходным положением после поворота вокруг некоторой оси на угол, меньший $180\degree$.
Таким образом, если тело сделает полный оборот вокруг этой оси, то в процессе вращения оно несколько раз совместится со своим первоначальным положением.
Такая ось вращения называется осью симметрии высшего порядка, причём число положений тела, совпадающих с первоначальным, называется порядком оси симметрии.
Эта ось может и не совпадать с осью симметрии второго порядка.
Так, правильная треугольная пирамида не имеет оси симметрии второго порядка, но её высота служит для неё осью симметрии третьего порядка.
В самом деле, после поворота этой пирамиды вокруг высоты на угол в $120\degree$ она совмещается сама с собой (рис.~\ref{1938/s-ris-119}).
При вращении пирамиды вокруг высоты она может занимать три положения, совпадающие с исходным, считая и исходное.
Легко заметить, что всякая ось симметрии чётного порядка есть в то же время ось симметрии второго порядка.

Примеры осей симметрии высших порядков:

1) Правильная $n$-угольная пирамида имеет ось симметрии $n$-го порядка.
Этой осью служит высота пирамиды.

2) Правильная $n$-угольная призма имеет ось симметрии $n$-го порядка.
Этой осью служит прямая, соединяющая центры оснований призмы.

\paragraph{Симметрии куба.}\label{1938/s104}
Как и для всякого параллелепипеда, точка пересечения диагоналей куба есть центр его симметрии.

Куб имеет девять плоскостей симметрии: шесть диагональных плоскостей и три плоскости, проходящие через середины каждой четвёрки его параллельных рёбер.

Куб имеет девять осей симметрии второго порядка: шесть прямых, соединяющих середины его противоположных рёбер, и три прямые, соединяющие центры противоположных граней (рис.~\ref{1938/s-ris-120}).
Эти последние прямые также являются осями симметрии четвёртого порядка.
Кроме того, куб имеет четыре оси симметрии третьего порядка, которые являются его диагоналями.
В самом деле, диагональ куба $AG$, очевидно, одинаково наклонена к рёбрам $AB$, $AD$ и $AE$, а эти рёбра одинаково наклонены одно к другому.
Если соединить точки $B$, $D$ и $E$, то получим правильную треугольную пирамиду $ADBE$, для которой диагональ куба AG служит высотой.
Когда при вращении вокруг высоты эта пирамида будет совмещаться сама с собой, весь куб будет совмещаться со своим исходным положением.
Других осей симметрии, как нетрудно убедиться, куб не имеет.

\begin{wrapfigure}{o}{40 mm}
\vskip-0mm
\centering
\includegraphics{mppics/s-ris-120}
\caption{}\label{1938/s-ris-120}
\vskip-0mm
\end{wrapfigure}

Посмотрим, сколькими различными способами куб может быть совмещён сам с собой.
Выберем пару соседних вершин куба, скажем $A$ и $B$ (рис.~\ref{1938/s-ris-120}).
При совмещении куба вершина $A$ может перейти в любую из 8 вершин, 
а вершина $B$ в любую из трёх её соседей.
Положение этих двух вершин полностью определяет положение остальных.
Значит всего существует $24=8\cdot 3$  способа совмещения куба с самим собой, включая исходное положение.

Заметим, что вращение вокруг обыкновенной оси симметрии даёт одно положение куба, отличное от исходного, при котором куб в целом совмещается сам с собой.
Вращение вокруг оси третьего порядка даёт два таких положения, и вращение вокруг оси четвёртого порядка — три таких положения.
Так как куб имеет шесть осей второго порядка (это обыкновенные оси симметрии), четыре оси третьего порядка и три оси четвёртого порядка, то имеются $6\cdot 1 + 4\cdot 2 + 3\cdot 3 = 23$ положения куба, отличные от исходного, при которых он совмещается сам с собой. 

Легко убедиться непосредственно, что все эти положения отличны одно от другого, а также и от исходного положения куба.
Вместе с исходным положением они составляют все 24 способа совмещения куба с самим собой.


\subsection*{Упражнения}

\begin{enumerate}

\item
Ребро данного куба равно $a$.
Найти ребро другого куба, объём которого вдвое более объёма данного куба.

\medskip

\so{Замечание}.
Эта \so{задача об удвоении куба}, известная с древних времён, легко решается вычислением (а именно $x=\sqrt[3]{2a^3}=a\cdot\sqrt[3]{2}\z=a\cdot1{,}26992 \dots $), но построением с помощью циркуля и линейки она решена быть не может, так как формула для неизвестного содержит радикал третьей степени из числа, не являющегося кубом рационального числа.

\item
Вычислить площадь поверхности и объём прямой призмы, у которой основание — правильный треугольник, вписанный в круг радиуса $r$ = 2 м, а высота равна стороне правильного шестиугольника, описанного около того же круга.

\item
Определить площадь поверхности и объём правильной восьмиугольной призмы, у которой высота $h$ = 6 м, а сторона основания $a$ = 8 см.

\item
Определить площадь боковой поверхности и объём правильной шестиугольной пирамиды, у которой высота равна 1 м, а апофема составляет с высотой угол в $30\degree$.

\item
Вычислить объём треугольной пирамиды, у которой каждое боковое ребро равно $l$, а стороны основания — $a$, $b$ и $c$.

\item
Дан трёхгранный угол $SABC$, у которого все три плоских угла прямые.
На его рёбрах отложены длины: $SA = a$;
$SB = b$ и $SC\z=c$.
Через точки $A$, $B$ и $C$ проведена плоскость.
Определить объём пирамиды $SABC$.

\item
Высота пирамиды равна $h$, а основание — правильный шестиугольник со стороной $a$.
На каком расстоянии $x$ от вершины пирамиды следует провести плоскость, параллельную основанию, чтобы объём образовавшейся усечённой пирамиды равнялся $V$?

\item
Определить объём тетраэдра с ребром $a$.

\item
Определить объём октаэдра с ребром $a$.

\item
Усечённая пирамида, объём которой $V = 1465 \text{см}^3$, имеет основаниями правильные шестиугольники со сторонами: $a = 23 \text{см}$ и $b = 17 \text{см}$.
Вычислить высоту этой пирамиды.

\item
Объём $V$ усечённой пирамиды равен $10{,}5\text{м}^3$, высота $h = \sqrt{3}\text{м}$ и сторона $a$ правильного шестиугольника, служащего нижним основанием, равна 2 м.
Вычислить сторону правильного шестиугольника, служащего верхним основанием.

\item
На каком расстоянии от вершины $S$ пирамиды $SABC$ надо провести плоскость, параллельную основанию, чтобы отношение объёмов частей, на которые рассекается этой плоскостью пирамида, равнялось $m$?

\item
Пирамида с высотой $h$ разделена плоскостями, параллельными основанию, на три части, причём объёмы этих частей находятся в отношении $m : n : p$.
Определить расстояние этих плоскостей до вершины пирамиды.

\item
Разделить усечённую пирамиду плоскостью, параллельной основаниям $B$ и $b$, на две части, чтобы объёмы находились в отношении $m:n$.

\item
Найти центр, оси и плоскости симметрии фигуры, состоящей из плоскости и пересекающей её прямой, не перпендикулярной к этой плоскости.

\so{Ответ}: центр симметрии — точка пересечения прямой с плоскостью;
плоскость симметрии — плоскость, перпендикулярная данной, проходящая через данную прямую;
осью симметрии служит прямая, лежащая в данной плоскости и перпендикулярная к данной прямой.

\item
Найти центр, оси и плоскости симметрии фигуры, состоящей из двух пересекающихся прямых.

\so{Ответ}: фигура имеет три плоскости симметрии и три оси симметрии (указать какие).

\item Доказать, что многогранник не может иметь двух различных центров симметрии.

\item Описать оси симметрии тетраэдра и указать их порядки.

\item Сколькими разными способами можно совместить тетраэдр с собой, включая его исходное положение?
\end{enumerate}
