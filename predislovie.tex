\section*{От редакции}


«Элементарная геометрия» Андрея Петровича Киселёва это один из лучших учебников по геометрии —
многие темы этого учебника просто невозможно объяснить более доходчиво.

Предмет «Геометрия» трактуется сегодня иначе;
во времена Киселёва, геометрия включала в себя начала анализа и даже элементы теории чисел.
Этим учебник Киселёва существенно отличается от современных учебников, и поэтому он может помочь современным школьникам;
например, для них может оказаться полезным геометрическое понимание предела, непрерывности, интеграла, алгоритма Евклида и иррациональных чисел.

Наша цель сделать удобным использование этого учебника сегодня;
мы заменили термины на современные и внесли незначительные уточнения, в основном исторические. 
Мы используем лицензию CC0 (то есть отказываемся от авторских прав); сам учебник Киселёва перешёл в общественное достояние несколько лет назад.
В частности, любой желающий может использовать любую часть этого учебника для написания своего под любой лицензией —
возможно, таким образом удастся продолжить традиции и облегчить создание хороших учебников. 

Автор учебника, Андрей Петрович Киселёв, сочетал в себе хорошее понимание современной ему математики и талант школьного учителя — такое сочетание редко и ценно во все времена.
В учебнике раскрывается несколько тем, которые были новыми к моменту его написания. 
Этот учебник (очевидно) написан на основе учебника Августа Юльевича Давидова, структура взята практически без изменения, но с существенными улучшениями.
Практически весь учебник строится на строгих доказательствах, и при этом Киселёв смог обойтись без зауми;
очевидно, что в этом помог опыт его преподавательской деятельности.
Доказательства в учебнике всегда строятся по наиболее наглядному пути (пусть даже чуть более сложному) — однажды поняв такое доказательство, его уже невозможно забыть. 

На основе этого учебника (прямо или косвенно) были написаны все современные русские учебники по геометрии.
Среди авторов последующих учебников были математики высокого уровня — 
Андрей Николаевич Колмогоров, 
Алексей Васильевич Погорелов, 
Левон Сергеевич Атанасян, 
Александр Данилович Александров, 
Игорь Фёдорович Шарыгин.
Каждый из этих учебников по своему хорош, но каждый проигрывает Киселёву в главном,
предъявляя существенно более высокие требования к преподавателю.

\begin{flushright}
Н. А. Ершов,\\ 
А. М. Петрунин,\\ 
С. Л. Табачников.        
\end{flushright}
\clearpage
