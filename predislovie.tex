\section*{От редакции}


«Элементарная геометрия» Андрея Петровича Киселёва это один из лучших учебников по геометрии ---
многие темы этого учебника просто невозможно объяснить более доходчиво.
Само понятие геометрии трактуется сегодня иначе;
во времена Киселёва, геометрия включала в себя начала анализа и даже элементы теории чисел.
Этим учебник Киселёва существенно отличается от современных учебников, и поэтому он может помочь современным школьникам;
например, для них может оказаться полезным понимание того, что алгоритм Евклида и иррациональные числа изначально понимались геометрически.

Наша цель сделать удобным использование этого учебника сегодня;
мы внесли незначительные уточнения, в основном исторические, заменили термины на современные. 
Мы используем лицензию CC0 (то есть отказываемся от авторских прав); сам учебник Киселёва перешёл в общественное достояние несколько лет назад.
В частности, любой желающий может использовать любую часть этого учебника для написания своего под любой лицензией ---
возможно, таким образом удастся продолжить традиции и улучшить современные учебники. 

Автор учебника, Андрей Петрович Киселёв, сочетал в себе хорошее понимание современной ему математики и талант школьного учителя --- такое сочетание редко и ценно во все времена.
В учебнике раскрывается несколько тем, которые были новыми к моменту его написания. 
Сам учебник продолжает долгую традицию, он (очевидно) написан на основе учебника Августа Юльевича Давидова, структура взята практически без изменения, но с существенными улучшениями.
Практически весь учебник строится на строгих доказательствах, и при этом Киселёв смог обойтись без зауми;
очевидно, что в этом помог опыт его преподавательской деятельности.
Доказательства в учебнике всегда строятся по наиболее наглядному пути (пусть даже чуть более сложному) --- однажды поняв такое доказательство, его уже невозможно забыть.

На основе этого учебника (прямо или косвенно) были написаны все современные русские учебники по геометрии.
Мы не берёмся их сравнивать; по-настоящему учебник может оценить только преподаватель, прочитавший по нему курс (нет смысла даже рассматривать оценки других людей).
Самый близкий по стилю современный учебник написан Игорем Фёдоровичем Шарыгиным.
Шарыгин, как и Киселёв, --- известный математик, который был напрямую вовлечён в преподавание геометрии школьникам, вероятно, это и делает их учебники похожими.

\clearpage
