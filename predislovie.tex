\section*{От редакции издания}


«Элементарная геометрия» Андрея Петровича Киселёва это один из лучших учебников по геометрии;
многие темы этого учебника просто невозможно объяснить более доходчиво.

Наша цель сделать удобным использование этого учебника сегодня;
мы внесли незначительные уточнения, в основном исторические, заменили термины на современные. 
Часто для понимания математических построений полезно понять как к ним пришли исторически,
и наше издание может в этом помочь современным школьникам.

Мы используем лицензию CC0 (то есть отказываемся от авторских прав); сам учебник Киселёва перешёл в общественное достояние несколько лет назад.
В частности, любой желающий может использовать любую часть этого учебника для написания своего под любой лицензией, включая самую злую;
быть может таким образом удастся сохранить и продолжить традиции учебника. 

Автор учебника Андрей Петрович Киселёв, сочетал в себе хорошее понимание современной ему математики и талант школьного учителя --- такое сочетание редко и ценно во все времена.
В учебнике раскрывается несколько тем, которые были новыми к моменту его написания. 
Сам учебник продолжает долгую традицию, он (очевидно) написан на основе учебника Димидова, структура взята практически без изменения, но с существенными улучшениями.
Практически весь учебник строится на строгих доказательствах, и при этом Киселёв смог обойтись без зауми;
очевидно, что в этом помог опыт его преподавательской деятельности.
Доказательства в учебнике, часто строятся по наиболее наглядному пути, пусть даже чуть более сложному --- однажды поняв такое доказательство его уже невозможно забыть.

На основе этого учебника (прямо или косвенно) были написаны все современные русские учебники по геометрии.
Мы не берёмся их сравнивать; по настоящему учебник может оценить только преподаватель прочитавший по нему курс (нет смысла даже рассматривать оценки других людей).
Самый близкий по стилю современный учебник написан Игорем Фёдоровичем Шарыгиным.
Его учебник очень хороший, хотя, по сравнению с учебником Киселёва, он предъявляет более высокие требования к преподавателю.
Шарыгин как и Киселёв --- известныйй математик который был напрямую вовлечён в преподавание геометрии школьникам, вероятно это и делает их учебники похожими.

Конечно само понятие геометрии трактуется сегодня \'{у}же; во времена Киселёва оно включало в себя начала анализа и даже элементы теории чисел.

\clearpage
