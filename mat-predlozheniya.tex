\section{Математические предложения}

\paragraph{Теоремы, аксиомы, определения.}\label{1938/28}
Из того, что было изложено, можно заключить, что некоторые геометрические истины мы считаем вполне очевидными (например, свойства плоскости и прямой в §§~\ref{1938/3} и \ref{1938/4}), а другие устанавливаем путём рассуждений (например, свойства смежных углов в §~\ref{1938/22} и вертикальных в §~\ref{1938/26}).
Такие рассуждения являются в геометрии главным средством обнаружить свойства геометрических фигур.
Поэтому для дальнейшего полезно заранее познакомиться с теми видами рассуждений, которые применяются в геометрии.
Все истины, которые устанавливаются в геометрии, выражаются в виде предложений.

Эти предложения бывают следующих видов.

\textbf{Определения.}
Определениями называют предложения, в которых разъясняется, какой смысл придают тому или другому названию или выражению.
Например, мы уже встречали определения центрального угла, прямого угла и перпендикуляра.

\textbf{Аксиомы.}
Аксиомами называют истины, которые принимаются без доказательства.
Таковы, например, предложения, встречавшиеся нам ранее (§~\ref{1938/4}):
через всякие две точки можно провести прямую и притом только одну;
если две точки прямой лежат в данной плоскости, то и все точки этой прямой лежат в той же плоскости.

Укажем ещё следующие аксиомы, относящиеся ко всякого рода величинам:

если две величины равны порознь одной и той же третьей величине, то они равны и между собой.

если к равным величинам прибавим поровну или от равных величин отнимем поровну, то равенство не нарушится.

если к неравным величинам прибавим поровну или от неравных величин отнимем поровну, то смысл неравенства не изменится, то есть б\'{о}льшая величина останется б\'{о}льшей.

\textbf{Теоремы.}
Теоремами называются предложения, истинность которых обнаруживается только после некоторого рассуждения (доказательства).
Примером могут служить следующие предложения.

если в одном круге или в равных кругах центральные углы равны, то и соответствующие им дуги равны.

если при пересечении двух прямых между собой один из четырёх углов окажется прямой, то и остальные три угла прямые.

\textbf{Следствия.}
Следствиями называются предложения, которые составляют непосредственный вывод из аксиомы или из теоремы.
Например, из аксиомы:
«через две точки можно провести только одну прямую» следует, что «две прямые могут пересечься только в одной точке».

\paragraph{Состав теоремы.}\label{1938/29}
Во всякой теореме можно различить две части:
условие и заключение.
\textbf{Условие} выражает то, что предполагается данным;
\textbf{заключение} — то, что требуется доказать.
Например, в теореме:
«если центральные углы равны, то и соответствующие им дуги равны» условием служит первая часть теоремы:
«если центральные углы равны», а заключением — вторая часть:
«то и соответствующие им дуги равны»;
другими словами, нам дано (нам известно), что центральные углы равны, а требуется доказать, что при этом условии и соответствующие дуги также равны.

Условие и заключение теоремы могут иногда состоять из нескольких отдельных условий и заключений;
например, в теореме:
«если число делится на 2 и на 3, то оно разделится и на 6» условие состоит из двух частей:
«если число делится на 2» и «если число делится на 3».

Полезно заметить, что всякую теорему можно подробно выразить словами так, что её условие будет начинаться словом «если», а заключение — словом «то».
Например, теорему:
«вертикальные углы равны» можно подробнее высказать так:
«если два угла вертикальные, то они равны».

\paragraph{Обратная теорема.}\label{1938/30}
Теоремой, обратной данной теореме, называется такая, в которой условием поставлено заключение (или часть заключения), а заключением — условие (или часть условия) данной теоремы.
Например, следующие две теоремы обратны друг другу.

\medskip

{
\sloppy

\columnratio{0.5}
\setlength{\columnseprule}{.2pt}
\begin{paracol}{2}
\textbf{\emph{Если центральные углы равны, то и соответствующие им дуги равны.}}
\switchcolumn
\textbf{\emph{Если дуги равны, то и соответствующие им центральные углы равны.}}
\end{paracol}

}

\medskip

Если одну из этих теорем назовём \textbf{прямой}, то другую следует назвать обратной.
В этом примере обе теоремы, и прямая, и обратная, оказываются верными.
Но так бывает не всегда.
Например, теорема:
«если два угла вертикальные, то они равны» верна, но обратное предложение:
«если два угла равны, то они вертикальные» неверно.

В самом деле, допустим, что в каком-либо углу проведена его биссектриса (рис.~\ref{1938/ris-13}).
Она разделит данный угол на два меньших угла.
Эти углы будут равны между собой, но они не будут вертикальными.

\paragraph{Противоположная теорема.}\label{1938/31}
Теоремой, противоположной данной теореме, называется такая, условие и заключение которой представляют отрицание условия и заключения данной теоремы.
Например, теореме:
«если сумма цифр делится на 9, то число делится на 9» соответствует такая противоположная:
«если сумма цифр не делится на 9, то число не делится на 9».

И здесь должно заметить, что верность прямой теоремы ещё не служит доказательством верности противоположной:
например, противоположное предложение:
«если каждое слагаемое не делится на одно и то же число, то и сумма не разделится на это число» — неверно, тогда как прямое предложение верно.

\paragraph{Зависимость между теоремами: прямой, обратной и противоположной.}\label{1938/32}
Для лучшего уяснения этой зависимости выразим теоремы сокращённо так (буквой $A$ мы обозначим условие теоремы, а буквой $B$ — её заключение).

1) \textbf{Прямая:}
если есть $A$, то есть и $B$.

2) \textbf{Обратная:}
если есть $B$, то есть и $A$.

3) \textbf{Противоположная прямой:}
если нет $A$, то нет и $B$.

4) \textbf{Противоположная обратной:}
если нет $B$, то нет и $A$.

Рассматривая эти предложения, легко заметить, что первое из них находится в таком же отношении к четвёртому, как второе к третьему, а именно:
предложения первое и четвёртое обратимы одно в другое, равно как второе и третье.
Действительно, из предложения:
«если есть $A$, то есть и $B$» непосредственно следует:
«если нет $B$, то нет и $A$» (так как если бы $A$ было, то, согласно первому предложению, было бы и $B$);
обратно, из предложения:
«если нет $B$, то нет и $A$» выводим:
«если есть $A$, то есть и $B$» (так как если бы $B$ не было, то не было бы и $A$).
Совершенно так же убедимся, что из второго предложения следует третье, и наоборот.

Таким образом, чтобы иметь уверенность в справедливости всех четырёх теорем, нет надобности доказывать каждую из них отдельно, а достаточно ограничиться доказательством только двух:
прямой и обратной, или прямой и противоположной.