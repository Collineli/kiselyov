\section{Замечательные точки треугольника}

\paragraph{}\label{1938/141}
Мы видели, что:

1) \emph{три срединных перпендикуляра к сторонам треугольника пересекаются в одной точке} (которая есть центр описанного круга). 

2) \emph{три биссектрисы углов треугольника пересекаются в одной точке} (которая есть центр вписанного круга).

Следующие две теоремы указывают ещё две замечательные точки треугольника:

3) точку пересечения трёх высот и 

4) точку пересечения трёх медиан.

\paragraph{}\label{1938/142}
\so{Теорема}.
\textbf{\emph{Три высоты треугольника пересекаются в одной точке.}}

\begin{wrapfigure}{o}{50mm}
\centering
\includegraphics{mppics/ris-160}
\caption{}\label{1938/ris-160}
\end{wrapfigure}

Через каждую вершину $\triangle ABC$ (рис.~\ref{1938/ris-160}) проведём прямую, параллельную противоположной стороне его.
Тогда получим вспомогательный $\triangle A_1B_1C_1$, к сторонам которого высоты данного треугольника перпендикулярны.
Так как $C_1B=AC=BA_1$ (как противоположные стороны параллелограммов), то точка $B$ есть середина стороны $A_1C_1$.

Подобно этому убедимся, что $C$ есть середина $A_1B_1$ и $A$ — середина $B_1C_1$.
Таким образом, высоты $AD$, $BE$ и $CF$ являются
срединными перпендикулярами к сторонам $\triangle A_1B_1C_1$ и как мы знаем (§~\ref{1938/104}), пересекаются в одной точке.

{\small
\smallskip
\mbox{\so{Замечание}.}
Точка, в которой пересекаются высоты треугольника, называется его \rindex{ортоцентр}\textbf{ортоцентром}.
}

\begin{wrapfigure}{r}{37mm}
\vskip-4mm
\centering
\includegraphics{mppics/ris-161}
\caption{}\label{1938/ris-161}
\end{wrapfigure}

\paragraph{}\label{1938/143} 
\mbox{\so{Теорема}.}
\textbf{\emph{Три медианы треугольника пересекаются в одной точке;
эта точка отсекает от каждой медианы третью часть, считая от соответствующей стороны.}}

Возьмём в $\triangle ABC$ (рис.~\ref{1938/ris-161}) какие-нибудь две медианы, например $AE$ и $BD$, пересекающиеся в точке $Z$, и докажем, что
\[ZD=\tfrac13 BD\quad\text{и}\quad ZE = \tfrac13 AE.\]

Для этого, разделив $ZA$ и $ZB$ пополам в точках $F$ и $G$, построим четырёхугольник $DEGF$.
Так как отрезок $FG$ соединяет середины двух сторон $\triangle ABZ$, то $FG\parallel AB$ и $FG=\tfrac12 AB$.
Отрезок $DE$ также соединяет середины двух сторон $\triangle ABC$;
поэтому $DE\parallel AB$ и $DE=\tfrac12 AB$.
Отсюда выводим, что $DE \parallel FG$ и $DE=FG$;
следовательно, четырёхугольник $DEGF$ есть параллелограмм (§~\ref{1938/89}), и потому $ZF=ZE$ и $ZD=ZG$.
Отсюда следует, что
\[ ZE=\tfrac13AE\quad\text{и}\quad ZD=\tfrac13 BD.\]

Если теперь возьмём третью медиану с одной из медиан $AE$ или $BD$, то также убедимся, что точка их пересечения отсекает от каждой из них $\tfrac13$ часть, считая от основания;
значит, третья медиана должна пересечься с медианами $AE$ и $BD$ в одной и той же точке $Z$.

Из физики известно, что пересечение медиан треугольника есть его \so{центр тяжести};
он всегда лежит внутри треугольника.

{\small

\subsection*{Упражнения}

\begin{center}
\so{Найти геометрические места}
\end{center}

\begin{enumerate}[noitemsep]


\item
Оснований перпендикуляров, опущенных из данной точки $A$ на прямые, проходящие через другую данную точку $B$.

\item
Середин хорд, проведённых в окружности через данную внутри неё точку.

\end{enumerate}

\begin{center}
\so{Доказать теоремы}
\end{center}

\begin{enumerate}[resume,noitemsep]

\item
Если две окружности касаются, то всякая секущая, проведённая через точку касания, отсекает от окружностей две противолежащие дуги одинакового числа градусов.

\item
Отрезки двух равных хорд, пересекающихся в одной окружности, соответственно равны.

\item
Две окружности пересекаются в точках $A$ и $B$, через $A$ проведена секущая, пересекающая окружности в точках $C$ и $D$;
доказать, что угол $CBD$ есть величина постоянная для всякой секущей, проведённой через точку $A$.

\smallskip
\so{Указание}.
Углы $ACB$ и $ADB$ имеют постоянную величину.

\item
Если через точку касания двух окружностей проведём две секущие и концы их соединим хордами, то эти хорды параллельны.

\item
Если через точку касания двух окружностей проведём внутри них какую-либо секущую, то касательные, проведённые через концы этой секущей, параллельны.

\item
Если основания высот остроугольного треугольника соединим прямыми, то получим новый треугольник, для которого высоты первого треугольника служат биссектрисами. 

\item
На окружности, описанной около равностороннего $\triangle ABC$, взята произвольная точка $M$;
доказать, что наибольший из отрезков $MA$, $MB$, $MC$ равен сумме двух остальных.

\item
Из точки $F$ проведены к окружности две касательные $PA$ и $FB$ и через точку $B$ — диаметр $BC$.
Доказать, что прямые $CA$ и $OP$ параллельны ($O$ — центр окружности).

\item
Через одну из точек пересечения двух окружностей проводят диаметр в каждой из них.
Доказать, что прямая, соединяющая концы этих диаметров, проходит через вторую точку пересечения окружностей.

\item
Диаметр $AB$ и хорда $AC$ образуют угол в $30\degree$.
Через $C$ проведена касательная, пересекающая продолжение $AB$ в точке $D$.
Доказать, что $\triangle ACD$ равнобедренный.

\item
Если около треугольника опишем окружность и из произвольной точки её опустим перпендикуляры на стороны треугольника, то их основания лежат на одной прямой (прямая Симпсона).

\smallskip
\so{Указание}.
Доказательство основывается на свойствах вписанных углов (§~\ref{1938/124}) и углов вписанного четырёхугольника (§~\ref{1938/139}).


\end{enumerate}

\begin{center}
\so{Задачи на построение}
\end{center}

\begin{enumerate}[resume,noitemsep]

\item
На данной бесконечной прямой найти точку, из которой данный отрезок был бы виден под данным углом.

\item
Построить треугольник по основанию, углу при вершине и высоте.

\item
К дуге данного сектора провести такую касательную, чтобы часть её, заключённая между продолженными радиусами (ограничивающими сектор), равнялась данному отрезку (свести эту задачу к предыдущей).

\item
Построить треугольник по основанию, углу при вершине и медиане, проведённой к основанию.

\item
Даны по величине и положению два отрезка $a$ и $b$.
Найти такую точку, из которой отрезок $a$ был бы виден под данным углом $\alpha$ и отрезок $b$ под данным углом $\beta$.

\item
В треугольнике найти точку, из которой его стороны были бы видны под равными углами.
(Не всякий треугольник допускает решение.)

\smallskip
\so{Указание}.
Обратить внимание на то, что каждый из этих углов должен равняться $120\degree$.

\item
Построить треугольник по углу при вершине, высоте и медиане, проведённой к основанию.

\smallskip
\so{Указание}.
Продолжив медиану на равное расстояние и соединив полученную точку с концами основания, рассмотреть образовавшийся параллелограмм.

\item
Построить треугольник, в котором даны:
основание, прилежащий к нему угол и угол, составленный медианой, проведённой из вершины данного угла, и стороной, к которой эта медиана проведена.

\item
Построить параллелограмм по двум его диагоналям и одному углу.

\item
Построить треугольник по основанию, углу при вершине и сумме или разности двух других сторон.

\item
Построить четырёхугольник по двум диагоналям, двум соседним сторонам и углу, образованному остальными двумя сторонами.

\item
Даны три точки $A$, $B$ и $C$.
Провести через $A$ такую прямую, чтобы расстояние между перпендикулярами, опущенными на эту прямую из точек $B$ и $C$, равнялось данному отрезку.

\item
В данный круг вписать треугольник, у которого два угла даны.

\item
Около данного круга описать треугольник, у которого два угла даны.

\item
Построить треугольник по радиусу описанного круга, углу при вершине и высоте.

\item
Вписать в данный круг треугольник, у которого известны:
сумма двух сторон и угол, противолежащий одной из этих сторон.

\item
Вписать в данный круг четырёхугольник, у которого даны сторона и два угла, не прилежащие к этой стороне.

\item
В данный ромб вписать круг.

\item
В равносторонний треугольник вписать три круга, которые попарно касаются друг друга и из которых каждый касается двух сторон треугольника.

\item
Построить четырёхугольник, который можно было бы вписать в окружность, по трём его сторонам и одной диагонали.

\item
Построить ромб по данной стороне и радиусу вписанного круга.

\item
Около данного круга описать равнобедренный прямоугольный треугольник.

\item
Построить равнобедренный треугольник по основанию и радиусу вписанного круга.

\item
Построить треугольник по основанию и двум медианам, исходящим из концов основания.
Указание: смотри §~\ref{1938/143}.

\item
То же по трём медианам.
Указание: смотри §~\ref{1938/143}.

\item
Дана окружность и на ней точки $A$, $B$ и $C$.
Вписать в эту окружность такой треугольник, чтобы его биссектрисы при продолжении встречали окружность в точках $A$, $B$ и $C$.

\item
Та же задача, с заменой биссектрис треугольника его высотами.

\item
Дана окружность и на ней три точки $M$, $N$ и $P$, в которых пересекаются с окружностью (при продолжении) высота, биссектриса и медиана, исходящие из одной вершины вписанного треугольника.
Построить этот треугольник.

\item
На окружности даны две точки $A$ и $B$.
Из этих точек провести две параллельные хорды, сумма которых дана.

\end{enumerate}

\begin{center}
\so{Задачи на вычисление}
\end{center}

\begin{enumerate}[resume,noitemsep]

\item
Вычислить вписанный угол, опирающийся на дугу, равную $\tfrac1{12}$ части окружности.

\item
Круг разделён на два сегмента хордой, делящей окружность на части в отношении 5:7.
Вычислить углы, которые вмещаются этими сегментами.

\item
Две хорды пересекаются под углом в $36\degree\, 15'\, 32''$.
Вычислить в градусах, минутах и секундах две дуги, заключённые между сторонами этого угла и их продолжениями, если одна из этих дуг относится к другой как 2:3.

\item
Угол, составленный двумя касательными, проведёнными из одной точки к окружности, равен $25\degree15'$.
Вычислить дуги, заключённые между точками касания.

\item
Вычислить угол, составленный касательной и хордой, если хорда делит окружность на две части, относящиеся как $3:7$.

\item
Две окружности одинакового радиуса пересекаются под углом $60\degree$;
определить в градусах меньшую из дуг, заключающихся между точками пересечения. (Два возможных решения.)

\smallskip
\so{Примечание}.
Углом двух пересекающихся дуг называется любой угол, составленный двумя касательными, проведёнными к этим дугам из точки пересечения.

\item
Из одного конца диаметра проведена касательная, а из другого — секущая, которая с касательной составляет угол в $20\degree 30'$.
Найти меньшую из дуг, заключённых между касательной и секущей.

\end{enumerate}

}
