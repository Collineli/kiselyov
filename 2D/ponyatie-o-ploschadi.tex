\section{Понятие о площади}

Каждый из нас имеет некоторое представление о площади из повседневной жизни.

Мы займёмся уточнением понятия о площади фигуры и установлением способов её измерения.

\paragraph{Основные допущения о площадях.}\label{1938/243}
Площадь многоугольника это положительное число, удовлетворяющее следующим условиям:

1) площади двух равных многоугольников, должны быть равны между собой; 

\begin{wrapfigure}{o}{42mm}
\centering
\includegraphics{mppics/ris-239}
\caption{}\label{1938/ris-239}
\end{wrapfigure}

2) если данный многоугольник разбит на несколько многоугольников ($M$, $N$, $P$, рис.~\ref{1938/ris-239}), то  площадь всего многоугольника равна сумме площадей отдельных его частей.

3) площадь квадрата со стороной равной линейной единице длины считается равной квадратной единице.


\smallskip
\mbox{\so{Замечания}.} 1) Единица площади зависит от выбора единицы длины и в каждой конкретной задаче мы вольны выбрать удобную единицу длины и соответственно площади, важно только чтобы все измерения были проведены используя только эту единицу.
Например если за линейную единицу взят отрезок длины 1 метр, то квадрат со стороной 1 метр имеет площадь равную 1 квадратный метр.


2) Многоугольники, имеющие равные площади, принято называть \rindex{равновеликие многоугольники}\textbf{равновеликими}.
В силу условия 1, равные многоугольники всегда и равновелики, но равновеликие многоугольники могут быть неравными (как те, которые изображены на рис.~\ref{1938/ris-240}).

{\sloppy
3) Поскольку площади измеряются положительными числами, а сумма двух положительных чисел всегда больше каждого из слагаемых, то из условия 2, мы получаем следующее заключение:
\emph{Площадь любого многоугольника больше площади любого другого многоугольника целиком в нём лежащего.}
Это утверждение иногда формулируется сокращённо как \emph{целое больше части}.

}

\begin{wrapfigure}{o}{60mm}
\centering
\includegraphics{mppics/ris-240}
\caption{}\label{1938/ris-240}
\end{wrapfigure}

4) Положим, что, разбив данный многоугольник $M$ на несколько многоугольников, мы будем переставлять эти части и получать таким образом новые многоугольники (подобно тому, как на рис.~\ref{1938/ris-240} перемещены части $A$ и $B$).
Спрашивается:
нельзя ли путём этих перестановок получить многоугольник $M'$, который мог бы целиком уместиться внутри $M$?
Если бы это оказалось возможным, то поскольку целое больше части, мы получили бы, что
\[\text{площадь}~M'<\text{площади}~M,\]
а при этом, в силу условий 1 и 2, мы получили бы, что
\[\text{площадь}~M'=\text{площади}~M.\]

Эти два утверждения противоречат друг другу.
Значит площадь многоугольников $M$ и $M'$ невозможно было бы определить так, чтобы удовлетворялись все условия.

Таким образом возможность определить площадь для многоугольников вовсе не очевидна.
Впервые обратил внимание на этот вопрос итальянский математик Антонио Де Цольт (1881).
Невозможность указанной выше перестановки частей многоугольника принималась вначале как некоторый постулат, но позднее эта невозможность была строго доказана.\footnote{Это доказательство довольно сложное; оно приводится например в статье «Площадь и объём» В. А. Рохлина (Энциклопедия элементарной математики, книга пятая, Геометрия).}
Используя  это утверждение можно доказать, что каждый многоугольник (а также фигуры из более широкого класса) имеет определённую площадь удовлетворяющую трём условиям выше (§~\ref{extra/kvad-fig}).

Начиная с §~\ref{1938/245}, нас будет интересовать как измерить площадь данного многоугольника принимая без доказательства, то что каждый многоугольники имеет определённую площадь удовлетворяющую трём условиям выше.

\begin{wrapfigure}{r}{22mm}
\centering
\includegraphics{mppics/ris-extra-2}
\caption{}\label{extra/ris-2}
\end{wrapfigure}

\paragraph{Об измерении площади.}\label{1938/244}
Заметим, что единичный квадрат можно разбить на $n^2$ равных квадратов со стороной $\tfrac1n$. 
Поскольку площади равных фигур равны, мы получаем, все эти квадраты имеют одну и ту же площадь, обозначим её за $s$.
Далее из свойства 2 (§~\ref{1938/243}), заключаем, что площадь единичного квадрата равна $n^2\cdot s$.
Поскольку площадь единичного квадрата равна единице (условие 3 в §~\ref{1938/243}) получаем, что площадь квадрата со стороной $\tfrac1n$ равна $\tfrac1{n^2}$.

\begin{wrapfigure}{o}{55mm}
\centering
\includegraphics{mppics/ris-1931-250}
\caption{}\label{1931/ris-250}
\end{wrapfigure}

Допустим, что на многоугольник $M$, площадь которого надо измерить, наложена сеть из единичных квадратов.
По отношению к данному многоугольнику $M$ все квадраты сети можно разбить на
три рода: 
1) внешние квадраты, 
2) внутренние квадраты и 
3) оставшиеся квадраты, то есть те через которые проходит контур многоугольника.
Оставив без внимания внешние квадраты, сосчитаем отдельно квадраты внутренние и квадраты 3-го рода.
Пусть первых окажется $m$, а вторых $n$.
Тогда, очевидно, площадь $M$ будет больше $m$, но меньше $m+n$ квадратных единиц.
Числа $m$ и $m+n$ будут в этом случае приближённые значения измеряемой площади, первое число с недостатком, второе с избытком, причём погрешность этого измерения меньше $n$ квадратных единиц.

Чтобы получить более точный результат, уплотним сеть квадратов, подраздели в каждый из них на более
мелкие квадраты.
Например мы можем разбить каждый единичный квадрат на 100 квадратов со стороной $\tfrac1{10}$.
Тогда мы получим другие приближённые меры площади, причём погрешность будет не больше прежней (так как все квадраты 3-го рода в уплотнённой сети лежат внутри квадратов 3-го рода в изначальной сети).

\paragraph{Квадрируемые фигуры.}\label{extra/kvad-fig}
Построение в предыдущем параграфе, применимо к произвольным ограниченным фигурам, не обязательно многоугольникам.
Если при последовательном уплотнении сети квадратов погрешность измерения стремится к нулю
то фигура называется \textbf{квадрируемой}.
Для квадрируемой фигуры, общий предел приближённых значений площадей с недостатком и избытком принимается равным её площади.
При этом площадь некоторых квадрируемых фигур может равняться нулю;
например несложно видеть, что фигуры состоящая из одной точки квадрируема и её площадь  равна нулю.

Как пример неквадрируемой фигуры, представьте себе фигуру $F$ состоящую из точек единичного квадрата отстоящих от одной из его сторон на рациональное расстояние (такую фигуру невозможно нарисовать). 
При попытке приближённо измерить площадь $F$ с помощью любой сетки, мы увидим, только квадраты рода 1 и 3 — любой квадрат содержащий точки из $F$ будет также содержать точки вне $F$ (лежащие на иррациональном расстоянии от стороны).
При этом погрешность измерения, то есть общая площадь квадратов рода 3 не может быть меньше единицы. 
То есть $F$ не является квадрируемой.

Следующие три утверждения являются ключевыми, при формальном введении понятия площади.
Мы приводим только идею их доказательства; полные доказательства не сложные, но довольно громоздкие.

1) \emph{Любой многоугольник является квадрируемой фигурой};
то есть при последовательном уплотнении сети квадратов погрешность измерения его площади стремится к нулю. \emph{То же верно для любой фигуры ограниченной конечным набором дуг окружностей и отрезков прямых.}
В частности, круги, сектора и сегменты являются кварируемыми фигурами.

Действительно, заметим, что погрешность измерения равна общей площади квадратов сетки прорезаемых контуром фигуры.
При этом сам контур состоит из нескольких отрезков и дуг.
Значит достаточно доказать, что общая площадь $s$ всех квадратов сетки прорезаемых одним отрезком (или дугой) стремится к нулю при неограниченном уплотнении сетки.
Последнее утверждение следует из неравенства, которое мы предлагаем проверить его самостоятельно:
\[s\le 4\cdot (\ell+a)\cdot a,\]
где $\ell$ есть длина отрезка (или дуги), а $a$ — сторона квадрата сетки.

2) \emph{Если фигура квадрируема в данной сетке то она квадрируема и в любой другой сетке.}
Рассмотрим две сети из квадратов равного размера.
Не трудно видеть, что любой квадрат первой сети 
может пересекать не более чем девятью квадратами второй сети.

Отсюда следует, что общая площадь квадратов третьего рода для первой сети  отличаются не более чем в девять раз от общей площади квадратов третьего рода для второй сети.
Действительно, любой квадрат третьего рода содержит точки контура фигуры.
Значит любой квадрат третьего рода второй сети должен пересекаться с каким-то квадратом третьего рода первой сети.
То есть на один квадрат первой сети приходится не более девяти квадратов второй.


Погрешность измерения площади для каждой сети равна общей площади квадратов третьего рода.
Значит погрешности измерений в обоих сетках отличаются друг от друга не более чем в 9 раз.
То же верно и для одинаковых уплотнений сетей.
Следовательно если погрешность измерения для уплотнений первой сети стремится к нулю, то тоже верно и для второй сети. 

3) \emph{Полученное значение площади квадрируемой фигуры не зависит от выбора сетки.}
В противном случае, при неограниченном уплотнении обоих сеток мы получили бы, что
приближённое значение площади взятое с избытком одной секте меньше 
приближённого значения площади взятое с недостатком в другой.
Но эти приближённые значения равны площадям некоторых многоугольников — многоугольника $M_1$ составленного из квадратов 2-го и 3-го рода первой сетки 
и многоугольника $M_2$ составленного из квадратов 2-го рода второй сетки.
Заметим что многоугольник $M_1$ содержит $F$, а многоугольник $M_2$ содержится в $F$.
В частности $M_2$ есть часть $M_1$, а целое должно быть больше части (§~\ref{1938/243}) — противоречие.

% добавил про квадрируемость недоделано
