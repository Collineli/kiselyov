{\small

\section{Об аксиоме параллельных}

\paragraph{Две аксиомы параллельных.}\label{1914/91} 
Принятая нами в §~\ref{1938/75} аксиома параллельных была предложена древнегреческим математиком Проклом (IV век нашей эры) но часто называется \rindex{аксиома!Плейфэра}\textbf{аксиомой Плейфэра} в честь шотландского математика Джона Плейфэра благодаря которому многие узнали эту формулировку.  

Легко показать, что пятый постулат Евклида (§~\ref{1938/78}) и аксиома Плейфэра обратимы одна в другую. 
То есть из аксиомы Плейфэра можно вывести, как логическое следствие, аксиому Евклида (что и сделано в §~\ref{1938/78}) и, обратно, из этого постулата можно логически получить постулат Плейфэра.
Последнее можно выполнить, например, так:

\begin{wrapfigure}[10]{r}{34mm}
\vskip-7mm
\centering
\includegraphics{mppics/ris-1914-88}
\caption{}\label{1914/ris-88}
\end{wrapfigure}

Пусть через точку $E$ (рис. \ref{1914/ris-88}), взятую вне прямой $CD$, проведены какие-нибудь две прямые $AB$ и $A_1B_1$.
Докажем, исходя из постулата Евклида, что эти прямые не могут быть обе параллельны прямой $CD$.
Для этого проведём через $E$ какую-нибудь секущую прямую $MN$;
обозначим внутренние односторонние углы, образуемые этою секущею
с прямыми $CD$ и $AB$, буквами $\alpha$ и $\beta$.
Тогда одно из двух: или сумма $\alpha+\beta$ не равна $180\degree$,
или она равна $180\degree$.
В первом случае согласно постулату Евклида,
прямая $AB$ должна пересечься с $CD$ и, следовательно, она не может быть параллельной $CD$. 
Во втором случае $\alpha+\angle B_1EN\ne 180\degree$, (так как $\angle B_1EN\ne \angle BEN$).
Значит, тогда, согласно тому же постулату, прямая $A_1B_1$ должна пересечься
с $CD$ и, следовательно, эта прямая не может быть параллельной
$CD$.
Таким образом, одна из прямых $AB$ или $A_1B_1$ непременно окажется непараллельной прямой $CD$; следовательно, через одну точку нельзя провести двух различных прямых, параллельных одной и той же прямой.

\paragraph{Другие предложения равносильные аксиоме параллельных.}\label{1914/92}
Есть много других предложений, также логически обратимых с постулатом Евклида (и, следовательно, ему логически равносильных).
Приведём несколько знаменитых примеров:

\emph{Существует по крайней мере один треугольник, у которого сумма углов равна $180\degree$} (французский математик Адриен Мари Лежандр, начало XIX столетия).

\emph{Существуот выпуклый четырёхугольник} (прямоугольник), \emph{у которого все четыре угла прямые} (французский математик Kлод  Kлеро, XVIII столетие).

\emph{Существует треугольник, подобный, но не равный, другому треугольнику} (итальянский математик Джироламо Саккери, начало XVIII столетия).

\emph{Через всякую точку, взятую внутри угла, меньшего $180\degree$, можно
провести прямую, пересекающую обе стороны этого угла} (немецкий
математик Иоганн Фридрих Лоренц, конец XVIII столетия).

Многие делали попытки доказать постулат Евклида
(или какой-нибудь другой, ему равносильный), то есть вывести его,
как логическое следствие, из других аксиом геометрий.
Все эти попытки оказались неудачными: в каждом из таких «доказательств», после подробного разбора его, находили логическую ошибку.

\paragraph{Открытие новой геометрии.}\label{1914/93} Неудачи в поисках доказательств
постулата Евклида привели к мысли, что этот постулат (как и любой ему равносильный) и не может быть выведен из других аксиом геометрии, а представляет собою независимое от них самостоятельное допущение о свойствах пространства.
Сначала эта мысль высказывалась только в частной переписке;
до нас дошло такое письмо 1816 года написанное Карлом Гауссом
и ещё более уверенное письмо 1818 года написано Фердинандом Швейкартом.
Однако эти математики воздерживались от публикации своих результатов, сейчас трудно понять тому истинную причину.

Независимо, те же идеи были развиты Николаем Ивановичем Лобачевским в 
сочинении изданным Казанским университетом в 1836—1838 годах.
Чуть позже, также независимо, те же результаты были опубликованы венгерским математиком Яношем Бояи.

\begin{wrapfigure}[8]{r}{37mm}
\vskip-5mm
\centering
\includegraphics{mppics/ris-1914-89}
\caption{}\label{1914/ris-89}
\end{wrapfigure}

В своём сочинении Лобачевский обнародовал особую геометрию, названую им «воображаемой», а теперь называемой «геометрией Лобачевского».
В её основание положены те же геометрические аксиомы, на которых основана
геометрия Евклида, за исключением только постулата параллельных линий, вместо которого Лобачевский взял следующее допущение:
\emph{через точку, лежащую вне прямой, можно провести бесчисленное множество параллельных этой прямой}.

То есть, он допустил, что если $AB$ (рис. \ref{1914/ris-89}) есть прямая и $C$ какая-нибудь точка вне её, то при этой точке существует некоторый угол $DEC$, обладающий следующим свойствам:
1) всякая прямая, проведённая через $C$ внутри этого угла (например, прямая $CF$) не пересекаются с $AB$,
2) то же верно и для продолжений сторон $DE$ и $EC$ угла,
а при этом 3) всякая прямая, проведённая через $C$ вне этого угла, пересекается с $AB$.

Понятно, что такое допущение отрицает аксиому параллельности Евклида.
Несмотря однако на это отрицание, геометрия Лобачевского представляет собою такую же стройную
систему геометрических теорем как и геометрия Евклида.
Конечно, теоремы геометрии Лобачевского существенно отличаются от теорем геометрии Евклида, но
в ней, как и в геометрии Евклида, не встречается никаких логических противоречий ни теорем с аксиомами, положенными в основание этой геометрии, ни одних теорем с другими теоремами.

Между тем, если бы постулат Евклида мог быть доказан, то есть если бы он представлял собою
некоторое, хотя бы и очень отдалённое, логическое следствие из других геометрических аксиом, то тогда отрицание этого постулата, положенное в основу геометрии вместе с принятием всех других аксиом, непременно привело бы к логически противоречивым следствиям.

Отсутствие таких противоречий в геометрии Лобачевского служит указанием на независимость пятого
постулата Евклида от прочих геометрических аксиом и, следовательно, на невозможность доказать его.
Заметим, однако, что одно только отсутствие противоречий в геометрии Лобачевского ещё не служит доказательством независимости пятого постулата от других аксиом геометрии.
Ведь всегда можно возразить, что это отсутствие противоречий есть только
случайное явление, происходящее, быть может, от того, что в геометрии Лобачевского не сделано ещё достаточного количества выводов, что со временем, быть может, и удастся кому-нибудь получить
такой логический вывод в этой геометрии, который окажется в противоречии с каким-нибудь другим выводом той же геометрии.

Однако доказано следующее: \emph{если бы в геометрии Лобачевского нашлось противоречие, то нашлось бы соответствующее противоречие и в Евклидовой геометрии};
также верно и обратное — \emph{если есть противоречие в Евклидовой геометрии то  было бы  противоречие и в геометрии Лобачевского}.
То есть удаётся доказать, что в логическом смысле геометрия Лобачевского «не хуже и не лучше» геометрии Евклида.


\paragraph{Неевклидовы геометрии.}\label{1914/94} 
Позже немецкий математик Бернхард Риман (1826—1866) построил ещё особую, также лишённую противоречий, геометрию (названной 
потом \textbf{геометрией Римана}),%https://en.wikipedia.org/wiki/Bernhard_Riemann
в которой вместо постулата Евклида принимается допущение, что
через точку, взятую вне прямой, нельзя провести ни одной параллельной этой прямой (другими словами, все прямые плоскости пересекаются).
Такие геометрии (как геометрии Лобачевского и Римана), в которых в основание положено какое-нибудь допущение о параллельных линиях, не согласное с постулатом Евклида, носят общее название
\textbf{неевклидовых геометрий}.

Другой пример неевклидовой геометрии это так называемая \rindex{абсолютная геометри}\textbf{абсолютная геометрия}, независимая от пятого постулата.
Другими словами, в этой геометрии пятый постулат может выполняться, а может и не выполняться.
Эта геометрия как раз и рассматривалась в упомянутом сочинении Яноша Бояи,
она включает геометрии Евклида и Лобачевского как частные случаи.

\paragraph{Теоремы геометрии Лобачевского.}\label{1914/95} Приведём некоторые теоремы геометрии Лобачевского, резко различающиеся от соответствующих теорем геометрии Евклида:

\emph{Два перпендикуляра к одной и той же прямой, по мере удаления
от этой прямой, расходятся неограниченно.}

\emph{Сумма углов треугольника меньше $180\degree$} (в геометрии Римана она
больше $180\degree$), при чём эта сумма не есть величина постоянная для разных треугольников.
\emph{Чем больше площадь треугольника, тем больше сумма его углов разнится от $180\degree$.}



\begin{wrapfigure}[6]{r}{25mm}
\vskip-5mm
\centering
\includegraphics{mppics/ris-extra-5}
\caption{}\label{extra/ris-5}
\end{wrapfigure}

\emph{Если в выпуклом четырёхугольнике три угла прямые, то четвёртый угол острый.}
В частности в этой геометрии нет прямоугольников.
Четырёхугольник с тремя прямыми углами называется \rindex{четырёхугольник Ламберта}\textbf{четырёхугольником Ламберта} в честь шведского математика Иоганна Генриха Ламберта (1728—1777).
Попытка изобразить четырёхугольник Ламберта (рис. \ref{extra/ris-5}) не очень удачная поскольку сам лист бумаги не похож на плоскость Лобачевского. % добавил про четырёхугольник Ламберта


\emph{Если углы одного треугольника соответственно равны углам другого треугольника, то такие треугольники равны} (следовательно, в геометрии Лобачевского не существует подобия).

\emph{Геометрическое место точек плоскости, равноотстоящих от какой-нибудь прямой этой плоскости не является прямой.}

}
