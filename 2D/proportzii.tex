\section{Пропорции} 

\paragraph{Теорема Фалеса}\label{1938/182}
Следующая теорема названа в честь древнегреческого математика Фалеса Милетского (VII—VII век до нашей эры).


\smallskip
\so{Теорема}.
\textbf{\emph{Стороны угла}} ($ABC$), \textbf{\emph{пересекаемые рядом параллельных прямых}} ($DD_1, EE_1, FF_1, \dots$), \textbf{\emph{рассекаются ими на пропорциональные части}} (рис.~\ref{1938/ris-192}).

\begin{figure}[!ht]
\centering
\includegraphics{mppics/ris-192}
\caption{}\label{1938/ris-192}
\end{figure}

Требуется доказать, что
\[\frac{BD}{BD_1}=\frac{DE}{D_1E_1}=\frac{EF}{E_1F_1},\]
или
\begin{align*}
\frac{BD}{DE}&=\frac{BD_1}{D_1E_1},
\\
\frac{DE}{EF}&=\frac{D_1E_1}{E_1F_1}\quad\text{и так далее}
\end{align*}
Проводя вспомогательные прямые $DM$, $EN$ и~т.~д., параллельные $BA$, мы получим треугольники $BDD_1$, $DEM$, $EFN$ и~т.~д., которые все подобны между собой, так как углы у них соответственно равны (вследствие параллельности прямых).
Из их подобия следует:
\[\frac{BD}{BD_1}=\frac{DE}{DM}=\frac{EF}{EN}\quad\text{и так далее}\]
Заменив в этом ряду равных отношений отрезок $DM$ на $D_1E_1$, отрезок $EN$ на $E_1F_1$ и~т.~д.
(противоположные стороны параллелограммов равны), мы получим то, что требовалось доказать.

\paragraph{}\label{1938/183}
\so{Теорема}.
\textbf{\emph{Две параллельные прямые}} ($MN, M_1N_1$, рис. \ref{1938/ris-193}), \textbf{\emph{пересекаемые рядом прямых}} ($OA, OB, OC, \dots$), \textbf{\emph{исходящих из одной и той же точки}} ($O$), \textbf{\emph{рассекаются ими на пропорциональные части.}}


Требуется доказать, что отрезки $AB$, $BC$, $CD,\dots$
прямой $MN$ пропорциональны отрезкам $A_1B_1$, $B_1C_1$, $C_1D_1,\dots$
прямой $M_1N_1$.

\begin{figure}[!ht]
\centering
\includegraphics{mppics/ris-193}
\caption{}\label{1938/ris-193}
\end{figure}

Из подобия треугольников $OAB$ и $O_1A_1B_1$ (§~\ref{1938/159}) и треугольников $OBC$ и $OB_1C_1$ выводим:
\[\frac{AB}{A_1B_1}=\frac{BO}{B_1O}
\quad\text{и}\quad
\frac{BO}{B_1O}=\frac{BC}{B_1C_1},
\]
откуда
\[\frac{AB}{A_1B_1}=\frac{BC}{B_1C_1},
\]
Подобным же образом доказывается пропорциональность и прочих отрезков.

\paragraph{}\label{1938/184}
\so{Задача}.
\emph{Разделить отрезок прямой $AB$ \emph{(рис.~\ref{1938/ris-194})} на три части в отношении $m:n:p$, где $m$, $n$, $p$ — данные отрезки или данные числа.}

\begin{figure}[!ht]
\centering
\includegraphics{mppics/ris-194}
\caption{}\label{1938/ris-194}
\end{figure}

Проведя луч $AC$ под произвольным углом к $AB$, отложим на нём от точки $A$ отрезки, равные отрезкам $m$, $n$ и $p$.
Точку $F$ — конец отрезка $p$ — соединяем с $B$ прямой $BF$ и через концы $G$ и $H$ отложенных отрезков проводим прямые $GD$ и $HE$, параллельные $BF$.
Тогда отрезок $AB$ разделится в точках $D$ и $E$ на части в отношении $m:n:p$.

Если $m$, $n$ и $p$ означают какие-нибудь числа, например 2, 5, 3, то построение выполняется так же, с той лишь разницей, что на $AC$ откладываются отрезки, равные 2, 5 и 3 произвольным единицам длины.

Конечно, указанное построение применимо к делению отрезка не только на три части, но на какое угодно иное число частей.

\paragraph{}\label{1938/185}
\so{Задача}.
\emph{К трём данным отрезкам $a$, $b$ и $c$ найти четвёртый пропорциональный} (рис.~\ref{1938/ris-195}), то есть
найти такой отрезок $x$, который удовлетворял бы пропорции $a:b=c:x$.

\begin{figure}[!ht]
\centering
\includegraphics{mppics/ris-195}
\caption{}\label{1938/ris-195}
\end{figure}

На сторонах произвольного угла $ABC$ откладываем отрезки:
$BD\z=a$, $BF=b$, $DE=c$.
Проведя затем через $D$ и $F$ прямую, построим $EG\parallel DF$.
Отрезок $FG$ будет искомый.

\subsection*{Свойство биссектрисы угла треугольника}

\paragraph{}\label{1938/186}
\so{Теорема}.
\textbf{\emph{Биссектриса}} ($BD$, рис.~\ref{1938/ris-196}) \textbf{\emph{любого угла треугольника}} ($ABC$) \textbf{\emph{делит противоположную сторону на части}} ($AD$ и $CD$), \textbf{\emph{пропорциональные прилежащим сторонам треугольника.}}

\begin{wrapfigure}{r}{30mm}
\vskip-4mm
\centering
\includegraphics{mppics/ris-196}
\caption{}\label{1938/ris-196}
\end{wrapfigure}

Требуется доказать, что если $\angle ABD\z=\angle DBC$, то 
\[\frac{AD}{DC}\z=\frac{AB}{BC}.\]

Проведём $CE \parallel BD$ до пересечения в точке $E$ с продолжением стороны $AB$.
Тогда, согласно теореме в §~\ref{1938/182}, мы будем иметь пропорцию:
\[\frac{AD}{DC}=\frac{AB}{BE}.\]
Чтобы от этой пропорции перейти к той, которую требуется доказать, достаточно обнаружить, что $BE=BC$, то есть что $\triangle BCE$ равнобедренный.
В этом треугольнике $\angle E=\angle ABD$ (как углы, соответственные при параллельных прямых) и $\angle BCE \z= \angle DBC$ (как углы, накрест лежащие при тех же параллельных прямых).
Но $\angle ABD=\angle DBC$ по условию;
значит, $\angle E = \angle BCE$, а потому равны и стороны $BC$ и $BE$, лежащие против равных углов.
Заменив в написанной выше пропорции $BE$ на $BC$, получим ту пропорцию, которую требуется доказать.

\medskip

\smallskip
\so{Численный пример}.
Пусть $AB = 10$,
$BC = 7$ и $AC \z= 6$.
Тогда, обозначив $AD$ буквой $x$, можем написать пропорцию:
\[\frac{x}{6 - x} = \frac{10}7;\]
отсюда найдём:
\begin{align*}
7x&=60-10x;
\\
7x+10x&=60;
\\
17x&=60;
\\
x&=\tfrac{60}{17}=3\tfrac9{17}.
\end{align*}
Следовательно,
\[DC=66-x=6-3\tfrac9{17}=2\tfrac8{17}.\]

{\small

\paragraph{}\label{1938/187}
\so{Теорема} (выражающая свойство биссектрисы внешнего угла треугольника).
\textbf{\emph{Если биссектриса}} ($BD$, рис.~\ref{1938/ris-197}) \textbf{\emph{внешнего угла}} ($CBF$) \textbf{\emph{треугольника}} ($ABC$) \textbf{\emph{пересекает продолжение противоположной стороны}} ($AC$) \textbf{\emph{в некоторой точке}} ($D$), \textbf{\emph{тогда расстояния}} ($AD$ и $DC$) \textbf{\emph{от этой точки до концов этой стороны пропорциональны прилежащим сторонам треугольника}} ($AB$ и $BC$).
Требуется доказать, что если $\angle CBD\z=\angle FBD$, то $DA:DC=AB:BC$.

\begin{wrapfigure}[9]{o}{45mm}
\vskip-4mm
\centering
\includegraphics{mppics/ris-197}
\caption{}\label{1938/ris-197}
\end{wrapfigure}

Проведя $CE \parallel BD$, получим пропорцию
\[\frac{DA}{DC}=\frac{BA}{BE}.\]

Так как $\angle BEC=\angle FBD$ (как соответственные), а $\angle BCE\z=\angle CBD$ (как накрест лежащие при параллельных прямых) и углы $FBD$ и $CBD$ равны по условию, то $\angle BEC\z=\angle BCE$;
значит, $\triangle BCE$ равнобедренный, то есть $BE=BC$.
Подставив в пропорцию вместо $BE$ равный отрезок $BC$, получим ту пропорцию, которую требовалось доказать:
\[\frac{DA}{DC}=\frac{AB}{BC}.\]

{\small 

\smallskip
\so{Примечание}.
Особый случай представляет биссектриса внешнего угла при вершине равнобедренного треугольника, которая параллельна основанию.

}

\paragraph{}\label{1914/227}
\so{Обратная теорема.}
\textbf{\emph{Если прямая, исходящая из вершины треугольника, пересекает противоположную сторону (или её продолжение) в точке, расстояния от которой до концов противоположной стороны пропорциональны соответственно двум другим сторонам, то она есть биссектриса угла треугольника (внутреннего или внешнего).}}


Пусть  $E$ есть точка лежащая на стороне $AC$ треугольника $ABC$ или на её продолжении, такая, что
$AE:EC=AB:BC$.
По доказанному (§~\ref{1938/186}), для основания $D$ биссектрисы угла $B$, а также для основания $D'$ биссектрисы его внешнего угла к $B$ выполнена та же пропорция, то есть
\[\frac{AD}{DC}=\frac{AD'}{D'C}=\frac{AB}{BC}.\]

Но если $AB\ne BC$, то существует только две точки (§~\ref{extra/proportions}).
Значит $E$ совпадает с $D$ или $D'$ и следовательно прямая $BE$ сливается с биссектрисой угла $B$ или его внешнего угла.

Если $AB=BC$, то точка $E$ есть середина основания $AC$ и треугольник $ABC$ равнобедренный. 
То есть $BE$ есть медиана, проведённая к основанию равнобедренного треугольника,
по доказанному (§~\ref{1938/38}), совпадает с биссектрисой угла $B$.
(Биссектриса внешнего угла в этом случае параллельна основанию.)



\paragraph{}\label{1914/228}
\so{Теорема}.
\textbf{\emph{Геометрическое место точек, до которых расстояния от двух данных точек находятся в постоянном отношении $\bm{m:n}$, есть окружность, когда $\bm{m\ne n}$, и прямая, когда $\bm{m=n}$.}}

Обозначим данные точки буквами $A$ и $B$.

Если $m\z=n$, то искомое место точек есть срединный перпендикуляр к отрезку $AB$ (§~\ref{1938/58}).

\begin{wrapfigure}{o}{50mm}
\centering
\includegraphics{mppics/ris-1914-209}
\caption{}\label{1914/ris-209}
\end{wrapfigure}

Предположим, что $m>n$.
Тогда на прямой $AB$, можно найти две точки $C$ и $C'$ (рис. \ref{1914/ris-209}),
принадлежащие искомому геометрическому месту (§~\ref{extra/proportions}); то есть такие, что
\[\frac{CA}{CB}=\frac{C'A}{C'B}=\frac mn.\]
Точка $C$ лежит на отрезке $AB$, а точка $C'$ на его продолжении.

Пусть ещё другая точка $M$ удовлетворяет пропорции
\[\frac{MA}{MB}=\frac mn.\eqno(1)\]
Проведя $MC$ и $MC'$ мы должны заключить (§~\ref{1914/227}), что первая из этих прямых есть биссектриса угла $AMB$, а вторая — биссектриса угла $BMN$;
вследствие этого угол $CMC'$, составленный из двух половин смежных углов, должен быть прямой.
Поэтому вершина $M$ лежит на окружности, описанной на $CC'$ как на диаметре (§~\ref{1938/125}).

Таким образом, мы доказали, что всякая точка $M$ удовлетворяющая пропорции (1), лежит на окружности с диаметром $CC'$.
Докажем обратное предложение, то есть, что всякая точка $M$, лежащая на окружности с диаметром $CC'$, удовлетворяет пропорции (1).
Обозначим центр этой окружности буквой $O$, это середина отрезка $CC'$.

Найдём сначала пропорции $\frac{OA}{OC}$ и $\frac{OC}{OB}$.
Поскольку
\[\frac{AC}{BC}=\frac{AC'}{BC'}=\frac{m}{n}\]
верны также следующие пропорции
\[\frac{AC'+AC}{BC'+BC}=\frac{AC'-AC}{BC'-BC}=\frac{m}{n}. \eqno(2)\]

Поскольку $OC=OC'$,
\begin{align*}
2\cdot AO&=AC'-OC'+AC+OC=
\\
&=AC'+AC.
\\
2\cdot BO&=BC'-OC'+OC-BC=
\\
&=BC'-BC.
\\
2\cdot CO&=CC'=AC'-AC=
\\
&=BC'+BC.
\end{align*}
Подставляя в эти значения в пропорции (2) получаем
\[\frac{OA}{OC}=\frac{OC}{OB}=\frac mn.\]

Пусть $M$ произвольная точка окружности отличная от $C$ и $C'$.
Тогда $OM=OC$, следовательно
\[\frac{OA}{OM}=\frac{OM}{OB}=\frac mn.\]
Значит $\triangle OBM\sim\triangle OMA$ и в частности
\[\frac{AM}{BM}=\frac{OA}{OM}=\frac mn,\]
то есть выполняется пропорция (1).

Случай $m<n$ аналогичен.

\so{Замечание}.
Окружность, о которой говорится в этой теореме, известна под названием \rindex{окружность!Аполлония}\textbf{окружность Аполлония} (Аполлоний Пергский — греческий геометр, живший во II вeке до нашей эры).

}
